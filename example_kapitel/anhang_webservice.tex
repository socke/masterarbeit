%% $Id: anhang_webservice.tex 14 2007-06-01 12:50:36Z stefan $
%\setchapterpreamble[u]{%
%\dictum[John F. Kennedy]{Einen Vorsprung im Leben hat, wer da anpackt, wo die anderen erst einmal reden. \dots}}
\chapter{Web Service} \label{anh:webservice}

\section{Was ist ein Web Service?} \index{Web Service}

Ein Web Service stellt Dienste �ber das Internet, meist �ber HTTP oder FTP, zur Verf�gung. Die Kommunikation wird mit Hilfe von XML \index{XML} Standards realisiert. 

Wikipedia schreibt zur Definition folgendes:

\begin{quotation}
Ein Web Service bzw. Webdienst ist eine Software-Anwendung, die mit einem Uniform Resource Identifier (URI) \index{URI} eindeutig identifizierbar ist und deren Schnittstellen als XML-Artefakte definiert, beschrieben und gefunden werden k�nnen. Ein Web Service unterst�tzt die direkte Interaktion mit anderen Software-Agenten unter Verwendung XML-basierter Nachrichten durch den Austausch �ber internetbasierte Protokolle. \citep[vgl.][]{wikiwebservice}	
\end{quotation}

Folglich ist aber nicht jeder Web Server oder Web Browser ein Web Service, denn die Idee ist, dass die Programme eines Web Services �ber ein XML-gekapseltes Protokoll kommunizieren.


\section{Der Nutzen eines Web Services}

\subsection{Protokolle}\index{Protokolle}

Ein Web Service ist ein XML-gekapseltes Protokoll. Das bedeutet, dass die Informationen als XML-Daten �bertragen werden. Diese XML-Daten werden �ber HTTP als Transportprotokoll gesendet. Abbildung \ref{fig:2007_04_07_Webservice-Protokolle} zeigt die Schichten der einzelnen Protokolle. 

\begin{figure}[htbp]
	\centering
		\includegraphics[width=0.7\textwidth]{bilder/2007_04_07_Webservice-Protokolle.png}
	\caption{Webservice Protokollschichten}
	\label{fig:2007_04_07_Webservice-Protokolle}
\end{figure}

\subsection{SOAP} \index{Web Service!SOAP}

Vor einigen Jahren war SOAP noch die Abk�rzung f�r \enquote{Simple Object Access Protokoll}, wird aber ab Version 1.2 nicht mehr als Akronym gebraucht, da es auch nicht unbedingt \enquote{simple} ist und nicht nur dem Zugriff auf Objekten dient. Weiterhin k�nnen Abk�rzungen in den USA nicht als Markenname registriert werden. 

SOAP ist ein Protokoll f�r Nachrichten, welche zwischen Web Service-Konsument und Web Service-Anbieter ausgetauscht werden. Dieses Protokoll ist XML-basierend, was bedeutet, das die Informationen nach einem bestimmten SOAP-Standard in XML-Format vorliegen. Diese Informationen k�nnen also von Menschen gelesen werden. 

Es gibt zwei Arten von Interaktionsmustern des SOAP-Protokolls. Einmal RPC, das f�r \enquote{Remote Procedure Call} \index{Web Service!RPC|see{Remote Procedure Call}} \index{Web Service!Remote Procedure Call} und zum anderen Document Exchange, also der Austausch von Dokumenten. RPC-kodiertes SOAP enth�lt einen Methodenaufruf und definiert die Datentypen. Document-kodiertes SOAP enth�lt nur die eigentliche Nachricht. 

\subsection{WSDL} \index{Web Service!WSDL}

WSDL steht f�r \enquote{Web Service Description Language} und ist ein Standard f�r die Beschreibung der Daten, die zwischen Konsument und Anbieter ausgetauscht werden.

\section{SOA} \index{SOA|see{Serviceorientierte Architektur}}

SOA steht f�r \enquote{Service orientierte Architektur}. \index{Serviceorientierte Architektur}

Abbildung \ref{fig:2007_04_07_Webservice-SOA} zeigt die serviceorientierte Architektur. Ein Konsument bzw. ein Klient sucht in einem Verzeichnis nach einem Dienst. Von diesem Verzeichnis erh�lt er ein Ergebnis mit der URL des Web Services sowie der Beschreibung des Dienstes. Der Konsument sendet nun einen Anfrage per XML an den Service Anbieter und erh�lt eine Antwort ebenfalls per XML. Der Anbieter stellt seinen Dienst zur Verf�gung. 

\begin{figure}[htbp]
	\centering
		\includegraphics[width=1.00\textwidth]{bilder/2007_04_07_Webservice-SOA.png}
	\caption{Servieorientierte Architektur}
	\label{fig:2007_04_07_Webservice-SOA}
\end{figure}

\section{Implementierung eines Web Services} 

Die Implementierung eines Web Services ist mit Hilfe der PEAR::SOAP Klasse leicht zu realisieren. Das Kapitel \ref{anh:technhandbuchwebservice} des technischen Handbuches zeigt die Implementierung des Web Services der importieren Tickets.

\section{Vorteile}\index{Web Service!Vorteile}

\begin{itemize}
	\item Zur �bertragung dient h�ufig das HTTP-Protokoll, so dass es nicht zu Problemen mit Firewalls kommt. Bei CORBA, DCOM oder Java RMI sieht das anders aus. 
	\item Durch die Verwendung von Internet-Standards ist es m�glich, dass die verschiedensten Systeme miteinander kommunizieren. Windows C\#-Clients k�nnen ohne Probleme mit Java-Servern kommunizieren, selbst mit einer Firewall vor der Anwendung.
\end{itemize}

\section{Nachteile}\index{Web Service!Nachteile}

\begin{itemize}
	\item Die Performance ist h�ufig ein Problem, denn das XML des SOAP-Protokolls beinhaltet sehr viel Overhead. Ein verdrei�igfachen der Gr��e ist durchaus keine Seltenheit. 
	\item Es ist mehr Know-How erforderlich als bei der Verwendung von RPC. Jede Programmiersprache ben�tigt zur Verwendung von Web Services spezielle Bibliotheken.
\end{itemize}


