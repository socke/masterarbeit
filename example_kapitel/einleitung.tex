% $ID$
% Einleitung 
\setchapterpreamble[u]{%
\dictum[Ernst Ferstl]{Verzicht auf �berfl�ssiges bewirkt einen Gewinn an Weit- und �bersicht. \dots}}
\chapter{Einleitung}

Die vollst�ndige Verf�gbarkeit aller Kommunikationsdienste, das ist einer der Hauptziele der Abteilung TBAN des Vodafone D2 Konzerns. Sollten Dienste einmal nicht verf�gbar sein oder kann ein Kunde einen Dienst nicht nutzen, sei es durch Ausfall eines Servers, Routers oder durch eigenes Verschulden des Kunden, so gilt es die m�gliche Ursache schnell und effizient zu finden. In einem gro�en Kommunikationsunternehmen mit weltweit �ber zweihundert Millionen Kunden ist dies nicht trivial. Diverse Applikationen, Hilfstools und Softwarepakete unterst�tzen die Mitarbeiter im t�glichen Arbeitsprozess. Die Verbesserung der Reportingm�glichkeiten, um mehr Informationen �ber Dienste, Hard- und Software zu erhalten, ist in diesem Kontext ein wichtiges Thema. Schnelle Antworten und L�sungen anstatt lange Wartezeiten, das ist das Ziel.

Im Rahmen meines Studiums Informatik mit der Fachrichtung Software Engineering absolviere ich, Stefan Sobek, ein f�nfmonatiges Praktikum, im Zeitraum Februar bis Juni 07, im Vodafone D2 GmbH Konzern im Bereich TBAN\footnote{Technischer Betrieb Assistance and Network}. W�hrend dieser Zeit schreibe ich meine Diplomarbeit zum Thema \enquote{Schnittstellenentwicklung - Datenimport aus TTWOS\footnote{Trouble Ticket System} und ChaMPS\footnote{Change Management System} in AiO}. 

Systeme im Rahmen des ITIL-Prozesses\footnote{IT Infrastructure Library}, wie TTWOS und ChaMPS, enthalten geplante �nderungen (ChaMPS) oder Trouble Tickets (TTWOS). Zur Verbesserung der Fehlerfindung respektive Reportingm�glichkeiten, sollen abteilungsrelevante Daten ins abteilungsinterne Tool AiO importiert und anschlie�end abgeglichen werden. Je nach Eintragsart soll ein Incident oder ein Change Management Eintrag, eventuell sogar beides, in die Datenbank von AiO eingetragen werden. Das Kapitel \ref{Situationsschilderung} beinhaltet eine Schilderung der aktuellen Situation der Abteilung TBAN. Eine genaue Beschreibung des Auftrages ist in Kapitel \ref{Auftragsbeschreibung} zu finden.

Die Umsetzung des Auftrages ben�tigt mehrere Schritte im Rahmen des Software Engineering Prozesses. Die Analyse und Definition der Anforderungen ist der Beginn dieses Prozesses. Hier werden die Anforderungen durch Interviews mit den Benutzern der vorhandenen Systeme TTWOS und ChaMPS, sowie des Systems AiO aufgestellt. Eine Durchf�hrbarkeitsstudie und die Spezifikation sind ein weiterer Teil der Analysephase. Die Machbarkeit und die Validierung der Anforderungen erfolgt mit Hilfe eines Protoypen. Kapitel \ref{analyseunddefinition} beschreibt die Analysephase. 

Sind die Anforderungen aufgestellt und die Spezifikation verfasst, folgt die Entwurfsphase. Mit Hilfe von Diagrammen, die den neuen Prozessablauf aufzeigen, sowie mit UML-Klassendiagrammen wird der Aufbau der Software beschrieben. Im Rahmen des Entwicklungsprozesses finden immer wieder �nderungen am Entwurf statt, selbst wenn diese Phase bereits abgeschlossen ist. Die Entwurfsphase befindet sich in Kapitel \ref{systemsoftwareentwurf}. 

W�hrend der Erstellung des Prototypen konnten bereits viele hilfreiche Erkenntnisse gesammelt und aussagekr�ftige Test gemacht werden. Hierzu z�hlen einzusetzende Techniken und Bibliotheken sowie hilfreiche Frameworks. Dies ist in der Implementierungsphase von gro�em Vorteil. Das Kapitel \ref{implementierung} beschreibt die Implementierungsphase und die Entwicklung des Importers, des Web Services sowie der ben�tigten Benutzeroberfl�chen. Vergleicht man die Prototypenerstellung im Anhang mit dem fertigen Softwareprodukt, so kann man erkennen, dass das grobe Konzept �bernommen wurde.

Die Planung der Diplomarbeit sowie die aufgestellten Teilaufgaben und T�tigkeiten beschreibt Kapitel \ref{planungablaufaufgaben}. Um beispielsweise einen fehlgeschlagenen Import neu zu starten, wurde ein manueller Importer erstellt. Dieser war in der ersten Planung nicht ber�cksichtigt. Lediglich ein automatischer Import war vorgesehen. W�hrend der Entwicklung des Prototypen stellte sich dies allerdings als umst�ndlich heraus, so dass beschlossen wurde eine grafische Benutzeroberfl�che zu erstellen, �ber die ein Importvorgang manuell gestartet werden kann. 

