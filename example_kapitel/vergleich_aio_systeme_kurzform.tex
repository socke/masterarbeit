%% $Id: vergleich_aio_systeme_kurzform.tex 7 2007-04-06 12:57:34Z stefan $
%% Vergleich AiO Systeme 

\chapter{Vergleich AiO mit anderen vorhandenen Systemen}

Dieses Kapitel geh�rt mit zur Analyse und vergleicht AiO\footnote{All in One. Das Change-, Incident- und Configurationmanagement System der Abteilung TBAN} und den darin enthaltenen Applikationen mit den entsprechenden bereits vorhandenen Systemen der Vodafone D2 GmbH. 

Diese Informationen wurde im Rahmen eines Brainstormings, mit den Teamleitern der Gruppe IPAS und IPAN sowie zwei Mitarbeitern dieser Teams, zusammengetragen. 

AiO verf�gt �ber ein Confiuration-, Change-, Incident- und Security Mangement System. Die einzelnen Applikationen werden mit Ihrem Pendant verglichen und es werden fehlende Funktionalit�ten im direkten Vergleich mit AiO aufgef�hrt. 

Dieser Vergleich soll die Vorteile des Systems AiO f�r die Abteilung TBAN darstellen und deutlich machen, �ber welche abstrakten Funktionalit�ten die konzernweiten Softwaresysteme nicht verf�gen.

Vodafone verf�gt �ber die verschiedensten Tools die alle autark voneinander arbeiten. C6000 ist das Configuration Mangement Tool, ChaMPS ist das Change Management Tool, TTWOS ist ein Change- und Incidentmanagement Tool, FNDT gibt Auskunft �ber Netzelemente. 

Dieses Kapitel zeigt auf, welchen Vorteil die abteilungsinterne Applikation AiO f�r die Abteilung TBAN hat. Anstatt m�hsam Daten aus mehreren unterschiedlichen Systemen zusammen zu suchen, verf�gt AiO �ber die M�glichkeit diese �bersichtlich gesammelt anzubieten. AiO verbindet die Daten aus den unterschiedlichsten Bereichen miteinander. Einer der gr��ten Vorteile ist, dass die Verkn�pfung von den Netzwerkelementen zu den Changes und Incidents besteht, und somit eine Art �nderungs- und Fehlerhistorie erstellt werden kann. So ist die Fehlersuche effizient. 