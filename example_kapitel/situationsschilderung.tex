%% $Id: situationsschilderung.tex 14 2007-06-01 12:50:36Z stefan $
%% Situationsschilderung 
\setchapterpreamble[u]{%
\dictum[Ludwig van Beethoven]{Sich selbst darf man nicht f�r so g�ttlich halten, dass man seine eigenen Werke nicht gelegentlich verbessern k�nnte. \dots}}
\chapter{Situationsschilderung}\label{Situationsschilderung}

Die Abteilung TBAN\footnote{Technischer Betrieb Access Network} verf�gt seit mehreren Jahren �ber eine eigene Intranet-Plattform, welche diverse Hilfs- sowie Statistiktools beinhaltet. Dar�ber hinaus wurde vor ca. zwei Jahren dieser Plattform eine Applikation namens AiO hinzugef�gt. AiO bedeutet \enquote{All in One} und beinhaltet ein Change-, Incident- und Configurationmanagement System. Weiterhin besitzt es noch ein Securitymanagement System, sowie ein System zur Erfassung von �bertragungslinien, genannt ION\footnote{Inter Office Network}. 

Der Konzern Vodafone verf�gt bereits �ber ein Change Management System mit Namen \enquote{ChaMPS\footnote{Change Management Process System}} und �ber ein Trouble Ticket System namens \enquote{TTWOS\footnote{Trouble Ticket and Work Order System}}. Diese werden konzernweit eingesetzt und m�ssen deshalb ein gro�es T�tigkeitsfeld abdecken, was sich schlie�lich negativ auf die Handhabung auswirkt. Gerade der Bereich des Reportings und das Finden von Fehlern stellt sich in den beiden Systemen als �u�erst schwierig dar. 

Im Dokument zum Change Management Prozess im technischen Betrieb hei�t es:
\begin{quotation}
Eine Konfigurationsdatenbank, die eine durchg�ngige Risikozuordnung erm�glicht, steht derzeit nicht zur Verf�gung. Die Auswirkungen von �nderungen auf weitere Services sind daher zur Zeit oft nur sehr schwer vollst�ndig zu beurteilen. 
 \citep[S. 5][]{cmbetrieb}	
\end{quotation}

Dadurch wird deutlich, dass es als Basis keine Konfigurationsdatenbank gibt, die eine genaue Zuordnung zu Configuration Items, beispielsweise Hard- oder Softwareelementen, erm�glicht.  
\index{Configuration Items} 

Aus diesem Grund wurde vor ca. zwei Jahren AiO eingef�hrt, um nicht nur geplante �nderungen an Systemen, die eventuell auch Auswirkungen auf andere Systeme haben k�nnten, sondern auch �nderungen an Systemen, die nur abteilungsintern Relevanz haben, zu erfassen. Mit diesen Informationen, die �ber AiO schnell abrufbar sind, lassen sich �nderungen und aufgetretende sowie erfasste Fehler an Systemen nachverfolgen und beheben.

\begin{figure}[h]
	\centering
		\includegraphics[width=1.00\textwidth]{bilder/vodafone_aio_cm.png}
	\caption{AiO Change Management}
	\label{fig:vodafone_aio_cm}
\end{figure}

Geplante �nderungen an Systemen m�ssen nach wie vor in \enquote{ChaMPS} und zus�tzlich noch ins AiO Changemanagement (siehe Abbildung \ref{fig:vodafone_aio_cm}) eingetragen werden. Genauso verh�lt es sich mit Incidents, also aufgetretene und unerwartete Fehler. Diese m�ssen nicht nur in TTWOS, sondern auch noch parallel ins Incident Management von AiO (siehe Abbildung \ref{fig:vodafone_aio_im}) eingetragen werden. 

\begin{figure}[h]
	\centering
		\includegraphics[width=1.00\textwidth]{bilder/vodafone_aio_im.png}
	\caption{AiO Incident Management}
	\label{fig:vodafone_aio_im}
\end{figure}


Um diesen Mehraufwand der doppelten Pflege zu vermeiden und gleichzeitig die Vorteile von AiO, wie �bersichtlichkeit, Reportingm�glichkeiten und Einfachheit weiterhin nutzen zu k�nnen, w�re es sinnvoll, die Daten von TTWOS und von ChaMPS in AiO zur Verf�gung zu haben bzw. eine definierte und anpassbare Schnittstelle zu besitzen.

\glossary{name={TTWOS}, description={Trouble Ticket and Workorder System. Dies ist das  Ticketsystem von Vodafone. Hier�ber werden Fehler- sowie Arbeitsanweisungen verteilt und dokumentiert.},
}

\glossary{name={ChaMPS}, description={Change Management Process System. Dies ist das offizielle Change Mangement Tool von Vodafone. Hier werden geplante �nderungen eingepflegt, die Zustimmung aus anderen Abteilungen ben�tigen.},
}

\glossary{name={AiO}, description={All in One. Dies ist das Configuration-, Change-, Incident- und Security Management Tool der Abteilung TBAN},
}

\glossary{name={TBAN}, description={Steht f�r Technischer Betrieb Access Network. Die Technikabteilung von Vodafone, die sowohl f�r den Betrieb der Technik, also der Server und des Vodafone Live Portals zust�ndig ist, als auch f�r den Bereich Security},
}


