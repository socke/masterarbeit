%% $Id: prototyp.tex 12 2007-05-23 13:11:21Z stefan $
%% Prototyp 

\chapter{Prototyp}\label{prototyp}

Die Entwicklung eines Prototypen zum Test hat zum einen den Zweck, die Durchf�hrbarkeit des Projektes zu testen, aber auch um zu schauen, ob die Ergebnisse der Analyse umzusetzen sind. Oft ist es so, dass viele Informationen aus dem Prototypen gewonnen werden k�nnen. So auch hier, denn diverse Techniken und Ans�tze sind genauso oder abgewandelt auch letztendlich in der finalen Version eingesetzt worden.

\section{Connection Test via Konsole}

Die ersten Verbindungstests zur Sybase-Datenbank von TTWOS \index{TTWOS} fanden auf der Unix-Maschine statt. Die Unix-Maschine verf�gt �ber einen SyBase-Clienten, der die Verbindung zu einer SyBase-Datenbank erm�glicht. Um zu testen, ob eine Verbindung der Unix-Maschine \emph{hqwww01tban} zur TTWOS Datenbank herstellbar ist, wurde �ber die SSH-Shell folgendes Kommando abgesetzt:

\glossary{name={hqwww01tban}, description={Der Rechnername des Intranet-Webservers der Abteilung TBAN},
}

\begin{verbatim}
isql -Sttwostst_hs -Ussobek -P12345678
\end{verbatim}

\glossary{name={isql}, description={Dies ist der Befehl des Sybase-UNIX-Clients, um eine Verbindung mit einer Sybasedatenbank aufzubauen. },
}
\index{isql}

Nachfolgend werden die Parameter erkl�rt. 
\begin{description}
	\item[-S] Dieser Parameter beschreibt den Host, mit dem verbunden werden soll. 
	\item[-U] Dieser Parameter beschreibt den User, welcher f�r die Verbindung benutzt wird. 
	\item[-P] Dieser Parameter beschreibt das Passwort, welches f�r die Verbindung benutzt wird. 
\end{description}

Nach jedem Parameter kann sofort der entsprechende Zeichensatz folgen. Es muss also kein Leerzeichen angegeben werden. 

Der Server meldet einen Fehler:

\begin{verbatim}
CT-LIBRARY error:
        ct_connect(): directory service layer: internal directory control 
        layer error: Requested server name not found.
\end{verbatim}

Der UNIX-Maschine ist der Servername nicht bekannt. Gel�st werden kann das Problem, indem der Datei \emph{/opt/sybase-oc/interfaces} des Sybase Open Clients folgendes hinzugef�gt wird.

\begin{verbatim}
ttwostst_hs
       master tcp ether 10.132.245.130 4100
       query tcp ether 10.132.245.130 4100
\end{verbatim}

Nun ist der TTWOS Server bekannt und es kann via isql-Kommando verbunden werden. 

% section 
\section{Connection Test via PHP}

Anschlie�end wird gepr�ft, ob die Verbindung �ber PHP funktioniert. Hierf�r wurde folgendes PHP-Skript erstellt.

\begin{lstlisting}[caption=PHP Connection Test, language=PHP, label=PHP_Connection_Test]
<?php
//DB-Modul laden
require_once "DB.php";

//setzen der variablen f�r connect
$dsn = array(
    'phptype'  => 'sybase',
    'username' => 'ssobek',
    'password' => '12345678',
    'hostspec' => 'ttwostst_hs',
    'database' => 'ARSystem',
);

//connect
$db =& DB::connect($dsn);
if (DB::isError($db)) {
    die($db->getMessage());
}

$query = "select StandortBezeichnung, StandortOrt from MMO_IST_Standorte WHERE StandortOrt LIKE '%Krefeld%'";
$query2 = "select GT_Ticketnummer, AR_ErstellungsDatum, GT_Buchungsdatum, GP_Priorit_t from MMO_TT_ATT_Bearbeiter_Ticket";

$res =& $db->limitQuery($query2, 0, 10); 

if (DB::isError($res)) {
    die($res->getMessage());
}

//echo "<pre>"; print_r($data); echo "</pre>";

echo "<pre>"; print_r($data); echo "</pre>";
echo '<table border="1">'; 
   echo '<tr>'; 
   echo "<td style=\"background-color:#CCCCCC\">GT_Ticketnummer</td>"; 
   echo "<td style=\"background-color:#CCCCCC\">AR_ErstellungsDatum</td>"; 
   echo "<td style=\"background-color:#CCCCCC\">GT_Buchungsdatum</td>";
   echo "<td style=\"background-color:#CCCCCC\">GP_Priorit_t</td>";
   
   echo '</tr>';
while ($data = $res->fetchRow(DB_FETCHMODE_ASSOC)) 
{    
   echo '<tr>'; 
   echo "<td>$data[GT_Ticketnummer]</td>"; 
   echo "<td>$data[AR_ErstellungsDatum]</td>";
   echo "<td>$data[GT_Buchungsdatum]</td>"; 
   echo "<td>$data[GP_Priorit_t]</td>";
   echo '</tr>';
   
  // print_r($data);
} 
echo '</table>'; 
$db->disconnect();

?>
\end{lstlisting}


\glossary{name={PHP}, description={Hypertext Preprocessor. Eine Webbasierende Programmiersprache. },
}
\index{PHP}

Dieses PHP-Skript benutzt die PEAR-DB Abstraktionsklasse. F�r mehr Informationen �ber die PEAR-Klassen siehe Kapitel \ref{anhang:PEAR}. \index{PEAR}

\glossary{name={PEAR}, description={PHP Extension and Application Repository. PEAR ist eine Sammlung oder auch Bibliothek von Modulen und Klassen f�r die Programmiersprache PHP.},
}
\index{PHP Extension and Application Repository|see{PEAR}}
\section{Datenimport von Sybase zu MySQL}

Weiterhin wurde ein Skript geschrieben, um zu testen, ob der Import aus der Sybase-Datenbank in die MySQL Datenbank problemlos funktioniert. Das Importskript zeigt das Listing \ref{Testimportskript}

\begin{lstlisting}[caption=Testimportskript, language=PHP, label=Testimportskript]
<?php
// include PEAR-classes
require_once 'DB.php';
// include helper classes
require_once 'inc/XmlParser.class.php';
require_once 'inc/HF.class.php';
require_once 'inc/QueryHelper.class.php';

// parse settings file with table and attribute names
$node = XmlParser::parse('inc/settings.xml');
// get debug value from xml
$debug = $node['root']['importer']['debug'];

HF::p($node);

// parse database settings
$database = XmLParser::parse('inc/database.xml');
HF::p($dbconn);
 
//setzen der variablen f�r den sybase-connect
$sybase_dsn = array(
    'phptype'  => $database['root']['databases']['db']['0']['type'],
    'username' => $database['root']['databases']['db']['0']['user'],
    'password' => $database['root']['databases']['db']['0']['pass'],
    'hostspec' => $database['root']['databases']['db']['0']['host'],
    'database' => $database['root']['databases']['db']['0']['name'],
);
//setzen der variablen f�r den mysql-connect
$mysql_dsn = array(
    'phptype'  => $database['root']['databases']['db']['1']['type'],
    'username' => $database['root']['databases']['db']['1']['user'],
    'password' => $database['root']['databases']['db']['1']['pass'],
    'hostspec' => $database['root']['databases']['db']['1']['host'],
    'database' => $database['root']['databases']['db']['1']['name'],
);

// check array keys, to know if only one table or more tables should be imported
$arrKeys = array_keys($node['root']['importer']['table']);
HF::p($arrKeys);
if ($arrKeys[0] == '0') {
    // if more than one table with attributes
	foreach ($node['root']['importer']['table'] as $key => $val) {
	  $arrayTables[] = $val['name'];
	  // create attribute vals
	  foreach ($val['attribute'] as $key2 => $val2) {
		 $arrayAttributes[] = $val['name'].".".$val2;
	  }
	}
} else {
    // if only one table with attributes
    $arrayTables = $node['root']['importer']['table']['name'];
   	foreach ($node['root']['importer']['table']['attribute'] as $key => $val) {
	   $arrayAttributes[] = $arrayTables.".".$val;
	}
}
// create query with QueryHelper
$query = QueryHelper::createQuery($arrayAttributes, $arrayTables);

echo $query."<br />";

// connect to sybase database
$dbsybase =& DB::connect($sybase_dsn);
if (DB::isError($dbsybase)) {
    die($dbsybase->getMessage());
}
// connnect to mysql database
$dbmysql =& DB::connect($mysql_dsn);
if (DB::isError($dbmysql)) {
    die($dbmysql->getMessage());
}

// send query but limit to 10 results
//$res =& $dbsybase->limitQuery($query, 0, 10);  
$res =& $dbsybase->query($query);

if (DB::isError($res)) {
   die($res->getMessage());
}

// prepare sql for import   
$importsql = $dbmysql->autoPrepare($arrayTables, $arrayAttributes, DB_AUTOQUERY_INSERT);

if (PEAR::isError($sth)) {
	die($sth->getMessage());
}

// if doit 1 then import
if ($_REQUEST['doit'] == 1) {
	// fetch associative array rows from database 
	while ($data = $res->fetchRow(DB_FETCHMODE_ASSOC)) 
	{ 
	   if ($debug == true) HF::p($data);

       // execute sql 
	   $mysqlresult =& $dbmysql->execute($importsql, array_values($data));
	   
	   if (PEAR::isError($mysqlresult)) {
			die($mysqlresult->getMessage()."<br />".$mysqlresult->getDebugInfo());
	   }
	} 
}
// close sybase connection
$dbsybase->disconnect();
// close mysql connection
$dbmysql->disconnect();

?>
\end{lstlisting}


Dieses Skript importiert Daten aus der Sybase Tabelle MMO\_TT\_ATT\_Bearbeiter\_Ticket in die MySQL-Tabelle MMO\_TT\_ATT\_Bearbeiter\_Ticket. Dies ist eine 1 zu 1 Kopie. Welche Tabelle und welche Attribute importiert werden sollen, wird aus der settings.xml gelesen. Listing \ref{settings.xml} zeigt den Aufbau der settings.xml Datei.

\begin{lstlisting}[caption=settings.xml, language=XML, label=settings.xml]
<?xml version="1.0" encoding="UTF-8"?>
<importer>
   <debug>0</debug>
   <urlprefix>/www/tban/dev/tban_portal/aio/</urlprefix>   
   <table>
	 <name>MMO_TT_ATT_Bearbeiter_Ticket</name>
	 <attribute>AR_TicketIdentifizierer</attribute>
	 <attribute>AR_Submitter</attribute>
	 <attribute>AR_ErstellungsDatum</attribute>
	 <attribute>GR_BearbeiterName</attribute>
	 <attribute>AR_LetzteModifikationDurch</attribute>
	 <attribute>AR_ModifikationDatum</attribute>
	 <attribute>GT_Status</attribute>
	 <attribute>GT_Anwendung</attribute>
	 <attribute>SP_Faxlayout</attribute>
	 <attribute>SP_Gruppen</attribute>
	 <attribute>SP_InterfaceSchema</attribute>
	 <attribute>Applikationen</attribute>
	 <attribute>GT_Ticketnummer</attribute>
	 <attribute>GT_Benachrichtigungsempf_nger</attribute>
	 <attribute>GT_Buchungsdatum</attribute>
	 <attribute>GT_Faxdatum</attribute>
	 <attribute>GT_Anhang</attribute>
	 <attribute>GT_LB_vorhanden</attribute>
	 <attribute>GP_Problem_berschrift</attribute>
	 <attribute>GP_Problembeschreibung</attribute>
	 <attribute>GP_Problembeginn</attribute>
	 <attribute>GP_Problemende</attribute>
	 <attribute>GP_Priorit_t</attribute>
	 <attribute>Reminder</attribute>
	 <attribute>Auswahl</attribute>
	 <attribute>GA_R_ckmelde_berschrift</attribute>
	 <attribute>GA_R_ckmeldebeschreibung</attribute>
	 <attribute>GA_DatumF_rDieL_sung</attribute>
	 <attribute>GZ_Zur_ckweisungs_berschrift</attribute>
	 <attribute>GZ_Zur_ckweisungsbeschreibung</attribute>
	 <attribute>GR_KoordinatorName</attribute>
	 <attribute>GR_KoordinatorFax</attribute>
	 <attribute>GR_KoordinatorZwischenbericht</attribute>
	 <attribute>GR_BearbeiterZwischenbericht</attribute>
	 <attribute>GR_BearbeiterAbteilung</attribute>
	 <attribute>GR_BearbeiterNL</attribute>
	 <attribute>GR_BearbeiterTelefon</attribute>
	 <attribute>GR_KoordinatorAbteilung</attribute>
	 <attribute>GR_KoordinatorNL</attribute>
	 <attribute>GR_KoordinatorTelefon</attribute>
	 <attribute>GR_KoordinatorBereich</attribute>
	 <attribute>GR_BearbeiterBereich</attribute>
	 <attribute>GR_ExternName</attribute>
	 <attribute>GR_ExternFirma</attribute>
	 <attribute>GR_ExternOrt</attribute>
	 <attribute>GR_ExternTelefon</attribute>
	 <attribute>GR_ExternFax</attribute>
	 <attribute>GR_ExternEmail</attribute>
	 <attribute>GR_ExternAbteilung</attribute>
	 <attribute>GT_Beteiligt</attribute>
	 <attribute>TP_TempServer</attribute>
	 <attribute>TP_Temp</attribute>
	 <attribute>ExtCoordEmail</attribute>
	 <attribute>GE_Externes_System</attribute>
	 <attribute>GE_Externe_EntryId</attribute>
	 <attribute>SlaExecution</attribute>
	 <attribute>SLA__berschritten</attribute>
	 <attribute>SLA_Last</attribute>
	 <attribute>GR_ExtKoordName</attribute>
	 <attribute>GR_ExtKoordTelefon</attribute>
	 <attribute>GR_ExtKoordFax</attribute>
	 <attribute>GE_Externes_System_Koordinator</attribute>
	 <attribute>Externer_Bereich</attribute>
	 <attribute>GP_StandortCode</attribute>
	 <attribute>GP_Netzelementcode</attribute>
	 <attribute>GP_Netzelementbeschreibung</attribute>
	 <attribute>CLIC_Code__</attribute>
	 <attribute>GP_Leitungsname</attribute>
	 <attribute>GP_Leitungsidentifizierer</attribute>
	 <attribute>GL_Logbuchnummer_Slave_</attribute>
	 <attribute>TP_Temp1</attribute>
	 <attribute>TemipProcessId</attribute>
	 <attribute>TP_TempSpoolID</attribute>
	 <attribute>TP_TempSpoolPath</attribute>
	 <attribute>TP_TempSpoolHostname</attribute>
	 <attribute>Koordinator_HO_</attribute>
	 <attribute>KoordinatorBereich_HO_</attribute>
	 <attribute>KoordinatorNL_HO_</attribute>
	 <attribute>BearbeiterName_HO_</attribute>
	 <attribute>BearbeiterBereich_HO_</attribute>
	 <attribute>BearbeiterNL_HO_</attribute>
	 <attribute>Query_List_Handover</attribute>
	 <attribute>TMP_TicketinHO</attribute>
	 <attribute>TMP_KoordinatorNLHO</attribute>
	 <attribute>TMP_KoordinatorBereichHO</attribute>
	 <attribute>TMP_KoordinatornameHO</attribute>
	 <attribute>TMP_BearbeiterNLHO</attribute>
	 <attribute>TMP_BearbeiterBereichHO</attribute>
	 <attribute>TMP_BearbeiterNameHO</attribute>
	 <attribute>TMP_ChckKoordInaktBereic</attribute>
	 <attribute>TMP_ChckBearbeiterInaktBereich</attribute>
	 <attribute>Vorknf__erforderlich</attribute>
	 <attribute>Freigabe_Koord</attribute>
	 <attribute>Freigabe_Bearb</attribute>
	 <attribute>TMP_ChckKoordHOBereich</attribute>
	 <attribute>TMP_ChckKoordHONL</attribute>
	 <attribute>TMP_ChckBearbHOBereich</attribute>
	 <attribute>TMP_ChckBearbHONL</attribute>
	 <attribute>SP_HODurchUser</attribute>
	 <attribute>TMP_GT_Ticketnummer</attribute>
	 <attribute>Referenz_Ticketnummer</attribute>
	 <attribute>Link_Count</attribute>
	 <attribute>Link_Count_Offen</attribute>
	 <attribute>TmpGUID</attribute>
	 <attribute>IsTicketRoutingChild</attribute>
   </table>   
</importer>  
\end{lstlisting}

\section{View}\label{prototypview}

Nachdem die Daten der Tabelle in die lokale Datenbank importiert sind, geht es darum einen Test der View zu fertigen. Hierf�r wurde in der Developer-Umgebung die Datei \enquote{cm\_newIncident.php} angepasst. Diese Datei ist f�r die Neuanlage eines Incident Eintrages verantwortlich. Sie wurde nun so erweitert, dass beim Tippen der Ticketnummer im Ticketfeld ein Dropdown-Menu erscheint und passend zur Eingabe die m�glichen Tickets aus der Datenbank anzeigt. Tippt der Benutzer \enquote{ATT}, so werden alle Tickets aus der Datenbank gelesen in denen \enquote{ATT} vorkommt. Dies w�ren zum Beispiel \enquote{N\_ATT00061679A001} oder auch \enquote{S\_ATT00076439A009}.

\begin{lstlisting}[caption=Result XML, language=XML, label=ResultXML]
<?xml version="1.0">
<root>
  <GT_Ticketnummer>Ticketnummer</GT_Ticketnummer>
	<GP_Problembeschreibung>Problembeschreibung</GP_Problembeschreibung>
	<GA_R_ckmeldebeschreibung>R�ckmeldebeschreibung</GA_R_ckmeldebeschreibung>
	<GZ_Zur_ckweisungsbeschreibung>Zur�ckweisungsbeschreibung</GZ_Zur_ckweisungsbeschreibung>
	<GR_KoordinatorZwischenbericht>Koordinator Zwischenbericht</GR_KoordinatorZwischenbericht>
</root>
\end{lstlisting}

\begin{figure}[htbp]
	\centering
		\includegraphics[width=1.00\textwidth]{bilder/vodafone_aio_im_prototyp.png}
	\caption{Incident Management Prototyp}
	\label{fig:vodafone_aio_im_prototyp}
\end{figure}

Wie in Abbildung \ref{fig:vodafone_aio_im_prototyp} zu sehen, wird w�hrend des Tippens ein XMLHttpRequest Objekt erstellt. Diesem wird als POST-Parameter die Eingabe mitgegeben und das Skript \emph{getTicketNumbers.php} aufgerufen. Diese Skript gibt per HTML eine Unordered List zur�ck, welche dann in einem DIV-Layer angezeigt wird. Per CSS wird dieser Layer noch so gestaltet, dass die Aufz�hlungspunkte verschwinden und beim MouseOver-Event der aktuell ausgew�hlte Eintrag gelb hinterlegt wird. Wird mit der Maus oder mit den Cursortasten ein Eintrag ausgew�hlt und per Mausklick oder Enter best�tigt, so wird erneut im Hintergrund ein XMLHttpRequest Objekt erzeugt. Dieses Objekt ruft nun das Skript \emph{getTicketData.php} auf, welches nun die gesamten Daten des ausgew�hlten Tickets aus der Datenbank abruft und per XML-Datei zur�ckgibt. Auf diese Daten kann dann ganz bequem per DOM zugegriffen werden. So ist es nun m�glich, Inhalte der Webseite auszutauschen, ohne die Seite neu zu laden. Der Vorteil liegt darin, dass im Hintergrund die Daten angefordert werden. Der Benutzer muss nicht warten, bis der Request abgesendet und abgearbeitet wurde. 

Eine weiterer Test wurde mit \enquote{JSON} durchgef�hrt. JSON bedeutet \enquote{JavaScript Object Notation}. Mit JSON ist es m�glich direkt JavaScript Objekte oder Datenstrukturen zu erzeugen. Hierf�r ist auch eine sehr kurze Syntax n�tig. Das JSON-Pendant des Test-XML-Files  zeigt Listing \ref{lst:JSONBeispiel} . 


\begin{lstlisting}[caption=JSON Beispiel, language=Java, label=lst:JSONBeispiel]{JSON Beispiel}
{"root": [
  {"GT_Ticketnummer":"Ticketnummer", "GP_Problembeschreibung":"Problembeschreibung", "GA_R_ckmeldebeschreibung":"R�ckmeldebeschreibung", "GZ_Zur_ckweisungsbeschreibung":"Zur�ckweisungsbeschreibung", "GR_KoordinatorZwischenbericht","Koordinator Zwischenbericht",
]}
\end{lstlisting}


\begin{figure}[htbp]
	\centering
		\includegraphics[width=1.00\textwidth]{bilder/vodafone_aio_im_prototyp_fe.png}
	\caption{Incident Management Prototyp Felder}
	\label{fig:vodafone_aio_im_prototyp_fe}
\end{figure}

Listing \ref{lst:AJAXProtoBeispiel} zeigt die AJAX-Beispielimplementierung des Prototypen.

\begin{lstlisting}[caption=Prototyp AJAX Beispiel, language=VBScript, label=lst:AJAXProtoBeispiel]{Prototyp_AJAX_Bsp}
<script type="text/javascript">
	new Ajax.Autocompleter("ticket", "autocomplete_choices", "modul_im/getTicketNumbers.php",{afterUpdateElement : getSelectionId, indicator: 'indicator1'});
	
	function getSelectionId(text, li) {

		var id = li.id;
		var url = 'modul_im/getTicketData.php';
		var pars = 'ticket=' + id;
		
		var myAjax = new Ajax.Request(
			url, 
			{
				method: 'get', 
				parameters: pars, 
				onComplete: showResponse
			});
	}
	
	function showResponse(originalRequest)
	{
		 var $myXML = originalRequest.responseXML;
		 var $myText = originalRequest.responseText;
		 if ($myXML == null) {
		   alert('Keine XML Daten vom Server erhalten!!!');
		 } else {
			new Effect.Highlight('error',{startcolor:'#ffffff', endcolor:'#ffaaaa', duration: 0.5});
			new Effect.Highlight('solution',{startcolor:'#ffffff', endcolor:'#ffaaaa', duration: 0.5});
			new Effect.Highlight('reason',{startcolor:'#ffffff', endcolor:'#ffaaaa', duration: 0.5});
			new Effect.Highlight('services',{startcolor:'#ffffff', endcolor:'#ffaaaa', duration: 0.5});
			$('error').value = $myXML.getElementsByTagName('GP_Problembeschreibung')[0].firstChild.nodeValue;
            $('solution').value = $myXML.getElementsByTagName('GA_R_ckmeldebeschreibung')[0].firstChild.nodeValue;
			$('reason').value = $myXML.getElementsByTagName('GZ_Zur_ckweisungsbeschreibung')[0].firstChild.nodeValue;
			$('services').value = $myXML.getElementsByTagName('GR_KoordinatorZwischenbericht')[0].firstChild.nodeValue;
			
			 }
	}
	
	
	</script>
\end{lstlisting}

\section{Messungen}

Die Dauer des Importvorganges wurde gemessen. Hierf�r wurden aus der Testdatenbank von TTWOS die Tabelle \emph{MMO\_TT\_ATT\_Bearbeiter\_Ticket importiert}. Diese Tabelle hat \emph{103 Attribute} und beinhaltete zum Testzeitpunkt \emph{8821 Tickets}.
 
\begin{scriptsize}
\begin{longtable}[htbp]{|r|c|}
			\hline
			\rowcolor[gray]{0.9}
			\textbf{Nr.} & \textbf{Zeit / s} \\
			\hline
			1 & 59,1 \\
			\hline
			2 & 57,1 \\
			\hline
			3 & 62,9 \\
			\hline
			4 & 59,7 \\
			\hline
			5 & 59,2 \\
			\hline
			6 & 58,9 \\
			\hline
			7 & 61,0 \\
			\hline
			8 & 60,1 \\
			\hline
			9 & 59,5 \\
			\hline
			10 & 57,9 \\
			\hline
			\hline
			 & Durchschnitt 59,54 \\
			 \hline
	\caption{Datenimport TTWOS Zeitmessung}
	\label{tab:DatenimportTTWOSZeitmessung}
\end{longtable}	
\end{scriptsize}

Das fertige System wird nur die Daten seit dem letzten Import importieren. Nur initial ist mit einer gr��eren Menge Tickets zu rechnen. 

Ein initialer Import bei gesch�tzten 25000 Tickets w�rde etwa zwei Minuten und 44 Sekunden dauern. Dies wird als initiale maximale Ticketanzahl angesehen.

Der t�gliche Import wird vermutlich nicht mehr als eine dreistellige Anzahl an Tickets importieren. Bei 100 Tickets dauert der Importvorgang zwischen 0,5 und 1 Sekunde. 500 Tickets w�rden demnach etwa 5-10 Sekunden dauern. 
