%% $Id: abstract.tex 13 2007-05-31 12:01:48Z stefan $
\chapter*{Zusammenfassung}

\addcontentsline{toc}{chapter}{Zusammenfassung}

Die Abteilung TBAN \index{TBAN} der Vodafone D2 GmbH in D�sseldorf ist f�r den technischen Betrieb der Hard- und Softwareelemente, wie z.B. Server, Router oder Muxer, zust�ndig. Zur Unterst�tzung der Abteilung stehen viele unterschiedliche Tools zu Verf�gung. Configuration-, Change- \index{Changemanagement}, Incident- \index{Incident Management} und Securitymanagement Systeme sind nur ein kleiner Teil der Software, die die t�gliche Arbeit unterst�tzen.

Vodafone verf�gt �ber globale Management Softwaresysteme, die ein breites Spektrum abdecken, da sie auch f�r den gesamten Konzern zur Verf�gung stehen. Vor einiger Zeit hat die Abteilung TBAN eigene Tools entwickelt, die zuerst nur als kleinere Hilfstools wie z.B. Monitoring-Werkzeuge gedacht, mittlerweile aber stark gewachsen und aus dem t�glichen Arbeitsbetrieb nicht mehr wegzudenken sind. 

Um den t�glichen Arbeitsprozess der TBAN-Teams noch weiter zu verbessern und zu erleichtern, ist eine Schnittstelle zum Vodafone Ticketsystem TTWOS \index{TTWOS} und zum Vodafone Changemanagement System ChaMPS \index{ChaMPS} unumg�nglich. 

Das TBAN-eigene System AiO bietet den Vorteil, dass es eigens f�r die Abteilung anzupassen ist und bereits �ber mehrere Schnittstellen verf�gt, die Daten zentralisiert und visualisiert. Dies soll nun auch mit den Daten aus TTWOS und ChaMPS erfolgen. 

Um die genauen Anforderungen festzulegen, wurden mehrere Gespr�che mit Mitarbeitern und Teamleitern der Abteilung TBAN gef�hrt. Der Erstellungsprozess von Tickets sowie die Integration der einzelnen Tools im Arbeitsablauf der einzelnen Mitarbeiter wurde erfasst und anhand diesen Informationen die Anforderungen festgelegt. 

Mit Hilfe dieser Erweitungen, die den Import der TTWOS- und ChaMPS-Daten erm�glicht, ist eine deutliche Zeitersparnis im t�glichen Arbeitsablauf der Mitarbeiter zu erwarten. Weiterhin bietet die Integration der Daten aus TTWOS und ChaMPS eine weitreichendere Reportingm�glichkeit als bisher, sowie eine schnellere und leichtere Fehlersuche f�r die gesamte Abteilung TBAN.

Die erstellte Software greift in das bestehende Change- und Incidentmanagement System von AiO ein und erweitert AiO um Funktionalit�ten, mit deren Hilfe auf die Daten aus TTWOS und ChaMPS zugegriffen werden kann. Hierf�r wurde das Changemanagement System dahingehend angepasst, dass die abteilungrelevanten Informationen aus ChaMPS importiert und direkt auf die Eintr�ge im AiO-eigenen Changemanagement System gemappt werden. Es tauchen die Eintr�ge aus ChaMPS direkt im AiO Change Management auf. 

Die TTWOS-Daten werden importiert und durch einen semi-automatischen Prozess dem AiO Change- oder Incidentmanagement zugef�gt. Bei Neuanlage eines Eintrages ins AiO Change- oder Incidentmanagement kann durch Eingabe der Ticketnummer auf die TTWOS Daten zugegriffen werden. Es wird automatisch eine Verkn�pfung erzeugt. Es ist m�glich, mehrere Tickets einem AiO Change- oder Incidenteintrag zuzuordnen. Ab sofort k�nnen jederzeit die entsprechenden Daten angezeigt werden. Es entf�llt zum einen die gesamte Doppelpflege und zum anderen wird eine weitere Reportingm�glichkeit geschaffen. Es ist vom Netzelement aus m�glich, zu erkennen welche Changes und Incidents stattgefunden haben, gleichzeitig sind die entsprechenden Ticketdaten aus TTWOS verf�gbar. 

Bei der Entwicklung wurden einige neue Techniken eingesetzt, wie z.B. AJAX und Web Services. AiO wurde mit einem Web Service erweitert, der Zugriff auf die importierten Ticketdaten erm�glicht. Der Web Service erwartet als Parameter eine Ticketnummer und liefert als Antwort ein XML-File, welches die gesamten importierten Ticketdaten enth�lt. Ein weiterer Web Service nimmt Ticketdaten entgegen und speichert diese in die AiO-Datenbank. Dies soll zur Demonstration dienen, um aufzuzeigen, welchen Nutzen Web Services in solch einem System wie AiO f�r AiO selbst aber auch f�r externe Systeme und Klienten haben kann. 

Der Web Service wird zu Demonstrationszwecken selbst konsumiert. Er wird dazu benutzt, die Ticketdaten abzufragen. 

Im Rahmen dieses Projektes wurde ein Importer entwickelt, der es erm�glicht, Daten aus einer beliebigen Datenbank zu importieren. Hierf�r ist lediglich die Erstellung eines XML-Files n�tig, welches die Zugriffsdaten auf die entfernte Datenbank sowie die Struktur der zu importierten Tabellen enth�lt. S�mtliche Zwischenschritte, wie die Erzeugung der Tabellen in AiO, werden automatisch durchgef�hrt. Es muss eine korrekte Konfiguration des beschreibenden XML-Files erfolgen. Mit Hilfe dieses Importers k�nnen Daten aus weiteren Datenbanken importiert werden, was auch bereits geschehen ist. Es wird ein t�glicher Import aus der C6000 \index{Configuration Management} Datenbank erfolgen. Dieser Import findet mit Hilfe des in diesem Projekt erstellten Importers statt. Damit stehen nun auch die Hardwareelemente aus C6000 in den AiO Network Elements zur Verf�gung.