%% $Id: auftragsbeschreibung_kurzform.tex 7 2007-04-06 12:57:34Z stefan $
%% Auftragsbeschreibung Kurzform

\chapter{Auftragsbeschreibung}\label{Auftragsbeschreibung}

Die Abteilung TBAN verf�gt derzeit �ber eine Applikation, welche ein Configuration-, Change- und Incidentmanagement System sowie ein System zur Erfassung von �bertragungslinien beinhaltet. Diese Systeme sind miteinander verkn�pft. So greifen die Systeme Configuration Management und Incident Management auf die Elemente des Configuration Management Systems zu. 


Die Firma Vodafone verf�gt �ber ein Trouble Ticket System namens TTWOS. Der Schwerpunkt der Anwendung TTWOS liegt im technischen Netzbetrieb. Die Mitarbeiter werden per E-Mail, per SMS oder per Notification Tool �ber �nderungen in den Tickets individuell benachrichtigt. Nachdem die Vorg�nge abgeschlossen sind, l�sst sich durch Auswertung der gespeicherten Daten eine Qualit�ts�berpr�fung und -verbesserung erzielen.

Weiterhin gibt es ein Change Management System namens ChaMPS, welches geplante �nderungen an Systemen beinhaltet. 

Meine Aufgabe ist es, eine Schnittstelle und somit einen Datenimport aus der Datenbank des TTWOS-Systems und des ChaMPS-Systems in die Datenbank von AiO zu schaffen. Die Ticketdaten sowie die Change Management Daten aus ChaMPS sollen nun jeweils in AiO im eigenen Change Management System oder Incident Management System vorhanden sein. Es ist kein Aufruf einer externen Clientsoftware mit erneutem Login n�tig. Zweck dieses Datenimportes ist es, Zeit zu sparen. Es ist keine Doppelpflege mehr n�tig. Interne Change-Eintr�ge werden nur in der AiO-Change Datenbank gepflegt. Relevante �nderungen, die Auswirkungen auf andere Systeme haben, werden in der TTWOS Datenbank gepflegt und automatisch in AiO importiert. Hier entf�llt der doppelte Pflegeaufwand. Weiterhin dient AiO als schnelle Informationsquelle. Eine m�gliche Erweiterung w�re es durch eine externe Verlinkung direkt neue Tickets in TTWOS �ber ein Webinterface anzulegen. Das Hauptaugenmerkt liegt jedoch auf der einfachen Art der �bersicht in diesem System AiO. Die Daten sollen schnell und effizient verf�gbar sein. 


Um dies zu gew�hrleisten muss mit Hilfe der vorhandenen Soft- und Hardware eine Schnittstelle zur Datenbank von TTWOS und ChaMPS erstellt und die daf�r notwendigen Ansichten erstellt werden. Weitere Aufgaben sind es ein technisches Handbuch f�r eine eventuelle Weiterentwicklung des Systems, sowie ein Handbuch f�r Benutzer zu verfassen.

\glossary{name={Trouble Ticket}, description={Trouble Tickets werden bei Auftreten von Fehlern erzeugt, sie sind die elektronische Form eines Anliegens, einer St�rung oder einer Anfrage.},
}