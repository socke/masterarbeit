%% $Id: header_kurzform.tex 7 2007-04-06 12:57:34Z stefan $

\documentclass[
a4paper,							% alle weiteren Papierformat einstellbar
%landscape,						% Querformat
10pt,						  		% Schriftgr��e (12pt, 11pt (Standard))
%BCOR1cm,							% Bindekorrektur, bspw. 1 cm
%DIVcalc,							% f�hrt die Satzspiegelberechnung neu aus scrguide 2.4
oneside,							% zweiseitig twoside/ einseitig oneside
%twocolumn,						% zweispaltiger Satz
%openany,							% Kapitel k�nnen auch auf linken Seiten beginnen
%openright,						% Kapitel beginnen auf der rechten Seite
%halfparskip*,				% Absatzformatierung s. scrguide 3.1
headsepline,					% Trennline zum Seitenkopf	
footsepline,					% Trennline zum Seitenfu�
%notitlepage,					% in-page-Titel, keine eigene Titelseite
%chapterprefix,				% vor Kapitel�berschrift wird "Kapitel Nummer" gesetzt
%appendixprefix,				% Anhang wird "Anhang" vor die �berschrift gesetzt 
%normalheadings,			% �berschriften etwas kleiner (smallheadings)
%idxtotoc,						% Index im Inhaltsverzeichnis
%liststotoc,					% Abb.- und Tab.verzeichnis im Inhalt
%bibtotoc,						% Literaturverzeichnis im Inhalt
%leqno,								% Nummerierung von Gleichungen links
%fleqn,								% Ausgabe von Gleichungen linksb�ndig
%draft								% �berlangen Zeilen in Ausgabe gekennzeichnet
%smallheadings
] {scrreprt}

% breitere seite
\usepackage{a4wide}

%% Normales LaTeX oder pdfLaTeX? %%%%%%%%%%%%%%%%%%%%%%%%%%%%
%% ==> Das neue if-Kommando "\ifpdf" wird an einigen wenigen
%% ==> Stellen ben�tigt, um die Kompatibilit�t zwischen
%% ==> LaTeX und pdfLaTeX herzustellen.
\newif\ifpdf
\ifx\pdfoutput\undefined
	\pdffalse              %%normales LaTeX wird ausgef�hrt
\else
	\pdfoutput=1           
	\pdftrue               %%pdfLaTeX wird ausgef�hrt
\fi


%% Fonts f�r pdfLaTeX %%%%%%%%%%%%%%%%%%%%%%%%%%%%%%%%%%%%%%%
%% ==> Nur notwendig, falls keine cm-super-Fonts installiert
\ifpdf
	%\usepackage{ae}       %%Benutzen Sie nur eines dieser Pakete:
	%\usepackage{zefonts}  %%je nachdem, welches Sie besitzen.
\else
	%%Normales LaTeX - keine speziellen Fontpackages notwendig
\fi


%% Deutsche Anpassungen %%%%%%%%%%%%%%%%%%%%%%%%%%%%%%%%%%%%%
\usepackage[ngerman]{babel}
\usepackage[T1]{fontenc}
\usepackage[latin1]{inputenc}


%% Packages f�r Grafiken & Abbildungen %%%%%%%%%%%%%%%%%%%%%%
\ifpdf %%Einbindung von Grafiken mittels \includegraphics{datei}
	\usepackage[pdftex]{graphicx} %%Grafiken in pdfLaTeX
\else
	\usepackage[dvips]{graphicx} %%Grafiken und normales LaTeX
\fi
%\usepackage[hang,tight,raggedright]{subfigure} %%Teilabbildungen in einer Abbildung
%\usepackage{pst-all} %%PSTricks - nicht verwendbar mit pdfLaTeX


%% Packages f�r Formeln %%%%%%%%%%%%%%%%%%%%%%%%%%%%%%%%%%%%%
\usepackage{amsmath}
\usepackage{amsthm}
\usepackage{amsfonts}


%% Zeilenabstand %%%%%%%%%%%%%%%%%%%%%%%%%%%%%%%%%%%%%%%%%%%%
%\usepackage{setspace}
%\singlespacing        %% 1-zeilig (Standard)
%\onehalfspacing       %% 1,5-zeilig
%\doublespacing        %% 2-zeilig
\usepackage{fancyhdr} %%Fancy Kopf- und Fu�zeilen
\usepackage{longtable} %%F�r Tabellen, die eine Seite �berschreiten
\usepackage[babel,german=guillemets]{csquotes} % Franz�sische Anf�hrungszeichen  \enquote{}
% fuer Zitate
\usepackage[round]{natbib}
% for listings
\usepackage{listings}


%workaround for lstlistoflistings
\makeatletter% --> De-TeX-FAQ
\renewcommand*{\lstlistoflistings}{%
  \begingroup
    \if@twocolumn
      \@restonecoltrue\onecolumn
    \else
      \@restonecolfalse
    \fi
    \lol@heading
    \setlength{\parskip}{\z@}%
    \setlength{\parindent}{\z@}%
    \setlength{\parfillskip}{\z@ \@plus 1fil}%
    \@starttoc{lol}%
    \if@restonecol\twocolumn\fi
  \endgroup
}
\makeatother% --> \makeatletter

\usepackage[usenames]{color}

% for writing line numbers          
%\usepackage[pagewise,mathlines,displaymath]{lineno}

\lstloadlanguages{PHP, XML, VBScript, Java}

%color definitions
\definecolor{mygray}{rgb}{0.2,0.2,0.2}
\definecolor{mydarkblue}{rgb}{0.2,0.2,0.9}
\definecolor{mydarkred}{rgb}{0.9,0.20,0.2}
\definecolor{mylightergray}{rgb}{0.9,0.9,0.9}

% listing settings
%\lstset{frame=single, numbers=left, numberstyle=\tiny, basicstyle=\footnotesize, stepnumber=1, numbersep=5pt, backgroundcolor=\color{MyGray}, breaklines=true}
\lstset{
        basicstyle=\ttfamily\scriptsize\mdseries,
        keywordstyle=\bfseries\color{mydarkblue},
        identifierstyle=,
        commentstyle=\color{mygray},      
        stringstyle=\itshape\color{mydarkred},
        numbers=left,
        numberstyle=\tiny,
        stepnumber=1,
        breaklines=true,
        frame=none,
        showstringspaces=false,
        tabsize=4,
        backgroundcolor=\color{mylightergray},
        captionpos=b,
        float=htbp,
} 

% f�r tabellen
\usepackage{array}
% f�r lange tabellen
\usepackage{longtable} 
% paket f�r farbige tabellen
\usepackage{colortbl}

\usepackage[pdftex,colorlinks=false,
                      pdfstartview=FitV,
                      linkcolor=blue,
                      citecolor=blue,
                      urlcolor=blue,
          ]{hyperref}
          \pdfinfo{
            /Title      (Diplomarbeit)
            /Author     (Stefan Sobek)
            /Keywords   (Diplomarbeit, Vodafone, TTWOS, AiO, ChaMPS)
          }



%Darstellung des Glossars einstellen
\usepackage[style=super, header=none, border=none, number=none, cols=2,toc=true]{glossary}

\makeglossary