%% $Id: auftragsbeschreibung.tex 14 2007-06-01 12:50:36Z stefan $
%% Auftragsbeschreibung 
\setchapterpreamble[u]{%
\dictum[Erwin Guido Kolbenheyer]{Die ungel�sten Probleme halten einen Geist lebendig und nicht die gel�sten. \dots}}
\chapter{Auftragsbeschreibung}\label{Auftragsbeschreibung}

Die Abteilung TBAN verf�gt derzeit �ber eine Applikation, welche ein Configuration-, Change- und Incidentmanagement System sowie ein System zur Erfassung von �bertragungslinien beinhaltet. Diese Systeme sind miteinander verkn�pft (siehe Abbildung \ref{fig:vodafone_portal_uebersicht} sowie Abbildung \ref{fig:vodafone_aio_ne}). So greifen die Systeme Changemanagement und Incidentmanagement auf die Elemente des Configurationmanagement Systems zu und umgekehrt. Dies erm�glicht eine Menge Reportingm�glichkeiten. 

\begin{figure}[h]
	\centering
		\includegraphics[width=1.00\textwidth]{bilder/vodafone_portal_uebersicht.png}
	\caption{TBAN Portal �bersicht}
	\label{fig:vodafone_portal_uebersicht}
\end{figure}


Die Firma Vodafone verf�gt �ber ein Trouble Ticket System namens TTWOS. Der Schwerpunkt der Anwendung TTWOS liegt im technischen Netzbetrieb. Die Mitarbeiter werden per E-Mail, SMS oder Notification Tool �ber �nderungen in den Tickets individuell benachrichtigt. 

Weiterhin gibt es ein Change Management System namens ChaMPS, welches geplante �nderungen an Systemen beinhaltet und den Change Management Prozess im technischen Betrieb des Konzerns Vodafone abbildet.

Meine Aufgabe ist es, eine Schnittstelle und somit einen Datenimport aus der Datenbank von TTWOS und ChaMPS in die Datenbank von AiO zu schaffen. Die Ticketdaten sowie die Change Management Daten aus ChaMPS sollen nun jeweils in AiO im eigenen Change Management System oder Incident Management System vorhanden sein. Es ist kein Aufruf einer externen Clientsoftware mit erneutem Login n�tig. Zweck dieses Datenimportes ist es, Zeit zu sparen. Es ist keine Doppelpflege mehr n�tig. Interne Change-Eintr�ge werden nur in der AiO-Change Datenbank gepflegt. Relevante �nderungen, die Auswirkungen auf andere Systeme haben, werden in der TTWOS Datenbank gepflegt und automatisch in AiO importiert. Weiterhin dient AiO als schnelle Informationsquelle. Da mit Hilfe des Tools AiO bestimmte Reporting- und Fehlersuchszenarien abgebildet werden k�nnen, soll dies nun mit Hilfe der importierten Daten noch pr�ziser vonstatten gehen.  

\begin{figure}[h]
	\centering
		\includegraphics[width=1.00\textwidth]{bilder/vodafone_aio_ne.png}
	\caption{AiO Network Elements}
	\label{fig:vodafone_aio_ne}
\end{figure}

Um dies zu gew�hrleisten, muss mit Hilfe der vorhandenen Soft- und Hardware eine Schnittstelle zur Datenbank von TTWOS und ChaMPS und die daf�r notwendigen Ansichten erstellt werden. Eine weitere Aufgabe ist es, ein technisches Handbuch f�r eine Weiterentwicklung des Systems zu verfassen. Der im Rahmen dieses Projektes erstellte Code zur Importierung soll es m�glich machen, �ber eine externe Konfiguration Daten aus anderen Datenbanken in die AiO-Datenbank zu importieren. Dadurch soll innerhalb von maximal 1 MT\footnote{Mann-Tag} ein Import von Daten aus einer weiteren Datenbank sichergestellt werden. 
\glossary{name={MT}, description={Mann-Tag. Zeiteinheit, welche die Arbeitskraft eines Mannes pro Tag beschreibt.},
}
\index{Mann-Tag}

Im Rahmen der Analysephase wurde eine weitere Aufgabe hinzugef�gt: die Entwicklung eines Web Services. Der Web Service soll einige Dienste anbieten, zum einen eine Abfrage der TTWOS-Daten und zum anderen die M�glichkeit, Daten in AiO einzuspielen. Hierf�r soll der Web Service Daten von TTWOS entgegennehmen und in die AiO-Datenbank speichern. Vorerst wird dieser Dienst extern nicht konsumiert, er kann aber zu Demonstrationszwecken z.B. in der View konsumiert werden, anstatt hier den direkten Datenbankzugriff zu benutzen. 

Der Zweck dieses Vorgehens ist der sp�tere leichte Anschluss an einen m�glichen ESB\footnote{Enterprise Service Bus}, welcher sich in der Planung befindet und gegebenenfalls gegen Ende dieses Jahres eingef�hrt wird. Ein weiterer Vorteil ist, dass nun auch die Daten aus AiO �ber einen Web Service anderen Abteilungen zur Verf�gung gestellt werden k�nnen, ohne dass ein direkter Datenbankzugang erstellt werden muss. Der Zugriff erfolgt dann �ber eine definierte Schnittstelle.  

\glossary{name={Trouble Ticket}, description={Trouble Tickets werden bei Auftreten von Fehlern erzeugt, sie sind die elektronische Form eines Anliegens, einer St�rung oder einer Anfrage.},
}