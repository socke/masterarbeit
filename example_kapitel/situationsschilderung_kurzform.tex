%% $Id: situationsschilderung_kurzform.tex 7 2007-04-06 12:57:34Z stefan $
%% Situationsschilderung Kurzform

\chapter{Situationsschilderung}\label{Situationsschilderung}

Die Abteilung TBAN verf�gt seit mehreren Jahren �ber eine eigene Intranet-Plattform, welche diverse Hilfs- sowie Statistiktools beinhaltet. Dar�ber hinaus wurde vor ca. zwei Jahren dieser Plattform eine Applikation namens AiO hinzugef�gt. AiO bedeutet \enquote{All in One} und ist eine Applikation, welche �ber eine Change-, ein Incident- und ein Configuration Mangagement System verf�gt. Weiterhin verf�gt es noch �ber ein Security Mangagement System, sowie ein System zur Erfassung von �bertragungslinien zwischen Systemen, genannt ION\footnote{Inter Office Network}. 

Der Konzern Vodafone verf�gt bereits �ber ein Change Management System namens \enquote{ChaMPS} und �ber ein Trouble Ticket System namens \enquote{TTWOS}, jedoch ist dieses konzernweit, nicht genau angepasst und zum Teil umst�ndlich zu handhaben. Aus diesem Grund wurde vor ca. zwei Jahren AiO eingef�hrt, um nicht nur geplante �nderungen an Systemen die eventuell auch Auswirkungen auf andere Systeme haben k�nnten, sondern auch �nderungen an Systemen, die nur abteilungsintern Relevanz haben, zu erfassen. Mit diesen Informationen, die �ber AiO schnell abrufbar sind, lassen sich �nderungen und eventuelle erfasste Fehler an Systemen nachverfolgen und beheben. 

Geplante �nderungen an Systemen m�ssen allerdings nach wie vor in \enquote{ChaMPS} eingetragen werden und zus�tzlich auch in AiO - Change Management. Genauso verh�lt es sich mit Incidents, also aufgetretene und unerwartete Fehler. Diese m�ssen nicht nur in TTWOS eingetragen werden, sondern auch noch parallel ins Incident Management von AiO. 

Um diesen Mehraufwand der doppelten Pflege zu vermeiden und gleichzeitig die �bersichtlichkeit und Einfachheit der Anwendung AiO weiterhin zu nutzen, w�re es sinnvoll die Daten von TTWOS und von ChaMPS in AiO zur Verf�gung zu haben. 



