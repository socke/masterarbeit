%% $Id: anhang_phpunit.tex 10 2007-05-15 08:52:47Z stefan $
%\setchapterpreamble[u]{%
%\dictum[John F. Kennedy]{Einen Vorsprung im Leben hat, wer da anpackt, wo die anderen erst einmal reden. \dots}}
\chapter{PHPUnit} \label{anh:phpunit}

\section{Was ist PHPUnit} \index{PHPUnit}


Wikipedia definiert PHPUnit folgenderma�en:

\begin{quotation}
PHPUnit ist ein in PHP geschriebenes Open Source Framework zum Testen von PHP-Scripten, das besonders f�r automatisierte Tests einzelner Einheiten (Units) (meist Klassen oder Methoden) geeignet ist. Es basiert auf dem Konzept und der Idee von JUnit. 
 \citep[vgl.][]{wikiphpunit}	
\end{quotation}

Mit Hilfe von PHPUnit ist es m�glich, den erstellten Code zu testen. \enquote{Unit-Tests sind essentieller Bestandteil verschiedener Software-Entwicklungsprozesse, wie Test-First-Programmierung\index{Test-First-Programmierung}, Extreme Programming \index{Extreme Programming} und testgetriebener Entwicklung} \citep[vgl.][Kapitel 8]{bergmanphpunit}. Test-First-Programmierung ist Bestandteil von Extreme Programming und testgetriebener Entwicklung. Dabei werden die Tests erstellt, bevor �berhaupt der Code geschrieben wird. Dadurch schlagen nat�rlich alle Tests fehl, was f�r den Entwickler anfangs sicher sehr ungew�hnlich ist. Daran gew�hnt der Entwickler sich jedoch recht schnell. Nach Erstellung der Tests geht man dazu �ber, den Code zu schreiben. Dieser wird so geschrieben, dass der Test nicht mehr fehlschl�gt. Man entwickelt eine Klasse nach der anderen, kommt kleine Schritte weiter, kann aber sicherstellen, dass der entwickelte Code getestet ist. Vor allem hat diese Art von Test den gro�en Vorteil, dass sie immer und immer wieder ausgef�hrt werden k�nnen. Herk�mmliches Testen, z.B. mit Hilfe von print- oder echo-Anweisungen, m�ssen nat�rlich sp�ter auskommentiert werden. Die entwickelten Unittests k�nnen, dank der heutigen schnellen Rechner, hunderte oder sogar tausende Male ausgef�hrt werden und das vollautomatisch. So f�llt ein Problem bei einer �nderung sofort auf. Unittest werden h�ufig in Kombination mit Coveragetests, also Tests, welche die Testabdeckung des Codes �berpr�fen, eingesetzt.  

\section{PHPUnit installieren}\label{PHPUnitinstallieren}

PHPUnit sollte am besten mit Hilfe des PEAR Installers installiert werden. Vorerst muss PEAR der PHPUnit Kanal bekannt gemacht werden:

\begin{verbatim}
pear channel-discover pear.phpunit.de
\end{verbatim}

Anschlie�end kann das PHPUnit-Paket installiert werden:

\begin{verbatim}
pear install phpunit/PHPUnit
\end{verbatim}

\section{PHPUnit f�r PHP4}

Da in diesem Projekt nur PHP4 zur Verf�gung steht, muss auch auf die PHPUnit-Version f�r PHP4 zur�ckgegriffen werden. Diese installiert man wie folgt:

\begin{verbatim}
pear install -f http://pear.phpunit.de/get/PHPUnit-1.3.2.tgz
\end{verbatim}

PHPUnit f�r PHP4 ist keine vollst�ndige Portierung von JUnit, da das Objektmodell von PHP4 gegen�ber dem von PHP5 sehr eingeschr�nkt ist. Es fehlen einige Leistungsmerkmale wie z.B. Code-Coverage-Analyse. 
\citep[vgl.][Anhang A]{bergmanphpunit}


\section{Beispielcode PHPUnit f�r PHP4}

Bei PHPUnit f�r PHP4 muss die Testklasse von PHPUnit\_TestCase abgeleitet werden. Dies zeigt Listing \ref{lst:PHPUnit_Testklasse}.

\begin{lstlisting}[caption=PHPUnit f�r PHP4 Testklasse, language=PHP, label=lst:PHPUnit_Testklasse]
<?php
require_once 'PHPUnit/TestCase.php';
 
class ArrayTest extends PHPUnit_TestCase {
    var $_fixture;
 
    function setUp() {
        $this->_fixture = array();
    }
 
    function testNewArrayIsEmpty() {
        $this->assertEquals(0, sizeof($this->_fixture));
    }
 
    function testArrayContainsAnElement() {
        $this->_fixture[] = 'Element';
        $this->assertEquals(1, sizeof($this->_fixture));
    }
}
?>
\end{lstlisting}

\textbf{Erkl�rung:}
\begin{description}
	\item[Zeile 4] Die Testklasse wird von PHPUnit\_Testcase abgeleitet.
	\item[Zeile 11-18] Testmethoden, welche sp�ter automatisch ausgef�hrt werden. 
\end{description}

Listing \ref{lst:PHPUnit_Testsuite} zeigt, wie der Test ausgef�hrt werden kann.

\begin{lstlisting}[caption=PHPUnit f�r PHP4 Testsuite, language=PHP, label=lst:PHPUnit_Testsuite]
<?php
require_once 'ArrayTest.php';
require_once 'PHPUnit.php';
 
$suite  = new PHPUnit_TestSuite('ArrayTest');
$result = PHPUnit::run($suite);
 
print $result->toString();
?>\end{lstlisting}

\textbf{Erkl�rung:}
\begin{description}
	\item[Zeile 5] Eine Testsuite wird erstellt. Hierf�r wird der Name der Klasse als String �bergeben.
	\item[Zeile 6] Der Test wird gestartet. 
\end{description}