%% $Id: abstract.tex 13 2007-05-31 12:01:48Z stefan $
\section*{Abstract}

\addcontentsline{toc}{chapter}{Abstract}

The department TBAN of the Vodafone D2 GmbH in D�sseldorf is responsible for the technical service of hardware and software elements like servers, routers and muxers. To support the daily work process of the department, several tools like Configuration-, Change-, Incident- and Securitymanagement are available. These tools are only a small part of the supporting tools of TBAN. 

Vodafone comes with global Management Software Systems, which covers a wide spectrum to support the whole company. Two years ago, the department TBAN created their own tools for supporting purposes, first intended as smaller helping tools but growing fast and now they are mandatory for the daily work. 

To improve the daily work process of the TBAN teams, an interface to the Vodafone ticketsystem TTWOS and the Changemanagement System ChaMPS is essential. 

The advantages of the TBAN application AiO is that it is customisable for the needs of the department and that it owns many interfaces to other systems to centralise and visualise information. Data from TTWOS and ChaMPS will now be added to AiO.

Several interviews with team members were held to set up the requirements. The process of ticket creation and the integration of other tools into the daily work process of the team members was acquired. 

By implementing the TTWOS and ChaMPS data import extension, considerable timesaving is expected in the daily work process of the team members. Furthermore this integration will improve the reportings and the fault finding process of the department TBAN.

The created software will be included into the actual Change- and Incidentmanagement System of AiO and extends AiO with TTWOS and ChaMPS data access functionality. Therefore the Changemanagement System will be adapted. ChaMPS changes will be mapped to AiO Changes. Now the ChaMPS changes will appear in AiO Changemanagement. 

The TTWOS data will be imported into AiO Change- or Incidentmanagement by a semi-automatic process. When creating a new entry in Change- or Incidentmanagement, the imported ticket data will be received from the AiO servers database and will be shown in the GUI. It is possible to add more than one ticket to a Change or Incident dataset. Change or Incidents will no longer be filled in twice, into TTWOS or ChaMPS and into AiO. 

During the implementation several new techniques has been used, like AJAX and Web Services. AiO has been extended with a Web Service which supports ticketdata for the consumer. The consumer has to deliver a ticketnumber as parameter to the Web Service and will receive a XML-file with the complete ticket data. Another Web Service must be filled with ticket data and saves this data into the AiO database. This is for demonstration purposes to show which benefit Web Services can have in such a sytem like AiO for external clients and for AiO itself. 

The Web Service will be consumed for demonstration purposes by AiO itself. It is used to receive the ticket data. 

An importer was created within this project which allows to import data from any database into the AiO database. Therefore only two XML config files have to be created, one with the database login data and second the configuration of the import tables and attributes. Every step like the creation of the import tables will be executed automatically with the help of the config files. With this importer data from the Vodafone Configurationmanagement System C6000 was already imported so that the Network Element data will be available in AiO Network Elements. 

 