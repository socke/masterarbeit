%include header with all settings
\documentclass[
paper=a4,							% alle weiteren Papierformat einstellbar
%landscape,						% Querformat
fontsize=12pt,						  		% Schriftgrˆfle (12pt, 11pt (Standard))
BCOR=20mm,							% Bindekorrektur, bspw. 1 cm
DIV=13,							% f¸hrt die Satzspiegelberechnung neu aus scrguide 2.4
twoside=true, 						% zweiseitig twoside/ einseitig oneside
%twocolumn,						% zweispaltiger Satz
%openany,							% Kapitel kˆnnen auch auf linken Seiten beginnen
open=right,						% Kapitel beginnen auf der rechten Seite
%parskip=half*,				% Absatzformatierung s. scrguide 3.1
parskip=half+,			   	
headsepline,					% Trennline zum Seitenkopf	
footsepline,					% Trennline zum Seitenfufl
%notitlepage,					% in-page-Titel, keine eigene Titelseite
chapterprefix=false,				% vor Kapitel¸berschrift wird "Kapitel Nummer" gesetzt
appendixprefix,				% Anhang wird "Anhang" vor die ‹berschrift gesetzt 
%normalheadings,			% ‹berschriften etwas kleiner (smallheadings)
%smallheadings, 
%idxtotoc,						% Index im Inhaltsverzeichnis
%liststotoc,					% Abb.- und Tab.verzeichnis im Inhalt
%bibtotoc,					% Literaturverzeichnis im Inhalt
%leqno,						% Nummerierung von Gleichungen links
%fleqn,						% Ausgabe von Gleichungen linksb¸ndig
%draft,						% ¸berlangen Zeilen in Ausgabe gekennzeichnet
%pointlessnumbers,
] {scrreprt}

% breitere seite
%\usepackage{a4wide}

%% Normales LaTeX oder pdfLaTeX? %%%%%%%%%%%%%%%%%%%%%%%%%%%%
%% ==> Das neue if-Kommando "\ifpdf" wird an einigen wenigen
%% ==> Stellen benˆtigt, um die Kompatibilit‰t zwischen
%% ==> LaTeX und pdfLaTeX herzustellen.
\newif\ifpdf
\ifx\pdfoutput\undefined
	\pdffalse              %%normales LaTeX wird ausgef¸hrt
\else
	\pdfoutput=1           
	\pdftrue               %%pdfLaTeX wird ausgef¸hrt
\fi

%% Fonts f¸r pdfLaTeX %%%%%%%%%%%%%%%%%%%%%%%%%%%%%%%%%%%%%%%
%% ==> Nur notwendig, falls keine cm-super-Fonts installiert
%\ifpdf
	%\DeclareGraphicsExtensions{.pdf,.jpg,.png}
	%\usepackage{ae}       %%Benutzen Sie nur eines dieser Pakete:
	%\usepackage{zefonts}  %%je nachdem, welches Sie besitzen.
%\else
	%\DeclareGraphicsExtensions{.eps}
	%%Normales LaTeX - keine speziellen Fontpackages notwendig
%\fi

%% Deutsche Anpassungen %%%%%%%%%%%%%%%%%%%%%%%%%%%%%%%%%%%%%
\usepackage[ngerman]{babel}
\usepackage[T1]{fontenc}
%\usepackage[latin1]{inputenc}
\usepackage[utf8]{inputenc}


%% Packages f¸r Grafiken & Abbildungen %%%%%%%%%%%%%%%%%%%%%%
\ifpdf %%Einbindung von Grafiken mittels \includegraphics{datei}
	\usepackage[pdftex]{graphicx} %%Grafiken in pdfLaTeX
\else
	\usepackage[dvips]{graphicx} %%Grafiken und normales LaTeX
\fi
%\usepackage[hang,tight,raggedright]{subfigure} %%Teilabbildungen in einer Abbildung
%\usepackage{pst-all} %%PSTricks - nicht verwendbar mit pdfLaTeX
\let\ifpdf\relax

%% Packages f¸r Formeln %%%%%%%%%%%%%%%%%%%%%%%%%%%%%%%%%%%%%
\usepackage{amsmath}
\usepackage{amsthm}
\usepackage{amsfonts}

%\usepackage{ccfonts} % Schrift  concrete
%\usepackage{chancery} % Schrift  Zapf Chancery -> sehr schnörkelig
%\usepackage{bookman} % Schrift  bookman -> sehr schöne Schrift!
\usepackage{mathpazo} % Schrift  Palatino -> sehr schöne Schrift!
\linespread{1.05}  
%\usepackage{newcent} % New Century Schoolbook -> sehr schön, kräftig!
%\usepackage{charter} % Schrift  Charter -> 
%\usepackage{times}    % Schriftstil Times New Roman (wie Word)
\usepackage{url}      % zur Dartellung von URLs mit Befehl \url{}
%\usepackage{xspace}   % Intelligenter Platzhalter nach Makros
\usepackage{multirow} 
\usepackage{tabularx} 
\usepackage{booktabs} % schˆne Tabellen mit \toprule, \midrule und \bottomrule


% Bildunterschrift

%\usepackage{type1cm} % scalable Fonts
%\usepackage{courier} % Adobe Courier
%\usepackage{sectsty} % eigene Kapitel-Stile
% Control the fonts and formatting used in the table of contents.
%\usepackage[titles]{tocloft}

%\usepackage[Lenny]{fncychap} % Kapitel-Rahmen am Beginn jedes neuen Kapitels

%% Aesthetic spacing redefines that look nicer to me than the defaults.
%\setlength{\cftbeforechapskip}{2ex}
%\setlength{\cftbeforesecskip}{0.5ex}

%\newcommand{\autor}[1]{\textsc{#1}} % Makro f¸r Autor(en) im Fliefltext
%\newcommand{\degr}{\ensuremath{^\circ}} % Grad-Kreis
%\newcommand{\mum}{\ensuremath{\,\mu\textrm{m}}}
%\newcommand{\subcaption}[1]{\footnotesize\itshape #1}
%\newcommand{\matlab}{\emph{MATLAB}\xspace}

%% Use Helvetica-Narrow Bold for Chapter entries
%\renewcommand{\cftchapfont}{%
%  \fontsize{11}{13}\usefont{OT1}{phv}{bc}{n}\selectfont
%}

% \renewcommand{\baselinestretch}{1.5} % Zeilenabstand 1.5

% Verhindern von "`Schusterjungen"' und "`Hurenkindern"'
\clubpenalty = 10000
\widowpenalty = 10000
\displaywidowpenalty = 10000
\tolerance=500 %Zeilenumbruch


%% Zeilenabstand %%%%%%%%%%%%%%%%%%%%%%%%%%%%%%%%%%%%%%%%%%%%
\usepackage{setspace}
\singlespacing        %% 1-zeilig (Standard)
%\onehalfspacing       %% 1,5-zeilig
%\doublespacing        %% 2-zeilig
\usepackage{fancyhdr} %%Fancy Kopf- und Fuflzeilen
\usepackage{longtable} %%F¸r Tabellen, die eine Seite ¸berschreiten
\usepackage[babel,german=guillemets]{csquotes} % Franzˆsische Anf¸hrungszeichen  \enquote{}
% fuer Zitate
%\usepackage[round]{natbib}
\usepackage[numbers,round]{natbib}

% for listings
\usepackage{listings}

% Schriften-Größen
\setkomafont{chapter}{\Huge\rmfamily} % Überschrift der Ebene
\setkomafont{section}{\Large\rmfamily}
\setkomafont{subsection}{\large\rmfamily}
\setkomafont{subsubsection}{\large\rmfamily}
\setkomafont{chapterentry}{\large\rmfamily} % Überschrift der Ebene in Inhaltsverzeichnis
\setkomafont{descriptionlabel}{\bfseries\rmfamily} % für description Umgebungen
\setkomafont{captionlabel}{\small\bfseries}
\setkomafont{caption}{\small}

%\B

%workaround for lstlistoflistings
%\makeatletter% --> De-TeX-FAQ
%\renewcommand*{\lstlistoflistings}{%
%  \begingroup 
 %   \if@twocolumn
 %     \@restonecoltrue\onecolumn
 %   \else
 %     \@restonecolfalse
 %   \fi
 %   \lol@heading
 %   \setlength{\parskip}{\z@}%
 %   \setlength{\parindent}{\z@}%
 %   \setlength{\parfillskip}{\z@ \@plus 1fil}%
 %   \@starttoc{lol}%
 %   \if@restonecol\twocolumn\fi
 % \endgroup
%}
%\makeatother% --> \makeatletter

\usepackage[usenames]{color}

%cmyk for coloring 
\usepackage[cmyk]{xcolor}

% for writing line numbers          
\usepackage[pagewise,mathlines]{lineno}

\lstloadlanguages{PHP, XML, VBScript, Java, HTML,SQL,sh}

%color definitions
\definecolor{mygray}{cmyk}{0.9,0.9,0.9,0,9}
\definecolor{mydarkblue}{cmyk}{1,0.7,0,0.16}
\definecolor{mydarkred}{cmyk}{0.1,0.75,0.75,0.33}
\definecolor{mylightergray}{cmyk}{0.02,0.02,0.02,0.02}
\definecolor{myyellow}{cmyk}{0.0, 0.05, 1.0, 0.24} 
\definecolor{mydarkgreen}{cmyk}{1.0, 0.0, 1.63, 0.67} 

\definecolor{hellgelb}{rgb}{1,1,0.9}
\definecolor{colKeys}{rgb}{0,0,1}
\definecolor{colIdentifier}{rgb}{0,0,0}
\definecolor{colComments}{rgb}{0,0.5,0}
\definecolor{colString}{rgb}{1,0,0}

% listing settings
%\lstset{frame=single, numbers=left, numberstyle=\tiny, basicstyle=\footnotesize, stepnumber=1, numbersep=5pt, %backgroundcolor=\color{mygray}, breaklines=true}
\lstset{
       basicstyle=\ttfamily\scriptsize\mdseries,
        %keywordstyle=\bfseries\color{mydarkblue},
        keywordstyle=\color{colKeys}, 
        identifierstyle=\color{colIdentifier},
        %commentstyle=\color{mydarkgreen},   
        commentstyle=\color{colComments},
        stringstyle=\color{colString},   
        %stringstyle=\itshape\color{mydarkred},
        numbers=left,
        numberstyle=\tiny,
        stepnumber=1,
        breaklines=true,
        frame=single,
        showstringspaces=false,
        tabsize=2,
        %backgroundcolor=\color{mylightergray},
        backgroundcolor=\color{hellgelb},
        captionpos=b,
        float=htbp,
        breakautoindent=true
} 

% Zum Einbinden von Programmcode --------------------------------------------

%\lstset{%
% float=hbp,%
%    basicstyle=\texttt\small, %
%    identifierstyle=\color{colIdentifier}, %
%    keywordstyle=\color{colKeys}, %
%    stringstyle=\color{colString}, %
%    commentstyle=\color{colComments}, %
%    columns=flexible, %
%    tabsize=2, %
%    frame=single, %
%    extendedchars=true, %
%    showspaces=false, %
%    showstringspaces=false, %
%    numbers=left, %
%    numberstyle=\tiny, %
%    breaklines=true, %
%    backgroundcolor=\color{hellgelb}, %
%    breakautoindent=true, %
%    captionpos=b%
%}

% f¸r tabellen
\usepackage{array}
% f¸r lange tabellen
\usepackage{longtable} 

\addtokomafont{caption}{\normalsize}

% paket f¸r farbige tabellen
\usepackage{colortbl}

% Einstellungen für links, unter anderem farbige links
\usepackage[pdftex,colorlinks=false,
                      pdfstartview=FitV,
                      linkcolor=blue,
                      citecolor=blue,
                      urlcolor=blue,
          ]{hyperref}
          \pdfinfo{
            /Title      (Implementierung eines Web Services nach ISO 29002-31 - Query for characteristic data)
            /Author     (Stefan Sobek)
            /Keywords   (FernUni Hagen, Masterarbeit, Stefan Sobek)
          }

\hypersetup{
    pdftitle={Masterarbeit - Implementierung eines Web Services nach ISO 29002-31 - Query for characteristic data},
    pdfauthor={Stefan Sobek},
    pdfkeywords={FernUni Hagen, ISO 29002-31, characteristic product data, 2014}
}

% dictum kapitel zitatsbreite vergrößern
\renewcommand*{\dictumwidth}{.6667\textwidth}

%Darstellung des Glossars einstellen
%\usepackage[style=long,toc]{glossaries}
\usepackage[
%nonumberlist, %keine Seitenzahlen anzeigen
acronym,      %ein Abkürzungsverzeichnis erstellen
toc,          %Einträge im Inhaltsverzeichnis
%section %im Inhaltsverzeichnis auf section-Ebene erscheinen
]      
{glossaries}

%glossar befehle einschalten
\makeglossaries

%index
\usepackage{makeidx}

\usepackage[intoc]{nomencl}

\makeindex


\DeclareGraphicsExtensions{.pdf,.png,.jpg} % bevorzuge pdf-Dateien
\usepackage{subfigure} % mehrere Abbildungen nebeneinander/übereinander
\newcommand{\subfigureautorefname}{\figurename} % um \autoref auch für subfigures benutzen
\usepackage[all]{hypcap} % Beim Klicken auf Links zum Bild und nicht zu Caption gehen
% Bildunterschrift
\setcapindent{0em} % kein Einrücken der Caption von Figures und Tabellen
\setcapwidth[c]{0.9\textwidth}
\setlength{\abovecaptionskip}{0.2cm} % Abstand der zwischen Bild- und Bildunterschrift

% Eigene Befehle %%%%%%%%%%%%%%%%%%%%%%%%%%%%%%%%%%%%%%%%%%%%%%%%%5
% Matrix
\newcommand{\mat}[1]{
      {\textbf{#1}}
}
\newcommand{\todo}[1]{
      {\colorbox{red}{ TODO: #1 }}
}
\newcommand{\todotext}[1]{
      {\color{red} TODO: #1} \normalfont
}
\newcommand{\info}[1]{
      {\colorbox{blue}{ (INFO: #1)}}
}
% Hinweis auf Programme in Datei
\newcommand{\datei}[1]{
      {\ttfamily{#1}}
}
\newcommand{\code}[1]{
      {\ttfamily{#1}}
}
% bild mit defnierter Breite einfügen
\newcommand{\bild}[4]{
  \begin{figure}[!hbt]
    \centering
      \vspace{1ex}
      \includegraphics[width=#2]{images/#1}
      \caption[#4]{\label{fig:#1} #3}
    \vspace{1ex}
  \end{figure}
}
% bild mit eigener Breite
\newcommand{\bilda}[3]{
  \begin{figure}[!hbt]
    \centering
      \vspace{1ex}
      \includegraphics{images/#1}
      \caption[#3]{\label{fig:#1} #2}
      \vspace{1ex}
  \end{figure}
}


% Bild todo
\newcommand{\bildt}[2]{
  \begin{figure}[!hbt]
    \begin{center}
      \vspace{2ex}
	      \includegraphics[width=6cm]{images/todo}
      %\caption{\label{#1} \color{red}{ TODO: #2}}
      \caption{\label{#1} \todotext{#2}}
      %{\caption{\label{#1} {\todo{#2}}}}
      \vspace{2ex}
    \end{center}
  \end{figure}
}



%%%%%%%%%%%%%%%%%%%%%%%%%%%%%%%%%%%%%%%%%%%%%%%%%%%%%%%%%%%%%
%% DOKUMENT
%%%%%%%%%%%%%%%%%%%%%%%%%%%%%%%%%%%%%%%%%%%%%%%%%%%%%%%%%%%%%
\begin{document}

\pagestyle{empty} %%Keine Kopf-/Fusszeilen auf den ersten Seiten.

\ifpdf
	\DeclareGraphicsExtensions{.pdf,.jpg,.png}

\else
	\DeclareGraphicsExtensions{.eps}
\fi

%% Deckblatt %%%%%%%%%%%%%%%%%%% %%%%%%%%%%%%%%%%%%%%%%%%%%%%%
% Titelseite 
\begin{titlepage}
\vspace{4em}
\begin{center}
	\includegraphics[width=0.50\textwidth]{images/feulogo.jpg}
\end{center}
\center

 \Large{\textsf{\textbf{Masterarbeit zum Thema}}}
 \vspace{1em}

\Huge{\textsf{Implementierung eines Webservices nach \\ ISO 29002-31 \\  \enquote{Query for characteristic data}}}
\vspace{1em}
\\


\vspace{1em}

{\normalsize 
\textsf{
Vorgelegt der\\
Fakultät für Mathematik und Informatik\\Fernuniversität Hagen\\Lehrgebiet Datenbanksysteme für neue Anwendungen
}
}
\vspace{2em}
\\

\normalsize{
	\textsf{
	von \\
Stefan Sobek \\ 
Matrikelnummer 7736096 \\
\vspace{2em}
Eingereicht am \\  
11. Februar 2014
\vspace{3em}
\\
Betreuer: Dr. Wolfgang Wilkes\\
Prof. Dr. Ralf Hartmut Güting \\
}
}
\end{titlepage}
 

%activate line numbers for corrections
%\linenumbers 
%\pagenumbering{arabic}
%\pagenumbering{Roman}
%\setcounter{page}{2} 
\chapter*{Zusammenfassung}

\addcontentsline{toc}{chapter}{Zusammenfassung}

In der heutigen Zeit stehen Daten im absoluten Fokus. Schlagwörter wie \enquote{Big Data}, \enquote{Cloud Services} und \enquote{Mobile Devices} sind in aller Munde. Daten werden überall erfasst, seien es Multimedia-Daten wie Bilder, Ton oder Filme oder seien es Personendaten oder Produktdaten. Mobile Endgeräte wie z.B. ein Smartphone, allerdings ebenso Kraftfahrzeuge, erfassen laufend Daten wie die Position, Geschwindigkeit oder über Sensoren Werte wie Temperatur oder Luftdruck. Solche Daten werden gespeichert, und gegebenenfalls weiterverarbeitet und fallen in schier unfassbaren Mengen an. 

Oft spielt in diesen gesammelten Massendaten die Qualität der Daten eine untergeordnete Rolle, wobei in diesem Kontext mit Qualität eine möglichst präzise konzeptuelle Beschreibung der Daten gemeint ist.  
Betrachtet man beispielsweise Stammdaten in der Industrie, findet man eine große Menge an Daten, wie z.B. Daten zu Kunden, Lieferanten, Produkten, Materialien, Wirtschaftsgüter oder Angestellten. Die Qualität dieser Daten hat in der Industrie eine deutlich höhere Gewichtung. Die Unternehmen katalogisieren und beschreiben Ihre Daten, so dass diese innerhalb der Organisation zwischen einzelnen Systemen ausgetauscht werden können und definiert ist, was die einzelnen Datensätze bedeuten. Nehmen wir beispielsweise einen Hersteller von Sechkantschrauben zur Befestigung von Rädern an Fahrzeugen. Die Information über diese herzustellende Schraube muss definiert und in der Produktion bekannt sein. Der Verkauf benötigt allerdings alle diese Informationen für den Vertrieb des Produktes. Ferner möchte der Kunde der diese Schraube für die Produktion seines Fahrzeuges benötigt selbstverständlich auch diese Information. Zum einen für den Einkauf, zum anderen in der Planung und Fertigung, denn es muss bekannt sein aus welchen Einzelteilen (Stückliste) ein Produkt besteht. 

Es ist hier ersichtlich, dass Stammdaten in den verschiedenen Bereichen benötigt werden. Zu nennen seien beispielsweise die Lieferketten, Design und Herstellungprozesse als auch im Produkt Lebenszyklus Management.     

Probleme und somit eine hoher Kostenaufwand können auftreten, wenn die Beschreibung der Daten nicht vollständig ist, wenn gleichsam Informationen zu Stammdaten an einem Punkt der Lieferkette oder im Produktlebenszyklusmanagement fehlen oder ungenau sind. Daten müssen im schlimmsten Falle manuell gesichtet, manuell geprüft und angereichert werden.  

Diese Daten in Konzepte zu beschreiben, diese Konzepte verfügbar zu machen und diese Daten standardisiert und vor allem die automatisierte Austauschbarkeit zu schaffen ist das Ziel. 

Die Abschlussarbeiten des Fachbereiches rund um die PLIB (Parts Library) befassen sich mit ISO-Standards zu diesen Themen. Das Ziel dieser Arbeit ist es, die Implementierung der Schnittstelle nach \enquote{ISO 29002-31 - Query for characteristic data} zu erstellen, die eben genannte automatisierte Austauschbarkeit der konzeptuell Beschriebenen Daten ermöglicht. 

Die Voraussetzung für diese Untersuchungen und Implementierung sind Vorarbeiten xxx

Das Speichern, Verarbeiten, Abfragen und Übermitteln von Stammdaten sind ein wichtiger Teil heutiger Produktions- und Servicearbeiten in Firmen. Dabei stellt sich die Frage des \enquote{Wie} diese Daten gespeichert und ausgetauscht werden können häufig nicht oder viel zu spät im Entwicklungsprozess der Industrie. Dabei ist das  \enquote{Wie} die entscheidene Frage, auf die eine Lösung gefunden werden muss; denn ist diese Frage nicht beantwortet stößt jedes Unternehmen in der Industrie früher oder später auf Probleme. Heterogene Daten und Systemlandschaften sind oft die Folge. Weiter betrachtet wirkt sich das auf den gesamten Lifecycle eines Produktes oder einer Verarbeitung aus. Man denke nur daran, dass ausgehend vom Bestellprozess eines Klienten nicht genau klar definiert ist welches Produkt er bestellen möchte. Alternativ kann es mehrere Ausprägungen geben und pro Bestellung ist jeweils ein menschliches Eingreifen zur Klärung nötig. 


\clearpage
\chapter*{Abstract}

\addcontentsline{toc}{chapter}{Abstract}

With the beginning of the 21st century, and the global success of the internet in all households and businesses, collecting and processing data is more and more in the daily focus. Buzzwords like \enquote{Big Data}, \enquote{Cloud Services} and \enquote{Mobile Devices} are broadcast through the internet and in the media. Data will be collected in nearly every area like Multimedia-data as pictures, audio or video data, or other data types like personal and product data. This data could for example be collected by Mobile devices as well as cars or trucks or through websites in the internet. Cars and trucks collect data in type of position, current speed or via sensors the temperature or barometric pressure. Websites collect and analyse data regarding the user's behavior on the website. Such data will then be saved and further processing will take place as the amount of data is enormous. 

When collecting such volumes of data, the quality of the data is not always the most important factor. In this context with the term \enquote{quality} the precise conceptual description of data is meant. The data will be collected and later be analysed through a time-consuming process. 
Considering master data in the industry, manufacturing and trade there can be observed data entities like customer, supplier, material, economic goods or simply personal data regarding employees. The quality of the data has a pretty high importance here. Companies create catalogues and descriptions of their data so that the information can be exchanged between several systems. It is very important to know precisely what the datasets mean in their context. A manufacturer for hexagon bolts, which are used to mount tyres on cars, needs exact information, description and definition of this particular product for its production division. Furthermore Sales needs the information to sell the product to interested customers and provide them with all information about the product. Last but not least, the customer needs precise information about the product, in his design and production department, for the parts list which defines the needed parts for his concrete product. Master data is needed in nearly every area, in supply chain, design and production as well as \glslink{PLM}{Product-Lifecycle-Management}. 

If the description of the data is not complete or not precise or information is missing at some point in the supply chain or in  \glslink{PLM}{Product-Lifecycle-Management}, this can lead to problems and higher costs.
The main goal is to describe data in concepts, make those concepts available and create standards which allow to automate the data flow. The automated exchange of product data is common in practice and this is realised through business processes. A precise model of the product data must be defined to transfer the product data from sender to receiver and this is often realised with a XML-representation of the data. The model is defined via schema, so that the sender and the receiver understand the meaning of the data and their properties. 

The problem with automated product data transfer is that interfaces and schemas are fixed and unflexible. Modifications and updates on interfaces and schemas are time consuming and costly, however, flexibility is rather important nowadays as we must react fast and efficiently on changes to specifications. 
 
All current thesis work in the area of studies around \gls{PLIB} topics, which describes a data model for dictionaries and libraries, consider the above mentioned problems. To support solving the problems several ISO-standards regarding these topics will be considered and implemented. 

This thesis considers the problem regarding automated transfer of characteristic product data based on the \gls{PLIB} data model. The ISO-standard \enquote{ISO 29002-31 - Query for characteristic data}, which is a description of a flexible \glslink{Abfrageschnittstelle}{query- response interface}, will be implemented as a \gls{Webservice} due to this purpose. 

The ISO-standard \enquote{ISO 29002-31 - Query for characteristic data} is the main query-interface, which references \glslink{Ontologie}{ontologies} via unique \glslink{IRDI}{identifier} (IRDI).
The response contains the data of the properties of the requested product data. An ontology is needed for that as well which will be described in \enquote{ISO 29002-10 - Characteristic data exchange format}. It describes the data format of the data responded. The data source for the \gls{Webservice} is the PLIB-database, an implementation of a product database according to ISO 13584-42. 

Thus meaningful use cases are a conjunction of several Standards. 
A simple use case is, for example, the request for all properties of a product which is identified by a unique \glslink{IRDI}{identifier}. The request will be sent via XML according to ISO 29002-31 and to a \glslink{Webservice}{webservice} and then processed. The response will be sent according to ISO 29002-10 as an XML file to the consumer. More complex cases will restrict the request to a specific value range of the requested properties or will select specific properties only. These will be then sent back as response. This \glslink{Abfrageschnittstelle}{query- response interface} correlates pretty much to a flexible \glslink{Abfrageschnittstelle}{query- response interface} like it is known by SQL. 

During the analysis and implementation several challenges and problems can occur which must be handled. For example, heterogeneity in the data representation of the called database procedures which will deliver the requested data or technical problems with the generation of the model classes for further processing in the system. Another problem is that a specific data type used in the procedures of the Oracle database has no support in the used programming language and framework.  
 


% Vorwort
%\input{chapter/04_vorwort}

%% Inhaltsverzeichnis %%%%%%%%%%%%%%%%%%%%%%%%%%%%%%%%%%%%%%%
\tableofcontents %Inhaltsverzeichnis
\cleardoublepage %Das erste Kapitel soll auf einer ungeraden Seite beginnen.

\pagestyle{headings} %%Ab hier die Kopf-/Fusszeilen: headings / fancy / ...
%\pagestyle{scrheadings}
%\clearscrheadfoot
%\ohead{\pagemark}
%\ihead{\headmark}
%\setheadsepline{0.3pt}
%\setfootsepline{0.3pt}

%\pagestyle{fancy}

%\renewcommand{\headrulewidth}{0.4pt}
%\renewcommand{\footrulewidth}{0.4pt}
%\fancyhead[RO,LE]{Masterarbeit}
%\fancyfoot[LE,RO]{\pagemark}

%% Abbildungsverzeichnis
\clearpage
\addcontentsline{toc}{chapter}{Abbildungsverzeichnis}
\listoffigures

%% Tabellenverzeichnis
\clearpage
\addcontentsline{toc}{chapter}{Tabellenverzeichnis}
\listoftables

%% The List of Listings
\clearpage
\addcontentsline{toc}{chapter}{Listingverzeichnis}
\lstlistoflistings

%% Glossar
% zeige Glossar in der Überschrift als "Glossar" an
\renewcommand{\glossaryname}{Glossar}
% zeige Glossar im Inhaltsverzeichnis als "Glossar" an
\deftranslation{Glossary}{Glossar}
%Glossar ausgeben
%\printglossary[style=altlist,title=Glossar]
\printglossaries

%Abkürzungen ausgeben
%\deftranslation[to=German]{Acronyms}{Abkürzungsverzeichnis}
%\printglossary[type=\acronymtype,style=long]

%Symbole ausgeben
%\printglossary[type=symbolslist,style=long]

%alle glossareintr‰ge gesammelt aus dem anhang
\newglossaryentry{Webservice}{name=Webservice, description={Oft auch als Web Service oder web service (Englisch) geschrieben. Stellt einen Web Dienst dar}}

\newglossaryentry{SOAP}{name=SOAP, description={Simple Object Access Protocol. Protokoll für Nachrichten, diese werden zwischen Webservice-Konsument und Webservice-Anbieter ausgetauscht}}

\newglossaryentry{WSDL}{name=WSDL, description={Web Service Description Language. W3C-Standard für die Beschreibung des Services und der Daten, die zwischen Konsument und Anbieter ausgetauscht werden}}

\newglossaryentry{Spring}{name=Spring, description={Ein nicht invasives Open Source Applikationsframework mit dem Ziel, die Softwareentwicklung zu vereinfachen}}

\newglossaryentry{JSF2.0}{name=JSF2.0, description={Java Server Faces in der Version 2.0 - Das ist ein komponentenbasiertes Framework zur Entwicklung von Weboberflächen}}

\newglossaryentry{Jersey}{name=Jersey, description={Framework zur Erstellung von RESTful Web Services in Java}}

\newglossaryentry{Apache Tomcat}{name=Apache Tomcat, description={Web Container zum Ausführen und Ausliefern von Webseiten basierend auf Java}}

\newglossaryentry{REST}{name=REST, description={Representational State Transfer. Ein Architekturmodell basierend auf dem HTTP-Protokoll. Häufig genannt in Verwendung mit RESTful Webservices}}

\newglossaryentry{PLIB}{name=PLIB, description={Parts Library gemäß ISO 13584-42, beschreibt ein Datenmodell für Dictionaries und Bibliotheken }}

\newglossaryentry{Annotation}{name=Annotation, description={Metainformation in der Java Programmierung. Wird zur Reflektion des Codes genutzt}}

\newglossaryentry{HTTP-Methode}{name=HTTP-Methode, description={Eine HTTP-Anfrage an einen Server, gemäß des Standards, z.B. eine GET-, POST-, PUT- oder DELETE Anfrage.  Es werden dadurch Ressourcen auf einem Server abgefragt, geändert oder gelöscht}}

\newglossaryentry{HTTP}{name=HTTP, description={Hyper Transfer Protocol. Protokoll zur Übertragung von Daten über ein Netzwerk. Wird in der Anwendungsschicht angesiedelt und ist das hauptsächlich eingesetzte Protokoll, um Webseiten im Internet (World Wide Web) zu übertragen}}

\newglossaryentry{MIME-Type}{name=MIME-Type, description={Internet Media Type, wird auch Content-Type genannt. Klassifiziert die Daten im Kopfbereich einer Nachricht. Hiermit wird dem Empfänger mitgeteilt, welche Art der Daten gesendet werden. Das können beispielsweise reine Textdaten, Bild-, XML- oder Videodaten sein}}

\newglossaryentry{Namespace}{name=Namespace, description={XML-Namensräume werden benutzt, um mehrere verschiedene Vokabulare in einer XML-Datei zu unterscheiden}}

\newglossaryentry{JAXB}{name=JAXB, description={Java Architecture for XML Binding, Programmierschnittstelle, die Daten aus XML-Schemata an Java-Klassen bindet}}

\newglossaryentry{pom}{name=pom.xml, description={Project Object Model - Konfigurationsdatei eines Maven Projektes}}

\newglossaryentry{Maven}{name=Maven, description={Ein Build-Management-Tool der Apache-Foundation. Wurde mit Java entwickelt und ermöglicht, Java-Programme standardisiert zu erstellen und zu verwalten}}

\newglossaryentry{Unmarshalling}{name=Unmarshalling, description={Beschreibt Umwandeln von XML Daten in programmatisches Modell, z.B. in Java Klassen}}

\newglossaryentry{Marshalling}{name=Marshalling, description={Beschreibt das Umwandeln von einem programmatischen Modell, z.B. einer Java Klasse, in XML Daten}}

\newglossaryentry{URL}{name=URL, description={Uniform Resource Location, identifiziert und lokalisiert eine Ressource, wie beispielsweise eine Webseite oder ein Bild über die verwendeten Netzwerkprotokolle}}

\newglossaryentry{URI}{name=URI, description={Uniform Resource Identifier, identifiziert eine abstrakte oder physische Ressource wie z.B. eine Webseite, ein E-Mail Empfänger oder eine Person}}

\newglossaryentry{VCS}{name=Versionskontrollsystem, description={Wird zur Versionswerwaltung eingesetzt. Erfasst Änderungen an Dokumenten oder Dateien, um diese später wiederherstellen zu können. Unterstützt häufig das gemeinsame Bearbeiten von Dateien und Dokumenten}}

\newglossaryentry{IRDI}{name=IRDI, description={International Registration Data Identifier, global eindeutiger Identifizierer für ein Objekt oder Konzept}}

\newglossaryentry{item}{name=Teil, description={Mit Teil, (englisch) Item oder auch Instanz genannt, ist eine Instanz eines konkreten Konzeptes der Teiledatenbank gemeint. Dieses Teil referenziert das Konzept und enthält konkrete Datensätze samt Wert, z.B. könnte es den Wert einer Eigenschaft eines Konzeptes enthalten}}

\newglossaryentry{Servlet}{name=Servlet, description={Servlets sind Java Klassen, welche nur innerhalb eines Webservers laufen. Diese nehmen Anfragen von Clients entgegen und beantworten sie}}

\newglossaryentry{ECCMA}{name=ECCMA, description={Electronic Commerce Code Management Association ist eine internationale nicht gewinnorientierte Organisation, mit dem Ziel bessere Qualitätsstandards für elektronische Daten zu erforschen, zu entwickeln und zu verbreiten}}

\newglossaryentry{Oracle}{name=Oracle, description={Oracle Corporation, einer der weltweit größten Softwarehersteller. Bekanntestes Produkt ist das Datenbankmanagementsystem Oracle Database. Oracle übernahm im Jahre 2009 Sun Microsystems und somit die Entwicklung der Java Programmiersprache}}

\newglossaryentry{Use Case}{name=Use Case, description={Use Case oder auch Anwendungsfall, beschreibt meist in Textform das Verhalten eines bestimmten Systems in Interaktion mit einem Benutzer in verschiedenen Szenarien}}

\newglossaryentry{POST}{name=POST, description={Eine HTTP Anfrage Methode, POST wird für das Übertragen von theoretisch unbegrenzter Menge an Daten an einen Server eingesetzt}}

\newglossaryentry{Stakeholder}{name=Stakeholder, description={Stakeholder sind Projektbeteiligte, gleichsam Personen oder Institutionen und Dokumente, die in irgendeiner Weise vom Betrieb des Systems betroffen sind}}

\newglossaryentry{Ontologie}{name=Ontologie, description={Der Begriff Ontologie bezeichnet das Studium der Kategorisierung von Dingen, die existieren oder in einigen Interessensgebieten bestehen können. Das Ergebnis einer solchen Studie, eine Ontologie, ist ein Katalog von Typen von Dingen D, von denen man annimmt, dass sie aus der Sicht einer Person, die eine Sprache L benutzt, um über diesen Interessensbereich D zu sprechen, existieren}}

\newglossaryentry{Interoperabel}{name=Interoperabilität, description={Die Fähigkeit homogener Systeme Informationen effizient auszutauschen}}

\newglossaryentry{PLM}{name=Produktlebenszyklusmanagement, description={Auch Product-Lifecycle-Management,  ist ein Konzept zur nahtlosen Integration sämtlicher Informationen, die im Verlauf des Lebenszyklus eines Produktes anfallen}}

\newglossaryentry{Abfrageschnittstelle}{name=Abfrage- und Antwortschnittstelle, description={Auch Query-Response-Schnittstelle. Eine flexible Schnittstelle über die vom Klienten eine Anfragenachricht an einen Server gesendet wird, welcher eine Antwortnachricht zurückliefert. Die Anfragen sind flexibel dadurch, dass Werte- und Attributeinschränkungen gemacht werden können}}

\newglossaryentry{IG}{name=Identification Guide, description={Ein Identification Guide beschreibt eine Menge von Regeln für die Beschreibung von Teilen (Items), welche zu einer bestimmten Klasse gehören. Hierbei werden die Eigenschaften und Klassendefinition zu Konzepten eines Dictionaries verlinkt}}

\newglossaryentry{JUnit}{name=JUnit, description={Ein Framework basierend auf Java, um Unit Tests zu entwickeln und auszuführen}}

\newglossaryentry{Unit-Test}{name=Unit Test, description={Ein Unit Test ist ein Test der Funktionalität einer Komponente ohne Einbindung der Abhängigkeiten von anderen Komponenten. Dieser Test ist somit ein Whiteboxtest, da die Komponente völlig autark auf Funktionalität getestet wird}}

\newglossaryentry{Integrationstest}{name=Integrationstest, description={Ein Integrationstest ist ein Test einer Komponente integriert in die System- oder Komponentenlandschaft. Es wird die Komponente in Zusammenarbeit mit anderen Komponenten im Rahmen von Szenarios getestet. Z.B. Test einer grafischen Benutzeroberfläche mit Speicherung von Eingabedaten in der Datenbank}}

\newglossaryentry{Applikationskontext}{name=Applikationskontext, description={Auch application context. Die Ausführungsumgebung einer Software Applikation innhalb eines Applikationsservers. Wird an eine URL gekoppelt, unter der die Applikation erreichbar ist}}

\newglossaryentry{Dictionary}{name=Dictionary, description={Ein Verzeichnis, welches Daten zu Konzepten enthält. Diese Konzepte werden mittels Eigenschaften präzise beschrieben, sodass im besten Falle die Beschreibung eindeutig ist}}

\newglossaryentry{Excel}{name=Excel, description={Microsoft Excel ist ein bekanntes Tabellenkalkulationsprogramm der Firma Microsoft.}}





%% Kapitel / Hauptteil des Dokumentes %%%%%%%%%%%%%%%%%%%%%%%

%\pagenumbering{arabic}


\chapter{Einleitung}\label{sec:einleitung}

...
			
\section{Abgrenzung}\label{sec:abgrenzung}
 

		
%\setchapterpreamble[u]{%
%\dictum[Johann Wolfgang von Goethe]{Es ist nicht genug, zu wissen, man muß auch anwenden; es ist nicht genug, zu wollen, man muß auch tun. \dots}}
\chapter{Aufgabenbeschreibung} \index{Aufgabenbeschreibung}\label{Aufgabenbeschreibung}

\section{Abstrakte Beschreibung}

Es soll eine einfach nutzbare Abfrageschnittstelle (Query-Schnittstelle) für die bereits vorhandene PLIB Produktteiledatenbank/Dictionary Implementierung nach ISO-13584 des Fachbereiches erstellt werden. Die Abfrageschnittstelle soll Abfragen in der Funktion analog zu SQL\footnote{Standard Query Language - Abfragesprach für relationale Datenbanken} ermöglichen. 
Das sind folgende Möglichkeiten und Operationen:
\begin{description}
\item[select] Das Keyword um in SQL Attribute zu selektieren.  
\item[where] Das Keyword um in SQL Einschränkungen vorzunehmen. 
\end{description}
 
Diese Abfrageschnittstelle soll die Standards der ISO 29002-31 sowie für deren Nutzung weitere nötige Standards unterstützen, wie z.B. ISO 29002-10. Diese Schnittstelle soll für Menschen sowie für Maschinen einfach nutz- und lesbar sein und sollen technisch auf bereits vorhandenen Abfrageprozeduren der Datenbankebene basieren. Die Implementierung dieser Prozeduren und des Datenbankschemas ist Teil der Arbeit zweier Kommilitonen (Herr Mende und Herr Loth).

\section{Zielsetzung}

Das Hauptziel ist die Machbarkeit und die Integration der ISO-Schnittstelle aufbauend auf die vorhandene PLIB Datenbankimplementierung des Fachbereiches aufzuzeigen. 
Ferner ist mit aktuellen Techniken eine Integration, gleichsam Nutzbarkeit der Abfrage der Datenbank zu ermöglichen. Wichtig ist die gegebenen ISO Normen zu unterstützen, um Wiederverwendbarkeit zu gewährleisten. Falls Teile der Implementierung von den ISO Normen abweichen, wird darauf mit ausführlicher Erläuterung der Gründe hingewiesen. 

\subsection{Anwendungsfälle und technische Implementierungsgrundlage}

Um das Ziel zu erreichen, soll im Rahmen der Arbeit untersucht werden welche mögliche sinnvollen Anwendungsfälle sich in der Praxis basierend auf den zu unterstützenden ISO Standards ergeben. Dafür ist eine Analyse der Standards nötig um ebenfalls zu prüfen, wie weit und in welcher Art die Abfragemöglichkeiten analog zu SQL unterstützt werden können. Für die Schnittstelle ist eine prototypische Implementierung zu erstellen. Weiterer Bestandteil der Arbeit ist unter Beachtung der technischen Vorgaben eine Analyse der marktaktuellen technischen Optionen mit anschließender Beschreibung des Auswahlprozesses der Techniken/Plattformen/Architektur und Programmiersprachen. Die Vor- und Nachteile der einzelnen Optionen des Auswahlprozesses und weitere Nutzungs-, respektive Erweiterungs- und Integrationsmöglichkeiten der entwickelten Schnittstelle sind zu erläutern. 
Die Entwicklung der Software soll nach einem aktuell üblichen Softwareentwicklungsprozess erfolgen. Ein enstprechender Prozess ist auszuwählen und zu dokumentieren. 

\section{Details und Abgrenzung}

Dieses Kapitel beschreibt Details und den Kontext der Aufgabenstellung sowie die Abgrenzung zu weiteren Standards und Abschlussarbeiten anderer Studenten. 

\subsection{Abgrenzung}

Die Arbeit umfasst die Implementierung der Use Cases nach Kapitel \ref{kap:Use_Cases}. Dies beinhaltet im wesentlichen den Teil 31 der ISO 29002 - einen Abfragestandard für Charakteristische Produktdaten. Weiterhin wird für die Datenübertragung eine Implementierung des Teils 10 der ISO 29002 benötigt (für eine Begründung siehe Kapitel XXX). Die Arbeit befasst sich nicht mit der Implementierung eines Identification Guides nach ISO 22745-30. Dieser wird in der Praxis für eine sinnvolle Vorauswahl auf Seite des Clients der benötigten Attribute der Produkte verwendet. Jeder Klient definiert für seinen Kontext sinnvolle Attribute und Produktdaten und definiert diese mit Hilfe des Schemas der ISO 22745-30. Dies kann mittels eines Webformular auf Klientseite erfolgen oder als allgemeines Formular mit z.B. Excel, welches die für den/die Klienten relevanten Attribute der Produkte enthält die abgefragt werden sollen. 

\subsection{Details der Aufgabe}

\subsubsection{Vorgaben}

Für die Implementierung sind folgende Anforderungen gegeben:
\begin{description}
\item[Datenbanksystem Oracle] beinhaltet die PLIB Datenbank samt Prozeduren und stellt als Dictionary und Produktdatenbank die Basis dar. Dies wird vom Fachbereich bzw. von den Studenten Herr Mende / Herr Loth gestellt. 
\item[Web Services] Die Schnittstelle soll auf Grund der hohen Verbreitung und Integrationsmöglichkeiten als Web Service entwickelt werden. ISO 29002-31 schlägt als Beispiel eine E-Mail Schnittstelle vor.  Dies ist aber keine Voraussetzung. 
\end{description}



\setchapterpreamble[u]{%
\dictum[René Descartes]{Vor eine Frage gestellt, die wir vollständig verstanden haben, müssen wir sie von jeder überflüssigen Darstellung abstrahieren, sie auf ihre einfachste Form reduzieren und sie in möglichst kleine Teile zerlegen, die wir dann aufzählen. \dots}}
\chapter{Analyse und Definition der Anforderungen } \index{Analyse und Definition der Anforderungen}\label{kap:analyse_und_definition}

Die ISO Standards 29002-31 und 22745-30 wurden vom Fachbereich als mögliche zu implementierende Standards genannt, um die Anforderungen einer Abfrageschnittstelle analog zu den Möglichkeiten von SQL (Projektion, Selektion) zu gewährleisten.   

Mittels den Erkenntnissen der Analyse werden mögliche Use Cases erarbeitet. Wichtig ist es, zu evaluieren, welche Standards gegebenenfalls betrachtet und implementiert werden müssen, um die Anforderungen umzusetzen. 

\section{Analyse der ISO 29002-31 - Exchange of characteristics data}\label{kap:analyseiso2900231}\index{ISO 29002-31}

In diesem Abschnitt wird auf die Analyse der vom Fachbereich zur Erfüllung der Anforderungen angegebenen ISO 29002-31 eingegangen. Ziel der Analyse ist es, herauszufinden ob die ISO-Norm zur Erfüllung der Anforderungen ausreichend ist, wie mit der ISO-Norm die Anforderungen umgesetzt werden können und ob gegebenenfalls weitere ISO Normen zur Erfüllung der Anforderungen betrachtet werden müssen. 

Die gesamte Analyse der ISO 29002-31 mit Beschreibung der einzelnen Datencontainer findet sich in \autoref{kap:analyse2900231}. Die Analyse zeigt anhand von Beispielen auf, welche Abfragen mit der ISO 29002-31 möglich sind.

Zusammenfassend aus der Analyse ergeben sich folgende Abfragemöglichkeiten:
\begin{description}
\item[Simple Query] Eine Abfrage nach eindeutigem Identifier eines Konzeptes, mit der Möglichkeit der Einschränkung nach Eigenschaften eines Konzeptes (Selektion). Ferner ermöglicht der Simple Query die Übergabe bekannter Daten eines konkreten Teiles, nach denen gesucht und entsprechend passendes Ergebnis geliefert wird. Siehe dazu \autoref{sec:query}.
\item[Parametric Query] Ermöglicht die Abfrage nach eindeutigem Identifier eines Konzeptes und zusätzlich die Einschränkung einer Eigenschaft nach konkretem Wert (Projektion). Dies erfolgt mittels Angabe einer characteristic\_data\_query\_expression. Siehe dazu \autoref{sec:characteristicdataqueryexpression}.
\end{description}

Die konkreten Query Beispiele finden sich dazu in \autoref{kap:query_beispiele}.

\section{Analyse ISO 22745-30 - Identification Guide}\label{kap:identification_guide}\index{ISO 22745-30}

Als weiterer möglicher zu betrachtender Standard wurde ISO 22745-30 - Identification Guide, genannt. 

Ein Identification Guide beschreibt, welche Daten für ein Objekt benötigt werden, damit dies überhaupt sinnvoll für einen bestimmten Zweck eingesetzt werden kann. Der Käufer, Produktmanager oder Benutzer definiert die Anforderungen an die Daten. Ein  \enquote{Datenanforderungsstatement} wird als ein i-xml identification guide xml file erzeugt. Siehe dazu auch \autoref{fig:lieferketten}. 

Es wird die Frage beantwortet, welche Daten und Eigenschaften zu einem bestimmten Konzept eines Objektes benötigt werden, um den Artikel beispielsweise zu kaufen oder sinnvoll zu verwalten. Diese Anforderungen werden von der Abfrageseite (Kundenseite) definiert, also diejenige, die Daten abfragen möchte\citep[Vergl.][]{bensonQuality}. 

Ein Identification Guide referenziert Konzepte eines Dictionaries um Datenanforderungen einer bestimmten Klasse zu beschreiben \citep[Vergl.][Kap. 5]{iso22745-30}.

Ein Datenempfänger kann eine Organisation oder eine Gruppe von Organisationen oder Firmen sein, welche ähnliche Datenanforderungen haben. Somit wird eine Identification Guide Gruppe von einer speziellen Organisation verwaltet, welche wiederum selbst Datenempfänger sein kann.  

Es kann somit festgehalten werden, dass für die Erfüllung der Aufgabenbeschreibung nach \autoref{kap:aufgabenbeschreibung} eine Implementierung des Standards ISO 22745-30 nicht notwendig ist, da dieser Standard eine zusätzliche Einschränkung der Konzepte und benötigten Daten auf Anfrageseite beschreibt. 
Die Abfrage gemäß einer SQL-Selektion und -Projektion ist mit dem Standard 29002-31 möglich. 

\section{Anwendungsfälle}\label{kap:Use_Cases}

Dieser Abschnitt beschreibt mögliche Anwendungsfälle\footnote{Use Case. Im weiteren Verlauf wird, gerade auch in den Anwendungsfall Beschreibungen von Use Cases gesprochen. Im deutschen Sprachraum ist das englische allgemein bekannte Äquivalent \enquote{Use Case} für Anwendungsfall eher geläufig als das deutsche, daher wird darauf verzichtet an manchen Stellen die Begrifflichkeiten einzudeutschen.} die sich aus ISO 29002-31 ergeben. 
Es wird nur beispielhaft ein Use Case aufgelistet und erläutert. Alle weiteren Use Cases sind in \autoref{kap:analyse_use_cases} zu finden. 

Die Query-Code-Beispiele sind gekürzt, d.h. es werden die referenzierte Schemata-Namen gemäß XSD nicht aufgeführt. 

\bild{usecases_plib.png}{13cm}{Use Case Übersicht}{Use Case Übersicht}

\subsection{Akteure}
Bei der Implementierung geht es um die Erstellung einer Software-Schnittstelle als \gls{Webservice}. Das bedeutet, dass diese Schnittstelle ohne eine Benutzeroberfläche nicht sinnvoll für einen menschlichen Akteur nutzbar ist. In den nachfolgenden Anwendungsfällen wird vom Akteur \enquote{Klient} gesprochen. Der Klient ist allgemein ein Nutzer der Schnittstelle, sei es als menschlicher Akteur welcher über eine Bedienerinterface die Schnittstelle benutzt oder eine direkte Maschinennutzung. Ein Beispiel für eine Maschinennutzung kann einen Anwendung sein, welche automatisiert die Schnittstelle aufruft und Daten abfragt.    

\subsection{Use Case Beschreibungen}

Als Vorlage für die \gls{Use Case} Beschreibungen wurden die \glspl{Use Case} aus dem Kurs Software Engineering I entnommen, da diese sehr übersichtlich, knapp und präzise sind \citep[Vgl.][S. 120ff]{sixse1}. 

\subsubsection{Alle Charakteristische Daten eines Produkts abfragen}

{\small

\begin{description}
     \item[use case] Charakteristische Daten abfragen
     \item[  actors]~\\
     Klient
     \item[  precondition]~\\
     Der Klient verwendet einen gültigen Identifier.
     \item[  main flow]~\\
     Der Klient gibt einen eindeutigen Identifier (\gls{IRDI}\footnote{International Registration Data Identifier}) eines Konzeptes von Elementen ein und sendet eine Anfrage ab. Die Anfrage wird auf Gültigkeit überprüft. Als Antwort bekommt er ein oder mehrere Datensätze von Elementen \footnote{Item, ISO 29002-10 Kapitel 5.3.2} mit den entsprechenden charakteristischen Daten \footnote{property\_values, ISO 29002-10 Kapitel 5.2.4}  des Elementes mit dem übergebenen Identifier zurück.
     \item[  postcondition]~\\
     Alle Daten aller Elemente der gewählten Konzepte des Identifiers wurden zurückgegeben.    
     \item[  alternative flow] Properties auswählen ~\\
     Zusammen mit dem Identifier übergibt der Klient einen oder mehrere Property-Identifier und sendet diese erweiterte Anfrage ab.    
     \item[  postcondition]~\\
     Die mittels Property-Identifier ausgewählten Daten aller Elemente der gewählten Konzepte wurden zurückgegeben.    
     \item[end] Charakteristische Daten abfragen
\end{description}

~\\

} %end small

\subsubsection{Beispiel}\label{lab:schraubendreher}

Ein Schraubendreher könnte folgendermaßen in einer Produktdatenbank repräsentiert werden:

\begin{description}
\item[Klassen-Identifier] 0173-1\#01-AAA352\#4 
\item[Länge] 300mm
\item[Typ] Kreuz
\item[Spannungsfest] ja
\end{description}

Korrekterweise müssten anstatt der Attribute wie Länge oder Typ ebenfalls ein Identifier stehen. Die Benamungen sind hier zur besseren Lesbarkeit aufgelöst. 

Um alle Eigenschaften (Properties), wie Länge, Typ und Spannungsfest zu erhalten, muss folgende Abfrage gesendet werden: 
\begin{quotation}
\enquote{Gib mir alle \glslink{item}{Teile} und alle Properties der Klasse mit dem Identifier 0173-1\#01-AAA352\#4 (Schraubendreher).}
\end{quotation}

Das Ergebnis ist ein \gls{item} mit allen Attributen (Properties) der gewünschten Konzepte und gegebenenfalls vorhandenen Unterkonzepte. In unserem Falle genau die oben angegebenen Werte.

Die XML-Abfrage sieht wie folgt aus:

\begin{lstlisting}[caption=Query Beispiel - Daten abfragen, language=XML, label=UseCaseDatenabfragen]
<?xml version="1.0" encoding="UTF-8"?>
<qy:query xsi:schemaLocation="...query query.xsd" xmlns:xsi="http://www.w3.org/2001/XMLSchema-instance" xmlns:cat="...catalogue" xmlns:val="...value" xmlns:qy="...query" xmlns:bas="...basic">
	<qy:class_ref>0173-1#01-AAA352#4</qy:class_ref>
</qy:query>
\end{lstlisting}

Eine Abfrage, welche mittels Selektion die Properties der Klasse auswählt, die zurückgeliefert werden sollen, könnte lauten: 
\begin{quotation}
\enquote{Gib mir alle \glslink{item}{Teile} und die Properties Länge und Typ der Klasse mit dem Identifier 0173-1\#01-AAA352\#4 (Schraubendreher).}
\end{quotation}

Das Ergebnis ist ein \gls{item} mit den gewünschten Attributen (Properties). 

Die XML-Abfrage:
\begin{lstlisting}[caption=Query Beispiel - Daten abfragen mit Propertyeinschränkung, language=XML, label=lst:UseCaseDatenabfragenProperty]

<?xml version="1.0" encoding="UTF-8"?>
<qy:query xsi:schemaLocation="...query query.xsd" xmlns:xsi="http://www.w3.org/2001/XMLSchema-instance" xmlns:cat="...catalogue" xmlns:val="...value" xmlns:qy="...query" xmlns:bas="...basic">
	<qy:class_ref>0173-1#01-AAA352#4</qy:class_ref>
	
	<!-- identifier von typ und laenge werden uebergeben -->
	<qy:property_ref>0173-1#01-BBB111#1 0173-1#01-BBB222#1</qy:property_ref> 
	
</qy:query>
\end{lstlisting}

\autoref{lst:UseCaseDatenabfragenProperty} beinhaltet ein XML-Attribut property\_ref. Das wird mit gewünschten Property Identifier gefüllt, welche mit Leerzeichen getrennt werden. 


\section{Zusammenfassung des Kapitels}

Das Ergebnis der Analyse der Standards ISO 29002-31 und 22745-30 ergibt, dass für die Erfüllung der Anforderungen einer Abfrageschnittstelle analog zur SQL-Selektion und -Projektion der Standard ISO 29002-31 ausreichend ist. Sowohl Projektion als auch Selektion ist dadurch möglich. ISO 22745-30 beschreibt als Identification Guide eine weitere klientseitige Einschränkung der abzufragenden Daten. Man stelle sich eine Einschränkung (Definition einer Teilmenge) durch ein Formular vor, das Formular schränkt die abzufragenden Daten auf die tatsächlich benötigte Menge von Daten gegebenenfalls stark ein und gibt ebenfalls den Typ der Daten vor.

Aus den Anforderungen wurden sinnvolle \glspl{Use Case} formuliert, welche das Verhalten des Systems beschreiben sollen.  

\setchapterpreamble[u]{%
\dictum[Douglas Adams]{A common mistake that people make when trying to design something completely foolproof is to underestimate the ingenuity of complete fools. \dots}}
\chapter{System- und Softwareentwurf} \index{System- und Softwareentwurf}\label{kap:systemundsoftwarentwurf}


Dieses Kapitel beschreibt den System- und Softwareentwurf sowie die Auswahl der Umgebung, Plattform, Software, Programmiersprache und Frameworks.

\section{Auswahlprozess}

Teil der Aufgabe der Arbeit ist es, für das System im Rahmen der nichtfunktionalen Anforderungen eine geeignete Umgebung zu schaffen. Dafür sind einige Entscheidungen zu treffen. 

\subsection{Webservice}\index{Webservice}
Wenn von Webservices gesprochen wird, dann werden meistens \gls{SOAP}-basierte Webservices gemeint. Allerdings gibt es die sogenannten \gls{SOAP}-basierten Webservices als auch die RESTful Webservices. Anforderung ist es ein Web Service zu implementieren. Dazu muss entschieden werden, ob ein \gls{SOAP}-basierter oder RESTful Webservice implementiert werden soll. \index{REST@\textbf{REST}} \index{REST!RESTful}

\subsubsection{Definition}

\begin{quotation}
\enquote{Web services provide the means to integrate disparate systems and expose reusable business functions over HTTP. They either leverage HTTP as a simple transport over which data is carried (e.g., SOAP/\gls{WSDL} services) or use it as a cimplete application protocol that defines the semantics for service behavior (e.g. RESTful services) \citep[S. 2][]{robinsonService}}	
\end{quotation}

\subsubsection{SOAP/WSDL Webservice}\index{SOAP}\index{WSDL}\index{Webservice}
Frau Janssen hat in ihrer Abschlussarbeit einen Webservice nach ISO 29002-20 mittels einem SOAP/WSDL Webservice implementiert \citep[vgl.][]{janssen}. Die ISO 29002-20 verweist in Annex-B auf entsprechende WSDL-Definitionen. 
Ich möchte an dieser Stelle darauf verzichten, die Einzelheiten eines SOAP/WSDL Webservice zu erläutern und verweise auf Frau Janßens Abschlussarbeit \citep[vgl.][Kap. 3]{janssen}. 

Es sei an dieser Stelle kurz erwähnt, dass Webservices auf SOAP/WSDL basierend ein W3C\footnote{World Wide Web Consortium - http://www.w3.org} Standard sind. 

\subsubsection{RESTful Webservice}\index{RESTful Webservice}\index{REST@\textbf{REST}} \index{REST!RESTful} \index{REST!RESTful Webservice}
RESTful Webservices sind per se kein Standard sondern eher ein Programmierparadigma respektive ein Architekturmuster. 
\todotext{Weiter ausführen und Quelle angeben}

\subsubsection{Fazit}
Es wurde ein RESTful Webservice ausgewählt. Die Gründe stellen sich wie folgt dar:

\begin{description}
\item[Einfache Implementierung] Da RESTful Webservices auf dem HTTP Protokoll basiert und ferner sehr gute geeignete Frameworks für Java vorhanden sind, siehe \autoref{kap:bibliotheken_und_frameworks}, stellt sich für Web Entwickler die Implementierung als einfach heraus.
\item[Payload XML] Der Payload des Anfrage Queries wird als XML angegeben (siehe \autoref{fig:datenfluesse}. Als mögliche Übertragung wird email angegeben. Das bedeutet für die Anforderung, das ein Web Service  implementiert werden soll, dass die XML-Repräsentation des Queries als Payload mittels XML zu übertragen ist. Folglich könnte sowohl SOAP als auch REST benutzt werden. Voraussetzung ist eine definierten Schnittstelle, welche lediglich ein XML als Payload akzeptiert. 
\item[Kein Vorteil bei SOAP] Somit hat SOAP keinen Vorteil gegenüber REST. Der Vorteil würde darin bestehen, wenn SOAP selbst die Operationen anbietet (definiert in der WSDL). Das hat den Nachteil, dass die wohlgeformte vorgegebenen Schemata query.xsd, in der Form nicht genutzt werden können. Es muss somit eine Einbindung in SOAP erfolgen. 
\item[Vorteil des Payloads] Der klare Vorteil bei der Variante eine Query XML-Datei gemäß query.xsd Schema des Standards als Payload zu versenden ist der, dass eine Validierungsprüfung der XML gegen vorhandene definierte Regeln des Schemas erfolgen kann (z.B. gültige IRDI), ferner können aus dem Schema passende Modellklassen zur Verarbeitung und Speicherung der Query-Daten in der Applikation generiert werden. Mehr Informationen in Kapitel \todotext{Kapitel raussuchen und verlinken} (generierung mit jax). 
\end{description}

\subsection{Plattform} \index{Apache Tomcat} \label{sec:plattform}
Als Laufzeitumgebung wurde der Apache Tomcat Server in der Version 7 ausgewählt. Das ist ein üblicher Web Container (Web Server), welcher mit entsprechenden Frameworks bzw. Bibliotheken sowohl \gls{SOAP} als auch RESTful Web Services anbieten kann. 
\todotext{Kapitel weiter ausführen und Quellen angeben}

\subsection{Bibilotheken und Frameworks} \index{Bibilotheken und Frameworks}\label{bibliotheken_und_frameworks}
\begin{description}

\item[Jersey] Framework zur Erstellung von RESTful Web Services \index{Jersey}\gls{Jersey}
\item[JSF2.0] Komponentenbasiertes Web Framework zur Erstellung von Benutzeroberflächen \index{JSF2.0}
\item[Spring] Dependency Injection Framework. Bietet darüber hinaus noch weitere Komponenten an. Ausgewählt wurde unter anderem \gls{Spring} JDBC und \gls{Spring} Data Oracle, welche einfacheren Zugriff auf relationale Datenbanken sowie auf Prozeduren von relationalen Datenbanken ermöglicht. Mehr Details zur Implementierung in Kapitel \todotext{Kapitel angeben}. \index{Spring} 
    
\end{description}

\todotext{Eine detaillierte Erklärung der Frameworks nötig, Erläuterung des Nutzens und weshalb diese ausgesucht wurden. Ggfs in Anhang}

\subsection{Programmiersprache}
Vorgegeben ist die Umsetzung eines Webservices. Diese lassen sich in fast allen aktuellen Programmiersprachen entwickeln, folglich kann prinzipiell jede Sprache ausgewählt werden die Webservices anbieten kann.  

Es wurde die Sprache Java gewählt, da diese zum einen in der aktuellen Industrie stark verbreitet und zum anderen der Autor dieser Arbeit seit vielen Jahren damit vertraut ist. Ferner besteht hier eine Abhängigkeit zur Auswahl der Plattform gleichsam dem Web Kontainer, siehe \autoref{sec:plattform}. 

Ein weiterer Aspekt ist, dass Software welche mit Java entwickelt wurde im Prinzip auf jedem Betriebssystem lauffähig, und somit portierbar ist. Dies ist zwar keine Anforderung des Projektes, aber ermöglicht die Arbeit und Entwicklung in beliebigen Systemen. 

\section{Softwaredesign und Architektur}

\subsection{Bausteinsicht}\index{Software-Bausteine}
\begin{quotation}
Die Bausteinsicht bildet die Aufgaben des Systems auf Software-Bausteine oder -Komponenten ab.
 \citep[S. 98ff][]{starke}	
\end{quotation}

Es soll mit Hilfe dieser Sicht ein Überblick über den Aufbau des Systems und den Abhängigkeiten der einzelnen Komponenten geschaffen werden. Dazu wird das System im top-down Ansatz aufgezeigt und verfeinert. 

\subsubsection{Level 0 - Systemüberblick mit angrenzenden Systemen} 

\begin{figure}[htbp]
	\centering
		\includegraphics[width=0.9\textwidth]{images/bausteinsicht_plib_level0.png}
	\caption{Bausteinsicht Level 0}
	\label{fig:bausteinsicht_level0}
\end{figure}

\paragraph{Klient}

Der Klient stellt den Nutzer des Query Services dar. Er erzeugt das XML File, welches als Query an den Service geschickt wird. Der Transport erfolgt über das HTTP Protokoll.  

\paragraph{Applikationsserver}

Der Applikationsserver ist der Hauptbaustein. Dieser Baustein enthält alle entwickelten Komponenten. Sichtbar von außen ist der QueryService, dieser Service ist ein REST WebService und nimmt XML-Dateien als Payload eines POST Requests entgegen. 

\subsubsection{Level 1 - Plib characteristic query} 

Die Bausteinsicht Level 1 zeigt alle Komponenten des entwickelten Systems auf und deutet die externen Schnittstellen an. Mittels <<use>> Beziehungen erkennt man die Abhängigkeiten der einzelnen Komponenten. 

\begin{figure}[htbp]
	\centering
		\includegraphics[width=0.98\textwidth]{images/bausteinsicht_plib_level1.png}
	\caption{Bausteinsicht - Level 1}
	\label{fig:bausteinsicht_level1}
\end{figure}

\begin{description}
\item[QueryService] Der Zweck dieser Komponente ist das entgegennehmen des Requests (Query-XML File), das Weiterleiten an die entsprechenden Weiterverarbeitenden Komponenten und letztlich das Zurücksenden der Rückantwort (Katalog-XML).
\item[Data Access] Diese technische Komponente beinhaltet die Zugriffsschicht auf die externe Datenbankschnittstelle und bietet entsprechend vereinfachte Abfrageschnittstellen für die anderen Komponenten an. 
\item[Marshaller] Eine weitere technische Komponente, diese ist für das Einlesen und Validieren der eingegangenen Query-XML Datei verantwortlich. Ferner transformiert diese Komponente die Informationen aus der Query-XML nach Validierung in das im System benutze Datenmodell aus der Komponente XMLData.
\item[XMLData] Beinhaltet das Datenmodell des Systems. Sowohl die eingehenden Query-XML Daten, als auch die ausgehenden Katalog-XML Daten werden intern in ein entsprechendes Model zur Verarbeitung abgelegt, so dass darauf gearbeitet werden kann.  
\item[Analyser] \todotext{Erläuterung fehlt}
\item[Handler] \todotext{Erläuterung fehlt}
\end{description}

\subsubsection{Level 2 - Whiteboxansicht - Komponente XMLData} 

Die Komponente XMLData beinhaltet alle Datenmodelle für die beiden Hauptkonzepte \enquote{Query for characteristic data} nach 

\begin{figure}[htbp]
	\centering
		\includegraphics[width=0.82\textwidth]{images/bausteinsicht_plib_level2_xmldata.png}
	\caption{Bausteinsicht - Level 2 - Komponente XMLData}
	\label{fig:bausteinsicht_level2_xmldata}
\end{figure}

\begin{description}
\item[Query] Diese Komponente beinhaltet das Datenmodell des Queries nach ISO/TS 29002-31. 
\item[Catalogue] Diese Komponente beinhaltet das Datenmodell des Kataloges nach ISO/TS 29002-10. 
\item[Basic] Diese Komponente beinhaltet das Datenmodell von Basistypen nach ISO/TS 29002-4.
\item[Value] Diese Komponente beinhaltet das Datenmodell der Wertetypen nach ISO/TS 29002-10.
\item[Identifier] Diese Komponente beinhaltet das Datenmodell für Identifier (IRDI) nach ISO/TS 29002-5. 
\end{description}


\setchapterpreamble[u]{%
\dictum[Johann Wolfgang von Goethe]{Es ist nicht genug, zu wissen, man muß auch anwenden; es ist nicht genug, zu wollen, man muß auch tun. \dots}}
\chapter{Implementierung} \index{Implementierung}\label{kap:implementierung}

Dieses Kapitel beschreibt die Implementierung. 

\section{Configuration Management und Setup}\index{Configuration Managment}

\subsection{Maven}\index{Maven}
Maven ist ähnlich wie Ant ein Build und Deployment Werkzeug. Darüberhinaus ist es ein Dependency Management Werkzeug. Das bedeutet, dass Maven die abhängigen Artefakte und deren Versionen verwalten kann. 

\subsection{Apache Tomcat}\index{Apache Tomcat}
Als Web Container wird der \gls{Apache Tomcat} in der Version 7 eingesetzt. 

\subsection{Java 6}

\section{Web Service}\index{Web Service}\index{REST!RESTful Web Service}\label{kap:webservice}

Wie in Kapitel \todotext{XXX} erwähnt, wurde entschieden einen \gls{REST}ful Web Service zu erstellen. Dafür wurde das Jersey-Framework ausgewählt.

Folgende Schritte sind notwendig um einen \gls{REST}ful Webservice mit Jersey zu erstellen: 

\subsection{Servlet Konfiguration in web.xml} \index{Jersey}

In die Konfigurationsdatei web.xml des Webcontainers (hier \gls{Apache Tomcat}) muss das Servlet für \gls{Jersey} hinzugefügt werden, so dass \gls{Web Service} Anfragen an dieses Servlet möglich sind. 

\autoref{lst:jerseywebxmlconfig} zeigt den Ausschnitt aus der web.xml des \gls{PLIB}-Projektes. 

 \begin{lstlisting}[caption=Jersey Servlet Konfiguration in web.xml, language=XML, label=lst:jerseywebxmlconfig]
 <!-- configure jersey REST-Web Service Servlet -->
    <servlet>
        <servlet-name>jersey-servlet</servlet-name>
        <servlet-class>com.sun.jersey.spi.container.servlet.ServletContainer</servlet-class>
        <init-param>
            <param-name>com.sun.jersey.config.property.packages</param-name>
            <param-value>de.feu.plib.webservice.rest</param-value>
        </init-param>
        <load-on-startup>1</load-on-startup>
    </servlet>
 \end{lstlisting}   
 
Ferner muss in der web.xml ein sogenannter \enquote{Mappingeintrag} angelegt werden. Hierdurch wird dem Web Server mitgeteilt, bei zu welchem Servlet Anfragen an eine bestimmte URL zur Verarbeitung geleitet werden sollen. 
 
  \begin{lstlisting}[caption=Jersey Servlet Mappingkonfiguration in web.xml, language=XML, label=lst:jerseywebxmlconfigmapping]
    <servlet-mapping>
        <servlet-name>jersey-servlet</servlet-name>
        <url-pattern>/rest/*</url-pattern>
    </servlet-mapping>
 \end{lstlisting}  
 
Das Konfigurationbeispiel in \autoref{lst:jerseywebxmlconfigmapping}  besagt, dass das Servlet mit dem Namen \enquote{jersey-servlet}, welches im Beispiel \autoref{lst:jerseywebxmlconfig}  konfiguriert wurde, alle Anfragen mit der URL \enquote{/rest/*} entgegennehmen soll. Das Muster \enquote{/rest/*} bedeutet, das beliebige URLs nach /rest/ akzeptiert werden. Zum Beispiel: /rest/webservice oder /rest/service/name.

Unter der Annahme, dass die Applikation auf dem lokalem Rechner installiert wurde und auf Port 8080 lauscht, der Applikationskontext\footnote{XXX} \enquote{plib-characteristic-query} ist, ergibt sich als aktuelle Gesamt-URL für den Web Service der Applikation \enquote{http://localhost:8080/plib-characteristic-query/rest/}.
\index{Apache Tomcat!Applikationskontext} \todotext{Applikationskontext footnote}

\subsection{Web Service Klasse}
Der Einstiegspunkt für den \gls{Web Service} ist eine Klasse. Eine Applikation kann mehrere solcher Einstiegspunkte haben. Damit nun die Navigation von der URL der Anfrage zur entsprechenden Klasse funktioniert, wird jede Klasse mittels Annotation markiert und ein weiterer Pfad-Präfix definiert. Das Beispiel 
\autoref{lst:jerseywebservice} zeigt, dass mittels @Path der Suffix /ws definiert wird. 
  \begin{lstlisting}[caption=Jersey Web Service Klasse, language=Java, label=lst:jerseywebservice]
...
@Path("/ws")
public class QueryService {
...
 \end{lstlisting}  
 
Somit ergibt sich als aktuelle Gesamt-URL für den Web Service der Applikation \\  \enquote{http://localhost:8080/plib-characteristic-query/rest/ws}.
 
Der nächste Schritt ist nun, die entsprechenden Methode zu definieren, welche die Anfrage final entgegennimmt und verarbeitet (siehe \autoref{lst:jerseymethode}). 
 
  \begin{lstlisting}[caption=Jersey Methode, language=Java, label=lst:jerseymethode]
    @POST
    @Path("/query")
    @Consumes("application/xml")
    @Produces(MediaType.APPLICATION_XML)
    public String query(String queryXML) {
        LOGGER.info("Incoming query XML content :" + queryXML);
        QueryType queryType = unmarshall(queryXML);
        LOGGER.info("QueryType: " + queryType);
        CatalogueType catalogue = queryPipe.filter(queryType);

        LOGGER.info("Filled Catalogue: " + catalogue);
        String marshalledCatalogue = marshall(catalogue);

        LOGGER.info("Marshalled catalogue: " + marshalledCatalogue);
        return marshalledCatalogue;
    }
 \end{lstlisting}  

Die Konfiguration der Methode wird über \Gls{Annotation} vorgenommen. Nachfolgend die Erklärung der \Gls{Annotation} aus \autoref{lst:jerseymethode}.

\begin{description}
\item[@POST] Definiert die \gls{HTTP-Methode}. Hier POST. Einige weitere Möglichkeiten des HTTP-Protokolls sind GET, PUT und DELETE
\item[@Path('/query')] Definiert den URL-Pfad Suffix für diese Methode. Um diese Methode als Web Service via HTTP aufzurufen lautet die finale URL \enquote{http://localhost:8080/plib-characteristic-query/rest/query}. 
\item[@Consumes('application/xml')] Definiert den \gls{MIME-Type}\footnote{Internet Media Type oder auch Content-Type.}, welcher von diesem Service (diese Methode) konsumiert werden kann. Wird ein anderer Typ als POST an diesen Service geliefert, weist der Service diese Anfrage ab. 
\item[@Produces(MediaType.APPLICATION\_XML)] Definiert den \gls{MIME-Type} des Inhaltes, der vom Service als Antwort zurückgeliefert wird.  
\end{description}
\index{HTTP-Methode!GET}\index{HTTP-Methode!POST}\index{HTTP-Methode!PUT}\index{HTTP-Methode!DELETE}
\index{MIME-Type}

% 
\section{Query-Verarbeitung}

Der \gls{Web Service}, welcher in \autoref{kap:webservice} beschrieben ist, nimmt in der Applikation das Query-XML File entgegen. 
Die Struktur des XML-Files ist durch die query.xsd vorgegben \citep[27]{iso29002-31}. 

Der nächste Schritt ist es, diese XML zu Verarbeiten. Dazu muss das xml geparsed und die Informationen des Queries in ein entsprechendes Modell überführt werden. Diesen Prozess nennt man \gls{Unmarshalling}. 

Die folgenden Schritte müssen für das Unmarshalling durchgeführt werden.

\begin{enumerate}
\item Ein Modell in Java erstellen
\item XML parsen und in das Modell überführen
\item Validierung des Modells gemäß der Regeln in Schema-Datei
\end{enumerate}

%modell
\subsection{Modell-Generierung}

Ein entsprechend valides Model anhand der XSD manuell in Java aufzubauen wäre sehr mühsam und fehleranfällig. Java liefert mit der JAXB Bibliothek die Möglichkeit die Modellklassen von Java aus den XSDs zu generieren.

\subsubsection{Benötigte XSD-Dateien}

Die XSD-Dateien der ISO 29002-31 wurden freundlicherweise von Dr. Gerry Radack von der ECCMA zur Verfügung gestellt. 

Die query.xsd referenziert die weiteren folgenden Schema-Dateien:
\begin{itemize}
\item basic.xsd
\item identifier.xsd
\item catalogue.xsd
\item value.xsd
\end{itemize}

\subsubsection{Generierung mit JAXB}\index{Codegenerierung}

Die Generierung startet man als Java Kommando in der Konsole, siehe \autoref{lst:jaxbgeneratemodel}.

\begin{lstlisting}[caption=JAXB Modellgenerierung von der Konsole, language=sh, label=lst:jaxbgeneratemodel]
xjc query.xsd -d plib-characteristic-data/ 
\end{lstlisting}

\subsubsection{Problem Namensräume}
Bei der Generierung wurde die folgende Fehlermeldung angezeigt:

\enquote{[ERROR] The package name iso.std.iso.ts.\_29002.\_\_-31.ed\_1.tech.xml\_schema.query used for this schema is not a valid package name. line 18 of file query.xsd}
  
Der Grund für den Fehler ist, dass in der query.xsd die \glslink{Namespace}{Namensräume} wie in  \autoref{lst:queryschemanamespace} definiert sind. Da JAXB die \Glspl{Namespace} nutzt um die entsprechenden Paketstrukturen für die Modelle in Java zu erstellen, schafft es JAXB nicht, diese in entsprechende valide Form umzuwandeln. Man erkennt es daran, dass aus \enquote{urn:iso:std:iso:ts:29002:-31:ed-1:tech:xml-schema:query} in der Fehlermeldung \enquote{iso.std.iso.ts.\_29002.\_\_-31.ed\_1.tech.xml\_schema.query} wird\footnote{Java Paketnamen enthalten als Trenner zwischen den Ebenen einen Punkt. Valide wäre z.B. iso.std.iso.ts.query}. Der Bindestrich vor der 31 ist nicht als erstes Zeichen eines  Unterpaketnamens erlaubt. Wenngleich man hier erwarten würde, dass dies abgefangen und entsprechend der Umwandlungsregeln in valide Bezeichner konvertiert wird, so wäre dennoch eine Benamung der Pakete in der Form unübersichtlich.  \index{Namespace}

\code{public void doThat()}

\begin{lstlisting}[caption=query.xsd Namespace Definitionen, language=XML, label=lst:queryschemanamespace]
<xs:schema xmlns:xs="http://www.w3.org/2001/XMLSchema"
           xmlns:qy="urn:iso:std:iso:ts:29002:-31:ed-1:tech:xml-schema:query"
           xmlns:cat="urn:iso:std:iso:ts:29002:-10:ed-1:tech:xml-schema:catalogue"
           xmlns:val="urn:iso:std:iso:ts:29002:-10:ed-1:tech:xml-schema:value"
           xmlns:bas="urn:iso:std:iso:ts:29002:-4:ed-1:tech:xml-schema:basic"
           xmlns:id="urn:iso:std:iso:ts:29002:-5:ed-1:tech:xml-schema:identifier"
           targetNamespace="urn:iso:std:iso:ts:29002:-31:ed-1:tech:xml-schema:query" elementFormDefault="qualified">
    <xs:import namespace="urn:iso:std:iso:ts:29002:-4:ed-1:tech:xml-schema:basic" schemaLocation="basic.xsd"/>
    <xs:import namespace="urn:iso:std:iso:ts:29002:-5:ed-1:tech:xml-schema:identifier" schemaLocation="identifier.xsd"/>
    <xs:import namespace="urn:iso:std:iso:ts:29002:-10:ed-1:tech:xml-schema:catalogue" schemaLocation="catalogue.xsd"/>
    <xs:import namespace="urn:iso:std:iso:ts:29002:-10:ed-1:tech:xml-schema:value" schemaLocation="value.xsd"/>
    ...
</xs:schema>    
\end{lstlisting}

Für dieses Problem gibt es zwei Lösungsoptionen. Entweder die Namespace-Definitionen aller Namespaces in den XSD-Dateien anpassen oder die XSD-Dateien in der Form belassen und einen programmatischen Weg finden um die Namespaces umzudefinieren. 

Das Anpassen aller XSD-Dateien hat zwei große Nachteile.
\begin{enumerate}
\item Es ist aufwändig und fehleranfällig alle Dateien anzupassen. Diese haben wie in \autoref{lst:queryschemanamespace} zu sehen, Abhängigkeiten untereinander.
\item Änderungen an der lokalen Schemadatei machen mögliche spätere Integrationen einer neuen XSD-Version des ISO-Komitees schwierig.
\end{enumerate}

Besser wäre es, wenn die Generierung konfigurierbar ist. \gls{JAXB} ermöglicht mit einem sogenanntem Binding-File, die Namespaces abzuändern. \autoref{lst:bindingfile} zeigt die Konfiguration. Die Generierung kann mit dem Shell-Befehl \enquote{xjc -b binding.xjb -d gen-src query.xsd} gestartet werden. \index{Binding} \index{JAXB}

\begin{lstlisting}[caption=Binding File binding.xjc, language=XML, label=lst:bindingfile]
<?xml version="1.0" encoding="UTF-8"?>
<jaxb:bindings xmlns:jaxb="http://java.sun.com/xml/ns/jaxb"
               xmlns:xsd="http://www.w3.org/2001/XMLSchema"
               xmlns:xjc="http://java.sun.com/xml/ns/jaxb/xjc"
               jaxb:version="2.0">
  <jaxb:bindings schemaLocation="query.xsd" node="/xsd:schema">
    <jaxb:schemaBindings>
      <jaxb:package name="de.feu.plib.xml.query" />
    </jaxb:schemaBindings>
  </jaxb:bindings>
  <jaxb:bindings schemaLocation="basic.xsd" node="/xsd:schema">
    <jaxb:schemaBindings>
      <jaxb:package name="de.feu.plib.xml.basic" />
    </jaxb:schemaBindings>
  </jaxb:bindings>
  <jaxb:bindings schemaLocation="catalogue.xsd" node="/xsd:schema">
    <jaxb:schemaBindings>
      <jaxb:package name="de.feu.plib.xml.catalogue" />
    </jaxb:schemaBindings>
  </jaxb:bindings>
  <jaxb:bindings schemaLocation="identifier.xsd" node="/xsd:schema">
    <jaxb:schemaBindings>
      <jaxb:package name="de.feu.plib.xml.identifier" />
    </jaxb:schemaBindings>
  </jaxb:bindings>
  <jaxb:bindings schemaLocation="value.xsd" node="/xsd:schema">
    <jaxb:schemaBindings>
      <jaxb:package name="de.feu.plib.xml.value" />
    </jaxb:schemaBindings>
  </jaxb:bindings> 
  
      <jaxb:globalBindings>
         <!-- let the classes implement serialiseable -->
        <jaxb:serializable uid="1" />
          <!-- let the classes extend own abstract class for providing some extra functionality for each one -->
     </jaxb:globalBindings>  
</jaxb:bindings> 
\end{lstlisting}

Wie man erkennen kann, wird für jede Schema Datei ein eigener Paketname definiert. Das hat den Vorteil, dass die einzelnen Datentypen aus den jeweiligen XSD-Dateien passend in eigene Pakete generiert werden und nicht alle in ein Verzeichnis. Das ist deutlich übersichtlicher. 

Die Modell-Dateien werden folgendermaßen abelegt:

\begin{description}
\item[query.xsd] de.feu.plib.xml.query
\item[basic.xsd] de.feu.plib.xml.basic
\item[catalogue.xsd] de.feu.plib.xml.catalogue
\item[identifier.xsd] de.feu.plib.xml.identifier
\item[value.xsd] de.feu.plib.xml.value
\end{description}

Zeile 6-10 aus \autoref{lst:bindingfile} zeigt wie im Binding-File ein anderer Paketname für die Datei query.xsd definiert werden kann. 

\subsection{Einbinden in Buildprozess mit Maven}\index{Maven}\index{Buildprozess}
Da als Build-Werkzeug \gls{Maven} verwendet wird, kann der gesamte Generierungsprozess darüber abgebildet werden. 

\subsubsection{Separates Source-Verzeichnis}
Die Standardprojektform eines Maven-Projektes hat einen sogenannten Source-Folder (src). Darin befindet sich ein main-Folder. Dort werden die Klassen der Applikation abgelegt. Ferner beinhaltet der src-Folder einen test-Folder. Darin werden Testklassen abgelegt. 
Damit die generierten Sourcen klar getrennt sind von den anderen Source-Dateien, soll ein separates Source-Verzeichnis angelegt werden. Der weiterer Vorteil ist, dass dieser Folder beispielsweise auch vom Versionskontrollsystem ausgenommen werden kann, da diese Klassen während des Build-Prozesses jeweils generiert werden und somit keiner Versionierung bedürfen. Das wäre aufwändiger zu realisieren, wenn die Dateien exakt im gleichen Source-Folder generiert würden wo alle anderen Source-Dateien liegen\footnote{Hinweis: Moderne Versionskontrollsysteme wie das eingesetzte GIT ermöglicht es nach Mustern Dateien oder Ordner von der Versionierung auszunehmen. Allerdings müsste das hier auf Paketbasis erfolgen, sozusagen ab de.feu.plib.xml.*, das macht die Umgebung im Ganzen komplexer}. 

Maven bietet mittels des Plugins \enquote{build-helper-maven-plugin} die Möglichkeit, dies zu erzeugen. 
Die generierten Sourcen der XSD-Dateien werden in ein separates Verzeichnis namens \enquote{generated} erzeugt (siehe \autoref{lst:buildhelperplugin}). 

\begin{lstlisting}[caption=Build Helper Maven Plugin, language=, label=lst:buildhelperplugin]
<plugin>
    <groupId>org.codehaus.mojo</groupId>
    <artifactId>build-helper-maven-plugin</artifactId>
    <executions>
        <execution>
            <id>add-source</id>
            <phase>generate-sources</phase>
            <goals>
                <goal>add-source</goal>
            </goals>
            <configuration>
                <sources>
                    <source>src/main/generated</source>
                </sources>
            </configuration>
        </execution>
    </executions>
</plugin>
\end{lstlisting}

\subsubsection{JAXB Maven Plugin}

Um JAXB mit \gls{Maven} zu nutzen, muss das \gls{Maven}-Plugin \enquote{maven-jaxb2-plugin} in die \gls{pom} eingetragen werden. \autoref{lst:jaxbplugin} zeigt den XML-Auschnitt aus der \gls{pom}. \index{Maven!pom.xml}

\begin{lstlisting}[caption=JAXB Maven Plugin, language=XML, label=lst:jaxbplugin]
            <plugin>
                <groupId>org.jvnet.jaxb2.maven2</groupId>
                <artifactId>maven-jaxb2-plugin</artifactId>
                <version>0.8.0</version>
                <configuration>
                    <schemaDirectory>src/main/resources/schema</schemaDirectory>
                    <generateDirectory>src/main/generated</generateDirectory>
                    <removeOldOutput>true</removeOldOutput>
<!-- we do not use bindingDirectory as if we put the binding.xjb in the schema directory it will be taken -->
<!--                     <bindingDirectory>src/main/resources/binding</bindingDirectory> -->

<!--  Setting the generated package in pom will override what you set in binding.xjb file, thus commented out -->
<!--                     <generatePackage>de.feu.plib.jaxb</generatePackage> -->
                    <strict>false</strict>
                    <extension>true</extension>
                    <plugins>
                        <plugin>
                            <groupId>org.jvnet.jaxb2_commons</groupId>
                            <artifactId>jaxb2-basics</artifactId>
                            <version>0.6.2</version>
                        </plugin>
                        <plugin>
                            <groupId>org.jvnet.jaxb2_commons</groupId>
                            <artifactId>jaxb2-basics-annotate</artifactId>
                            <version>0.6.2</version>
                        </plugin>
                    </plugins>
                    <args>
                        <arg>-Xannotate</arg>
                        <arg>-XtoString</arg>
                    </args>
                </configuration>
                <executions>
                    <execution>
                        <id>generate</id>
                        <goals>
                            <goal>generate</goal>
                        </goals>
                    </execution>
                </executions>
            </plugin>    
\end{lstlisting}

\subsection{Abfrage der PLIB Prozeduren}

Die Anforderung ist es, so weit wie möglich die vorhandenen Prozeduren aus der Arbeit von Herrn Mende zu nutzen. 

Die Prozeduren von Herrn Mende nehmen einen sogenannten Externen Identifier entgegen. Dieser identifiziert eindeutig eine Instanz eines Teils. 
Beispiel: Ich habe das Konzept \enquote{Sechskantschraube} und in der Teiledatenbank gibt es davon genau eine gespeicherte Instanz, also eine Schraube mit definierten Werteeigenschaften. 

Zur Abfrage stehen folgende Oracle-Prozeduren zur Verfügung:

\begin{description}
\item[GET\_PROP\_VALS\_STRING] Nimmt als IN-Parameter eine Externe Produkt-ID entgegen und liefert eine Tabelle vom Typ PROP\_STRING\_NTT zurück. 
Diese Tabelle beinhaltet die folgenden Werte: 
  \begin{description}
  \item[IRDI] xxx
  \item[VALUE] xxx
  \item[UNIT] xxx
  \item[PREFIX] xxx
  \item[TOLERANCE] xxx
  \item[VALUE\_ID] xxx
  \end{description}

\todotext{Attribute beschreiben}  

\item[GET\_PROP\_VALS\_NUMBER]  xxx
\item[GET\_PROP\_VALS\_REFERENCES]  xxx
\item[GET\_PROP\_VALS\_LIST\_NUMBER] xxx
\item[GET\_PROP\_VALS\_LIST\_STRING] xxx
\item[GET\_PROP\_VALS\_MULTILIST\_NUMBER] xxx
\item[GET\_PROP\_VALS\_MULTILIST\_STRING]  xxx
\item[GET\_PROP\_VALS] xxx  
\end{description}

\todotext{Attribute beschreiben} 

\subsubsection{Problem - Externer Identifier}

Hier stellt sich das Problem, dass eine query.xsd generell immer auf die IRDI eines Konzeptes basiert. 
Beispiel: \enquote{Gib mir bitte alle Instanzen und Werte der Eigenschaften des Klassenkonzeptes mit dem Identifier 0173-1\#01-BAD803\#2 (Skalpell)}
Der Query des Standards fragt folglich explizit nach den Instanzen eines Konzeptes (hier Skalpell). Die Prozeduren benötigen allerdings einen nicht im Standard definierten Identifier einer konkreten Instanz, um alle Eigenschaftswerte zu ermitteln. 

\subsubsection{Lösung}

Zwei Lösungsoptionen stellen sich hier zur Auswahl:
\begin{enumerate}
\item Anpassen der Datenbank-Prozeduren, so dass anstatt eines externen Identifiers eine IRDI entgegengenommen werden kann.
\item Separate Abfrage an die Datenbank in der Applikation realisieren. Diese Abfrage ermittelt anhand der vom Query übergebenen IRDI alle externen Identifier der Instanzen dieses Konzeptes. Anschließend können die Prozeduren aufgerufen werden.   
\end{enumerate}

Es wurde die Lösung Nummer zwei gewählt. Hierfür gibt es mehrere Gründe. Der Hauptgrund ist, dass zum Zeitpunkt der Erstellung dieser Arbeit, die Abschlussarbeit von Herrn Mende noch nicht abgeschlossen ist. Es ist damit nicht auszuschließen, dass sich Prozeduren oder Logik in den Prozeduren zu einem späteren Zeitpunkt noch ändern. Das ist im Prinzip kein Problem, da diese Arbeit sich auf einen bestimmten Entwicklungsstand der Arbeit von Herrn Mende bezieht, allerdings wird in der Zukunft versucht, die neuere Version von Herrn Mende in diese Arbeit zu integrieren, wird es sehr schwierig und aufwändig dies zu bewältigen, wenn im Rahmen dieser Arbeit die Prozeduren geändert werden. Man stände vor dem Problem, dass beide Seiten die Basis modifiziert haben. 
Ein weiterer Grund für die separate Abfrage ist, dass dies in der Ebene der Applikationslogik für einen Nicht-Oracle Experten einfacher zu implementieren, und damit weniger fehleranfällig ist. Die Prozeduren sind bereits sehr komplex und im Detail nicht einfach zu verstehen. 
Ferner stellt diese Abfrage eine Teilfunktionalität dar und ist somit prima separat testbar.  

Das \autoref{lst:getexternalids} zeigt den SQL-Query der Lösung des Problems.

\begin{lstlisting}[caption=JAXB Maven Plugin, language=SQL, label=lst:getexternalids]
SELECT o.DI_ID FROM DE_CLASS c, DO_OBJECT o WHERE c.ID = o.C_ID AND c.IRDI = ?
\end{lstlisting}

Zur Erklärung des \autoref{lst:getexternalids}:
DI\_ID ist der Attributsname des oben genannten externen Identifiers einer Instanz. Das Prädikat IRDI = ? nimmt die IRDI des eigentlichen Queries gemäß query.xsd entgegen. Das Ergebnis ist eine Liste der externen Identifier, welche dann je an die Prozedur übergeben werden können. 

Um zu verstehen, wie diese Abfrage in der Applikation im Kontext verwendet wird, sei auf das Sequenzdiagramm in \autoref{fig:sequenzdiagrammsimplequery} verwiesen. 

\begin{figure}[htbp]
	\centering
		\includegraphics[width=0.99\textwidth]{images/plib_simple_query_sequence_diagram.png}
		\caption{Sequenzdiagramm Simple Query}
	\label{fig:sequenzdiagrammsimplequery}
\end{figure}

\section{Zusammenfassung des Kapitels}




%%\setchapterpreamble[u]{
%\dictum[Johann Wolfgang von Goethe]{Es ist nicht genug, zu wissen, man muß auch anwenden; es ist nicht genug, zu wollen, man muß auch tun. \dots}}
\section{Anwendungsfälle}\label{kap:Use_Cases}
% Funktionale Anforderungen

Dieses Kapitel beschreibt mögliche Anwendungsfälle\footnote{Use Case. Im weiteren Verlauf wird, gerade auch in den Anwendungsfall Beschreibungen von Use Cases gesprochen. Im deutschen Sprachraum ist das englische allgemein bekannte Äquivalent \enquote{Use Case} für Anwendungsfall eher geläufig als das deutsche, daher wird darauf verzichtet an manchen Stellen die Begrifflichkeiten im weiteren Verlauf zwangsweise einzudeutschen.} die sich aus ISO 29002-31 ergeben. 


Die Query-Code-Beispiele sind gekürzt, d.h. es werden die referenzierte Schemata-Namen gemäß XSD nicht aufgeführt. 

\begin{figure}[htbp]
	\centering
		\includegraphics[width=0.75\textwidth]{images/usecases_plib.png}
	\caption{Use Case Übersicht}
	\label{fig:usecaseuebersicht}
\end{figure}

\subsection{Akteure}
Bei der Implementierung geht es um die Erstellung einer Software-Schnittstelle als Web Service. Das bedeutet, dass diese Schnittstelle ohne eine Benutzeroberfläche nicht sinnvoll für einen menschlichen Akteur nutzbar ist. In den nachfolgenden Anwendungsfällen wird vom Akteur \enquote{Klient} gesprochen. Der Klient ist allgemein ein Nutzer der Schnittstelle, sei es als menschlicher Akteur über welcher über eine Bedienerinterface die Schnittstelle benutzt oder eine direkte Maschinennutzung. Ein Beispiel für eine Maschinennutzung kann einen Anwendung sein, welche automatisiert die Schnittstelle aufruft und Daten abfragt.    

%TODO Use Case Diagram

\subsection{Use Case Beschreibungen}

\subsubsection{Alle Charakteristische Daten eines Produkts abfragen}

{\small

\begin{description}
     \item[use case] Charakteristische Daten abfragen
     \item[  actors]~\\
     Klient
     \item[  precondition]~\\
     Der Klient verwendet einen gültigen Identifier.
     \item[  main flow]~\\
     Der Klient gibt einen Identifier (IRDI\footnote{International Registration Data Identifier}) einer Klasse von Elementen ein und sendet eine Anfrage ab. Die Anfrage wird auf Gültigkeit überprüft. Als Antwort bekommt er ein oder mehrere Datensätze von Elementen \footnote{Item, ISO 29002-10 Kapitel 5.3.2} mit den entsprechenden charakteristischen Daten \footnote{property\_values, ISO 29002-10 Kapitel 5.2.4}  des Elementes mit dem übergebenen Identifier zurück.
     \item[  postcondition]~\\
     Alle Daten aller Elemente der gewählten Klassen des Identifiers wurden zurückgegeben.    
     \item[  alternative flow] Properties auswählen ~\\
     Zusammen mit dem Identifier übergibt der Klient einen oder mehrere Property-Identifier und sendet diese erweiterte Anfrage ab.    
     \item[  postcondition]~\\
     Die mittels Property-Identifier ausgewählten Daten aller Elemente der gewählten Klassen wurden zurückgegeben.    
     \item[end] Charakteristische Daten abfragen
\end{description}

~\\

} %end small

\paragraph{Beispiel}

Ein Schraubendreher könnte folgendermaßen in einer Produktdatenbank repräsentiert werden:

\begin{description}\label{lab:schraubendreher}
\item[Klassen-Identifier] 0173-1\#01-AAA352\#4 
\item[Länge] 300mm
\item[Typ] Kreuz
\item[Spannungsfest] ja
\end{description}

Korrekterweise müssten anstatt der Attribute wie Länge oder Typ ebenfalls ein Identifier stehen. Die Benamungen sind hier zur besseren Lesbarkeit aufgelöst. 

Um nun alle Eigenschaften (Properties), wie Länge, Typ und Spannungsfest zu erhalten muss folgende Abfrage gesendet werden: 
\textbf{"Gib mir alle Items und alle Properties der Klasse mit dem Identifier 0173-1\#01-AAA352\#4 (Schraubendreher)".}
Das Ergebnis ist ein Item mit allen Attributen (Properties) der gewünschten Klassen und gegebenenfalls vorhandenen Unterklassen. In unserem Falle genau die oben angegebenen Werte.

Die XML-Abfrage sieht wie folgt aus:

\begin{lstlisting}[caption=Query Beispiel - Daten abfragen, language=XML, label=UseCaseDatenabfragen]
<?xml version="1.0" encoding="UTF-8"?>
<qy:query xsi:schemaLocation="...query query.xsd" xmlns:xsi="http://www.w3.org/2001/XMLSchema-instance" xmlns:cat="...catalogue" xmlns:val="...value" xmlns:qy="...query" xmlns:bas="...basic">
	<qy:class_ref>0173-1#01-AAA352#4</qy:class_ref>
</qy:query>
\end{lstlisting}

Eine Abfrage, welche die Properties der Klasse auswählt die zurückgeliefert werden sollen könnte lauten: 
\textbf{"Gib mir alle Items und die Properties Länge und Typ der Klasse mit dem Identifier 0173-1\#01-AAA352\#4 (Schraubendreher)".}
Das Ergebnis ist ein Item mit den gewünschten Attributen (Properties). 

Die XML-Abfrage:
\begin{lstlisting}[caption=Query Beispiel - Daten abfragen mit Propertyeinschränkung, language=XML, label=lst:UseCaseDatenabfragenProperty]

<?xml version="1.0" encoding="UTF-8"?>
<qy:query xsi:schemaLocation="...query query.xsd" xmlns:xsi="http://www.w3.org/2001/XMLSchema-instance" xmlns:cat="...catalogue" xmlns:val="...value" xmlns:qy="...query" xmlns:bas="...basic">
	<qy:class_ref>0173-1#01-AAA352#4</qy:class_ref>
	
	<!-- typ und laenge -->
	<qy:property_ref>0173-1#01-BBB111#1 0173-1#01-BBB222#1</qy:property_ref> 
	
</qy:query>
\end{lstlisting}

Listing \ref{lst:UseCaseDatenabfragenProperty} beinhaltet ein XML-Attribut property\_ref. Das wird mit gewünschten Property Identifier gefüllt, welche mit Leerzeichen getrennt werden. 

\subsubsection{Charakteristische Daten eines Produkts validieren}

{\small

\begin{description}
     \item[use case] Charakteristische Daten validieren
     \item[  actors]~\\
     Klient
     \item[  precondition]~\\
     Der Klient verwendet einen gültigen Identifier sowie auf den Identifier passende Daten..
     \item[  main flow]~\\
     Der Klient gibt einen Identifier eines Elementes (Klasse) ein. Zusätzlich übermittelt er zu diesem bekanntem Element Eigenschaften dieser Instanz des Elements und sendet eine Anfrage ab. Die Anfrage wird auf Gültigkeit überprüft. Als Antwort bekommt er ein oder mehrere Datensätze von Elementen mit den entsprechenden charakteristischen Daten zurück, auf welche die übergebenen Eigenschaften am besten zutreffen. 
     \item[  postcondition]~\\
     Alle Daten aller Elemente der gewählten Klassen des Identifiers werden zurückgegeben. Dies ermöglicht dem Klienten eine Validierung der ihm bereits bekannten Daten über ein Element. 
     \item[end] Charakteristische Daten validieren
\end{description}

~\\

} %end small

\paragraph{Beispiel}

In diesem Anwendungsfall verfügen wir bereits über Elemente/Wertenpaare einer bestimmten Klasse, z.B. eben jenen Schraubendreher

\textbf{"Ich habe hier ein mir bekanntes Item mit bestimmten Eigenschaften (Properties), Länge=300mm. Gib mir alle Items und alle Properties der Klasse mit dem Identifier 0173-1\#01-AAA352\#4 (Kreuzschraube) welche die mitgelieferten Eigenschaften haben".}
Das Ergebnis sind Items mit allen Properties der angegebenen Klasse, welche über die übergebenen Eigenschaften (Properties) verfügen. In unserem Fall vervollständigen wir unsere Properties mit den weiteren Properties "Typ" und "Spannungsfest".

Die XML-Abfrage gemäß query.xsd\footnote{Schema Datei ist referenziert in ISO 29002-31, liegt der Arbeit bei} sieht so aus:

\begin{lstlisting}[caption=Query Beispiel - Daten abfragen, language=XML, label=UseCaseDatenabfragen]
<?xml version="1.0" encoding="UTF-8"?>
<qy:query xsi:schemaLocation="...query query.xsd" xmlns:xsi="http://www.w3.org/2001/XMLSchema-instance" xmlns:cat="...catalogue" xmlns:val="...value" xmlns:qy="...query" xmlns:bas="...basic">
	<cat:item class_ref="0173-1#01-AAA352#4..">
		<cat:property_value property_ref="0173-1#01-BBB111#1">
			<val:integer_value></val:integer_value>
		</cat:property_value>
	</cat:item>
</qy:query>
\end{lstlisting}

\subsubsection{Chrarakteristische Daten mittels Suchausdruck abfragen }

{\small

\begin{description}
     \item[use case] Charakteristische Daten mit Suchausdruck abfragen
     \item[  actors]~\\
     Klient
     \item[  precondition]~\\
     Der Klient verwendet einen gültigen Identifier.
     \item[  main flow]~\\
     Der Klient gibt einen Identifier eines Elementes (Klasse) ein. Ferner übergibt er ein oder mehrere bekannte Property Identifier sowie passend dazu Werte zur Sucheinschränkung. 
     \item[  postcondition]~\\
     Alle Elemente auf jene diese Einschränkung der übergebenen Werte zutrifft wurden zurückgegeben. 
     \item[end] Charakteristische Daten mit Suchausdruck abfragen
\end{description}

~\\

} %end small

\paragraph{Beispiel}

Wir nehmen das Schraubendreher Beispiel aus \ref{lab:schraubendreher} zur Hand, und möchten eine Abfrage absenden, welche von der Klasse Schraubendreher alle Items erhalten soll die eine Länge zwischen 200 und 300 mm haben. 

Um nun alle Eigenschaften (Properties), wie Länge, Typ und Spannungsfest zu erhalten muss folgende Abfrage gesendet werden: 
\textbf{"Gib mir alle Items und alle Properties der Klasse mit dem Identifier 0173-1\#01-AAA352\#4 (Kreuzschraube)".}
Das Ergebnis ist ein Item mit allen Attributen (Properties) der gewünschten Klassen und gegebenenfalls vorhandenen Unterklassen. In unserem Falle genau die oben angegebenen Werte.

Die XML-Abfrage gemäß query.xsd\footnote{Schema Datei ist referenziert in ISO 29002-31, liegt der Arbeit bei} sieht so aus:

\begin{lstlisting}[caption=Query Beispiel - Daten abfragen, language=XML, label=UseCaseDatenabfragen]
<?xml version="1.0" encoding="UTF-8"?>
<qy:query xsi:schemaLocation="...query query.xsd" xmlns:xsi="http://www.w3.org/2001/XMLSchema-instance" xmlns:cat="...catalogue" xmlns:val="...value" xmlns:qy="...query" xmlns:bas="...basic">
	<qy:class_ref>0173-1#01-AAA352#4</qy:class_ref>
	<qy:characteristic_data_query_expression>
		<qy:range>
			<qy:property_reference property_ref="0173-1#01-BBB111#1"/>
			<qy:min_value>200</qy:min_value>
			<qy:max_value>300</qy:max_value>
			<qy:is_inclusive>true</qy:is_inclusive>
		</qy:range>
	</qy:characteristic_data_query_expression>
</qy:query>
\end{lstlisting}


%\section{Automatisierte Benutzerebene}
%Der Unterschied zur manuellen Benutzerebene ist der, dass hierbei automatisiert Daten angefragt und übermittelt werden. Es findet keine Mensch zu %Maschine Kommunikation statt sondern eine Maschine zu Maschine Kommunikation. 
%Ziel der automatisierten Anfragen ist das Abgleichen oder Validieren von Massendaten eines (Teil)-Katalogs. 

%\begin{description}
%\item[Alle Klassen abfragen] Der Klient sendet eine Anfrage und erhält alle vorhandene Klassen (ohne Items).
%\item[Items einer Klasse abgleichen] Der Klient möchte seine Daten abgleichen und fragt alle Items einer Klasse ab.  
%\item[Items einer Klasse validieren] Der Klient möchte seine Daten validieren und fragt alle Items einer Klasse ab.
%\end{description}
\chapter*{Schlussfolgerung und Ausblick in die Zukunft}

\addcontentsline{toc}{chapter}{Schlussfolgerung}

Die Zielsetzung dieser Arbeit, die Machbarkeit und Integration der ISO-Schnittstelle aufbauend auf die vorhandene PLIB-Datenbankimplementierung aufzuzeigen, wurde erreicht. Aktuelle Techniken und Frameworks ermöglichen eine relativ schnelle Integration der den Standards zugehörigen Schemata. Es lässt sich mit überschaubarem Aufwand ein Web Service erstellen. 
Es wurde aufgezeigt, dass im Rahmen der Abschlussarbeiten des Fachbereiches die Abfrageschnittstelle aufbauend auf die PLIB-Datenbank des Fachbereiches möglich ist. Während der Analyse und Implementierung ergaben sich einige Problemstellungen, die gelöst werden konnten. Beispielsweise ist die Heterogenität der Datenstrukturen ein nicht triviales Problem. Die Prozeduren konnten nicht direkt mit den von der Abfrage erhaltenen Daten aufgerufen werden. Es sind vorab Anfragen zur Transformation an die Datenbank nötig. 
Die zurückgelieferten Daten der Prozeduren liefern nicht alle Daten zurück die für das Füllen der XML-Katalog Daten nötig sind. Ferner passt die Struktur und Semantik der zurückgelieferten Daten nicht mit den Antwort des Standards überein. 
Dies konnte durch entsprechende Absprache unter den Studenten des Fachbereiches gelöst werden konnte. An den Prozeduren wurden keine umfangreichenden Änderungen durchgeführt, jedoch ermöglichten die Absprachen ein Verständnis, welches einen Lösungsweg ermöglichten. 

Die Umsetzung der Arbeit erfolgte mit Hilfe aktueller Technologien auf Basis von REST Web Services. Die Applikation kann automatisiert mit Hilfe eines Build Management Systems einfach erzeugt werden. Das ermöglicht es, dass andere Studenten das System ohne Aufwand erzeugen können. Die Quelltexte sind in einem Versionskontrollsystem abgelegt und ermöglichen anderen Studenten und Entwicklern den Zugriff darauf. 

Die Arbeit umfasst nicht die Implementierung einer GUI und ebenfalls nicht die Implementierung der Schnittstelle nach ISO 22745-30 Identification Guide. Dieser Standard beschreibt eine Möglichkeit, Regeln zu definieren, welche Konzepte (z.B aus PLIB) näher beschreiben. Das ermöglicht sinnvollerweise eine Einschränkung der Anfragedaten in der Form, wie es der Kunde respektive Klient in seinem Kontext benötigt. Man stelle sich vor, dass in einer Produktionsabteilung andere Daten von bestimmten Konzepten nötig sind als in der Verkaufsabteilung. Ein Beispiel wären bestimmte Eigenschaften des Materials über die Verformung bei großer Hitze in der Produktion. Bei einem Metall ist diese Information in der Produktion wichtig, da hier ggfs. hohe Hitzeentwicklung vorzufinden ist. Der Verkauf benötigt diese Informationen nicht oder nur gewisse Teile davon. Dies kann mit Hilfe von Identification Guides eingeschränkt werden. 
Als Folgearbeit wäre beispielsweise denkbar, die Implementierung von Identification Guides vorzunehmen. Um diese Einschränkung nach Regeln gemäß ISO 22745-30 zu demonstrieren könnte eine Benutzeroberfläche aus den Angaben generiert werden. Nehmen wir aus obigem Beispiel eine Applikation welche eine Benutzeroberfläche für die Abteilung Produktion und eine welche die Abteilung Verkauf repräsentiert. Diese könnten nach ISO 22745-30 unterschiedliche definierte Regelwerke besitzen und folglich eine andere Maske darstellen. Die Abfrage die somit von der Abteilung Produktion über die Maske gesendet werden kann, kann nur gewisse Eigenschaften eines Produktes abfragen die für die Abteilung relevant sind. Die Abfrage der Abteilung Verkauf kann folglich andere Eigenschaften abfragen die nur für diese Abteilung relevant sind.

Als mögliche weitere Arbeit wäre eine Gesamtintegration aller Arbeiten im Fachbereich denkbar. Diese könnte die Arbeiten von Herrn Mende, Herrn Loth, Frau Janßen und Herrn Sobek integrieren.  

\todotext{Folgenarbeiten beschreiben,  22745-30 zu implementieren, näher beschreiben warum und was da getan werden könnte, z.B. generierung der GUI aus dem Identification Guide Angaben der Produkte. Sinn: Vorauswahl beim Kunden, GUI wird generiert, dann Erzeugung query.xml nach 29002-31, wie in dieser Arbeit usw. Möglicherweise Gesamtintegration eine weitere Arbeit (Hr. Mende, Hr. Loth, Frau Janßen, Hr. Sobek; so könnte ein Praxisbeispiel komplett gezeigt werden, möglicherweise anhand eines simplen konkretem Beispiels.}
\chapter*{Eidesstattliche Erklärung}

Hiermit versichere ich, dass ich die vorliegende Arbeit selbstständig verfasst und keine anderen als die angegebenen Quellen und Hilfsmittel benutzt habe, dass alle Stellen der Arbeit, die wörtlich oder sinngemäß aus anderen Quellen übernommen wurden, als solche kenntlich gemacht und dass die Arbeit in gleicher oder ähnlicher Form noch keiner Prüfungsbehörde vorgelegt wurde.

\vspace{3cm}
Ort, Datum \hspace{5cm} Unterschrift (Stefan Sobek)\\

%%%%%%%%%%%%%%%%%%%%%%%%%%%%%%%%%%%%%%%%%%%%%%%%%%%%%%%%%%%%%
%% LITERATUR UND ANDERE VERZEICHNISSE
%%%%%%%%%%%%%%%%%%%%%%%%%%%%%%%%%%%%%%%%%%%%%%%%%%%%%%%%%%%%%
%% Ein kleiner Abstand zu den Kapiteln im Inhaltsverzeichnis (toc)
\addtocontents{toc}{\protect\vspace*{\baselineskip}}

%% Literaturverzeichnis

\addcontentsline{toc}{chapter}{Literaturverzeichnis}
\nocite{*} %Auch nicht-zitierte BibTeX-Eintr‰ge werden angezeigt.
\bibliographystyle{alphadin}
%\bibliographystyle{plain}
\bibliography{literatur}


%%%%%%%%%%%%%%%%%%%%%%%%%%%%%%%%%%%%%%%%%%%%%%%%%%%%%%%%%%%%%
%% ANHÄNGEvi _
%%%%%%%%%%%%%%%%%%%%%%%%%%%%%%%%%%%%%%%%%%%%%%%%%%%%%%%%%%%%%
\appendix
\addcontentsline{toc}{chapter}{Anhang}


\chapter{Schema Dateien} \index{Schema-Dateien}\label{kap:anhang_schema}

\section{query.xsd}\index{Schema!query.xsd}\label{sec:query_xsd}

 \begin{lstlisting}[caption=query.xsd, language=XML, label=lst:query_xsd]
<?xml version="1.0" encoding="UTF-8"?>
<!--
$Id: query.xsd 411 2009-07-21 01:57:47Z radack $
ISO TC 184/SC 4/WG 12 N6664 - ISO/TS 29002-31 Query - XML schema
-->
<!--
The following permission notice and disclaimer shall be included in all copies of this XML schema ("the Schema"), and derivations of the Schema:

Permission is hereby granted, free of charge in perpetuity, to any person obtaining a copy of the Schema, to use, copy, modify, merge and distribute free of charge, copies of the Schema for the purposes of developing, implementing, installing and using software based on the  Schema, and to permit persons to whom the Schema is furnished to do so, subject to the following conditions:

THE SCHEMA IS PROVIDED "AS IS", WITHOUT WARRANTY OF ANY KIND, EXPRESS OR IMPLIED, INCLUDING BUT NOT LIMITED TO THE WARRANTIES OF MERCHANTABILITY, FITNESS FOR A PARTICULAR PURPOSE AND NONINFRINGEMENT. IN NO EVENT SHALL THE AUTHORS OR COPYRIGHT HOLDERS BE LIABLE FOR ANY CLAIM, DAMAGES OR OTHER LIABILITY, WHETHER IN AN ACTION OF CONTRACT, TORT OR OTHERWISE, ARISING FROM, OUT OF OR IN CONNECTION WITH THE SCHEMA OR THE USE OR OTHER DEALINGS IN THE SCHEMA.

In addition, any modified copy of the Schema shall include the following notice:

THIS SCHEMA HAS BEEN MODIFIED FROM THE SCHEMA DEFINED IN ISO 29002-31, AND SHOULD NOT BE INTERPRETED AS COMPLYING WITH THAT STANDARD.

-->
<xs:schema xmlns:xs="http://www.w3.org/2001/XMLSchema"
           xmlns:qy="urn:iso:std:iso:ts:29002:-31:ed-1:tech:xml-schema:query"
           xmlns:cat="urn:iso:std:iso:ts:29002:-10:ed-1:tech:xml-schema:catalogue"
           xmlns:val="urn:iso:std:iso:ts:29002:-10:ed-1:tech:xml-schema:value"
           xmlns:bas="urn:iso:std:iso:ts:29002:-4:ed-1:tech:xml-schema:basic"
           xmlns:id="urn:iso:std:iso:ts:29002:-5:ed-1:tech:xml-schema:identifier"
           targetNamespace="urn:iso:std:iso:ts:29002:-31:ed-1:tech:xml-schema:query" elementFormDefault="qualified">
    <xs:import namespace="urn:iso:std:iso:ts:29002:-4:ed-1:tech:xml-schema:basic" schemaLocation="basic.xsd"/>
    <xs:import namespace="urn:iso:std:iso:ts:29002:-5:ed-1:tech:xml-schema:identifier" schemaLocation="identifier.xsd"/>
    <xs:import namespace="urn:iso:std:iso:ts:29002:-10:ed-1:tech:xml-schema:catalogue" schemaLocation="catalogue.xsd"/>
    <xs:import namespace="urn:iso:std:iso:ts:29002:-10:ed-1:tech:xml-schema:value" schemaLocation="value.xsd"/>
    <!-- Global Elements -->
    <xs:element name="characteristic_data_query_expression" type="qy:characteristic_data_query_expression_Type"/>
    <xs:element name="query" type="qy:query_Type"/>
    <xs:element name="query_context" type="qy:query_context_Type"/>
    <!-- Global Types -->
    <!-- Characteristic Data Query Expression -->
    <xs:complexType name="characteristic_data_query_expression_Type">
        <xs:choice>
            <xs:element name="cardinality" type="qy:cardinality_expression_Type"/>
            <xs:element name="range" type="qy:range_expression_Type"/>
            <xs:element name="string_pattern" type="qy:string_pattern_expression_Type"/>
            <xs:element name="string_size" type="qy:string_size_expression_Type"/>
            <xs:element name="subset" type="qy:subset_expression_Type"/>
            <xs:element name="data_environment" type="qy:data_environment_expression_Type"/>
            <xs:element name="or" type="qy:or_expression_Type"/>
            <xs:element name="and" type="qy:and_expression_Type"/>
            <xs:element name="not" type="qy:not_expression_Type"/>
        </xs:choice>
    </xs:complexType>
    <!-- Or Expression -->
    <xs:complexType name="or_expression_Type">
        <xs:sequence>
            <xs:element name="operand" type="qy:characteristic_data_query_expression_Type" minOccurs="2"
                        maxOccurs="unbounded"/>
        </xs:sequence>
    </xs:complexType>
    <!-- And query expression -->
    <xs:complexType name="and_expression_Type">
        <xs:sequence>
            <xs:element name="operand" type="qy:characteristic_data_query_expression_Type" minOccurs="2"
                        maxOccurs="unbounded"/>
        </xs:sequence>
    </xs:complexType>
    <!-- Not Expression -->
    <xs:complexType name="not_expression_Type">
        <xs:sequence>
            <xs:element name="operand" type="qy:characteristic_data_query_expression_Type"/>
        </xs:sequence>
    </xs:complexType>
    <!-- -->
    <!-- Query Expression -->
    <xs:complexType name="query_expression_Type" abstract="true">
        <xs:sequence>
            <xs:element name="property_reference" type="qy:property_reference_Type"/>
        </xs:sequence>
    </xs:complexType>
    <!-- Cardinality Expression -->
    <xs:complexType name="cardinality_expression_Type">
        <xs:complexContent>
            <xs:extension base="qy:query_expression_Type">
                <xs:sequence>
                    <xs:element name="minimum" type="xs:int" minOccurs="0"/>
                    <xs:element name="maximum" type="xs:int" minOccurs="0"/>
                </xs:sequence>
            </xs:extension>
        </xs:complexContent>
    </xs:complexType>
    <!-- Range Expression -->
    <xs:complexType name="range_expression_Type">
        <xs:complexContent>
            <xs:extension base="qy:query_expression_Type">
                <xs:sequence>
                    <xs:element name="min_value" type="xs:float" minOccurs="0"/>
                    <xs:element name="max_value" type="xs:float" minOccurs="0"/>
                    <xs:element name="UOM_ref" type="id:IRDI_type" minOccurs="0"/>
                    <xs:element name="UOM_code" type="xs:string" minOccurs="0"/>
                    <xs:element name="currency_ref" type="id:IRDI_type" minOccurs="0"/>
                    <xs:element name="currency_code" type="bas:ISO_currency_code_Type" minOccurs="0"/>
                    <xs:element name="is_inclusive" type="xs:boolean"/>
                </xs:sequence>
            </xs:extension>
        </xs:complexContent>
    </xs:complexType>
    <!-- String Pattern Expression -->
    <xs:complexType name="string_pattern_expression_Type">
        <xs:complexContent>
            <xs:extension base="qy:query_expression_Type">
                <xs:sequence>
                    <xs:element name="pattern" type="xs:string"/>
                    <xs:element name="language_ref" type="id:IRDI_type" minOccurs="0"/>
                    <xs:element name="language_code" type="bas:ISO_language_code_Type" minOccurs="0"/>
                    <xs:element name="country_code" type="bas:ISO_country_code_Type" minOccurs="0"/>
                </xs:sequence>
            </xs:extension>
        </xs:complexContent>
    </xs:complexType>
    <!-- String Size Expression -->
    <xs:complexType name="string_size_expression_Type">
        <xs:complexContent>
            <xs:extension base="qy:query_expression_Type">
                <xs:sequence>
                    <xs:element name="min_length" type="xs:int" minOccurs="0"/>
                    <xs:element name="max_length" type="xs:int" minOccurs="0"/>
                    <xs:element name="language_ref" type="id:IRDI_type" minOccurs="0"/>
                    <xs:element name="language_code" type="bas:ISO_language_code_Type" minOccurs="0"/>
                    <xs:element name="country_code" type="bas:ISO_country_code_Type" minOccurs="0"/>
                </xs:sequence>
            </xs:extension>
        </xs:complexContent>
    </xs:complexType>
    <!-- Subset Expression -->
    <xs:complexType name="subset_expression_Type">
        <xs:complexContent>
            <xs:extension base="qy:query_expression_Type">
                <xs:sequence>
                    <xs:element name="value" type="qy:value_Type" minOccurs="0" maxOccurs="unbounded"/>
                </xs:sequence>
            </xs:extension>
        </xs:complexContent>
    </xs:complexType>
    <!-- Value -->
    <xs:complexType name="value_Type">
        <xs:sequence>
            <xs:group ref="val:value_Group"/>
        </xs:sequence>
    </xs:complexType>
    <!-- Data Environment Expression -->
    <xs:complexType name="data_environment_expression_Type">
        <xs:complexContent>
            <xs:extension base="qy:query_expression_Type">
                <xs:sequence>
                    <xs:element name="environment" type="qy:characteristic_data_query_expression_Type"
                                maxOccurs="unbounded"/>
                </xs:sequence>
            </xs:extension>
        </xs:complexContent>
    </xs:complexType>
    <!-- -->
    <!-- Property Reference -->
    <xs:complexType name="property_reference_Type">
        <xs:sequence>
            <xs:element name="index" type="xs:int" minOccurs="0"/>
            <xs:element name="property" type="qy:property_reference_Type" minOccurs="0"/>
        </xs:sequence>
        <xs:attribute name="property_ref" type="id:IRDI_type" use="required"/>
    </xs:complexType>
    <!-- Query -->
    <xs:complexType name="query_Type">
        <xs:sequence>
            <xs:element name="item_description" type="xs:string" minOccurs="0"/>
            <xs:element name="data_specification_ref" type="id:IRDI_type" minOccurs="0"/>
            <xs:element name="class_ref" type="id:IRDI_type" minOccurs="0"/>
            <xs:element name="property_ref" type="id:IRDI_list_type" minOccurs="0"/>
            <xs:element ref="cat:item" minOccurs="0"/>
            <xs:element ref="qy:characteristic_data_query_expression" minOccurs="0" maxOccurs="unbounded"/>
        </xs:sequence>
    </xs:complexType>
    <!-- Query Context -->
    <xs:complexType name="query_context_Type">
        <xs:sequence>
            <xs:element name="requesting_organization_ref" type="id:IRDI_type"/>
            <xs:element name="request_date_time" type="xs:dateTime"/>
            <xs:element name="response_due_date_time" type="xs:dateTime" minOccurs="0"/>
            <xs:element name="response_email_address" type="xs:string"/>
            <xs:element name="point_of_contact" type="xs:string" minOccurs="0"/>
            <xs:element ref="qy:query" minOccurs="0" maxOccurs="unbounded"/>
        </xs:sequence>
    </xs:complexType>
</xs:schema>

 \end{lstlisting}  

\section{catalogue.xsd}\index{Schema!catalogue.xsd}

 \begin{lstlisting}[caption=catalogue.xsd, language=XML, label=lst:catalogue_xsd]
<?xml version="1.0" encoding="UTF-8"?>
<!--
$Id: catalogue.xsd 332 2009-05-29 02:16:28Z radack $
ISO TC 184/SC 4/WG 12 N5178 - ISO/TS 29002-10 Catalogue - XML schema
-->
<!--
The following permission notice and disclaimer shall be included in all copies of this XML schema ("the Schema"), and derivations of the Schema:

Permission is hereby granted, free of charge in perpetuity, to any person obtaining a copy of the Schema, to use, copy, modify, merge and distribute free of charge, copies of the Schema for the purposes of developing, implementing, installing and using software based on the  Schema, and to permit persons to whom the Schema is furnished to do so, subject to the following conditions:

THE SCHEMA IS PROVIDED "AS IS", WITHOUT WARRANTY OF ANY KIND, EXPRESS OR IMPLIED, INCLUDING BUT NOT LIMITED TO THE WARRANTIES OF MERCHANTABILITY, FITNESS FOR A PARTICULAR PURPOSE AND NONINFRINGEMENT. IN NO EVENT SHALL THE AUTHORS OR COPYRIGHT HOLDERS BE LIABLE FOR ANY CLAIM, DAMAGES OR OTHER LIABILITY, WHETHER IN AN ACTION OF CONTRACT, TORT OR OTHERWISE, ARISING FROM, OUT OF OR IN CONNECTION WITH THE SCHEMA OR THE USE OR OTHER DEALINGS IN THE SCHEMA.

In addition, any modified copy of the Schema shall include the following notice:

THIS SCHEMA HAS BEEN MODIFIED FROM THE SCHEMA DEFINED IN ISO/TS 29002-10, AND SHOULD NOT BE INTERPRETED AS COMPLYING WITH THAT STANDARD.

-->
<xs:schema xmlns:xs="http://www.w3.org/2001/XMLSchema" xmlns:cat="urn:iso:std:iso:ts:29002:-10:ed-1:tech:xml-schema:catalogue" xmlns:val="urn:iso:std:iso:ts:29002:-10:ed-1:tech:xml-schema:value" xmlns:bas="urn:iso:std:iso:ts:29002:-4:ed-1:tech:xml-schema:basic" xmlns:id="urn:iso:std:iso:ts:29002:-5:ed-1:tech:xml-schema:identifier" targetNamespace="urn:iso:std:iso:ts:29002:-10:ed-1:tech:xml-schema:catalogue" elementFormDefault="qualified">
	<xs:import namespace="urn:iso:std:iso:ts:29002:-5:ed-1:tech:xml-schema:identifier" schemaLocation="identifier.xsd"/>
	<xs:import namespace="urn:iso:std:iso:ts:29002:-4:ed-1:tech:xml-schema:basic" schemaLocation="basic.xsd"/>
	<xs:import namespace="urn:iso:std:iso:ts:29002:-10:ed-1:tech:xml-schema:value" schemaLocation="value.xsd"/>
	<!-- Global Elements -->
	<xs:element name="catalogue" type="cat:catalogue_Type"/>
	<xs:element name="item" type="cat:item_Type"/>
	<xs:element name="property_value" type="cat:property_value_Type"/>
	<xs:element name="reference" type="cat:reference_Type"/>
	<!-- Global Types -->
	<!-- Catalogue -->
	<xs:complexType name="catalogue_Type">
		<xs:sequence>
			<xs:element ref="cat:item" minOccurs="0" maxOccurs="unbounded"/>
		</xs:sequence>
	</xs:complexType>
	<!-- Item -->
	<xs:complexType name="item_Type">
		<xs:sequence>
			<xs:element name="classification_ref" type="id:IRDI_type" minOccurs="0" maxOccurs="unbounded"/>
			<xs:element ref="cat:reference" minOccurs="0" maxOccurs="unbounded"/>
			<xs:element ref="cat:property_value" minOccurs="0" maxOccurs="unbounded"/>
		</xs:sequence>
		<xs:attribute name="class_ref" type="id:IRDI_type" use="required"/>
		<xs:attribute name="data_specification_ref" type="id:IRDI_type" use="optional"/>
		<xs:attribute name="local_id" type="xs:ID" use="optional"/>
		<xs:attribute name="information_supplier_reference_string" type="xs:string" use="optional"/>
		<xs:attribute name="is_proprietary" type="xs:boolean" use="optional"/>
		<xs:attribute name="is_dependent" type="xs:boolean" use="optional"/>
		<xs:attribute name="is_global_id" type="xs:boolean" use="optional"/>
		<xs:attribute name="is_model" type="xs:boolean" use="optional"/>
		<xs:attribute name="created_view" type="id:IRDI_type" use="optional"/>
		<xs:attribute name="view_of" type="xs:IDREF" use="optional"/>
	</xs:complexType>
	<!-- Property Value -->
	<xs:complexType name="property_value_Type">
		<xs:sequence>
			<xs:group ref="val:extended_value_Group"/>
			<xs:element ref="val:environment" minOccurs="0"/>
		</xs:sequence>
		<xs:attribute name="property_ref" type="id:IRDI_type" use="required"/>
		<xs:attribute name="subitem_path_property_ref" type="id:IRDI_list_type" use="optional"/>
		<xs:attribute name="is_proprietary" type="xs:boolean" use="optional"/>
	</xs:complexType>
	<!-- Reference -->
	<xs:complexType name="reference_Type">
		<xs:sequence>
			<xs:element name="designation" type="bas:international_text_Type" minOccurs="0"/>
		</xs:sequence>
		<xs:attribute name="organization_ref" type="id:IRDI_type" use="optional"/>
		<xs:attribute name="organization_code" type="xs:string" use="optional"/>
		<!-- Constraint: organization_ref or organization_code (or both) must be specified. -->
		<!-- Constraint: If both organization_ref and organization_code are specified, they must denote the same organization. -->
		<xs:attribute name="reference_number" type="xs:string" use="required"/>
	</xs:complexType>
</xs:schema>

 \end{lstlisting} 
 
 \section{basic.xsd}\index{Schema!basic.xsd}

 \begin{lstlisting}[caption=basic.xsd, language=XML, label=lst:basic_xsd]
<?xml version="1.0" encoding="UTF-8"?>
<!--
$Id: basic.xsd 411 2009-07-21 01:57:47Z radack $
ISO TC 184/SC 4/WG 12 N6650 - ISO/TS 29002-4 Basic - XML schema
-->
<!--
The following permission notice and disclaimer shall be included in all copies of this XML schema ("the Schema"), and derivations of the Schema:

Permission is hereby granted, free of charge in perpetuity, to any person obtaining a copy of the Schema, to use, copy, modify, merge and distribute free of charge, copies of the Schema for the purposes of developing, implementing, installing and using software based on the  Schema, and to permit persons to whom the Schema is furnished to do so, subject to the following conditions:

THE SCHEMA IS PROVIDED "AS IS", WITHOUT WARRANTY OF ANY KIND, EXPRESS OR IMPLIED, INCLUDING BUT NOT LIMITED TO THE WARRANTIES OF MERCHANTABILITY, FITNESS FOR A PARTICULAR PURPOSE AND NONINFRINGEMENT. IN NO EVENT SHALL THE AUTHORS OR COPYRIGHT HOLDERS BE LIABLE FOR ANY CLAIM, DAMAGES OR OTHER LIABILITY, WHETHER IN AN ACTION OF CONTRACT, TORT OR OTHERWISE, ARISING FROM, OUT OF OR IN CONNECTION WITH THE SCHEMA OR THE USE OR OTHER DEALINGS IN THE SCHEMA.

In addition, any modified copy of the Schema shall include the following notice:

THIS SCHEMA HAS BEEN MODIFIED FROM THE SCHEMA DEFINED IN ISO/TS 29002-4, AND SHOULD NOT BE INTERPRETED AS COMPLYING WITH THAT STANDARD.

-->
<xs:schema xmlns:xs="http://www.w3.org/2001/XMLSchema" xmlns:basic="urn:iso:std:iso:ts:29002:-4:ed-1:tech:xml-schema:basic" xmlns:id="urn:iso:std:iso:ts:29002:-5:ed-1:tech:xml-schema:identifier" targetNamespace="urn:iso:std:iso:ts:29002:-4:ed-1:tech:xml-schema:basic" elementFormDefault="qualified">
	<xs:import namespace="urn:iso:std:iso:ts:29002:-5:ed-1:tech:xml-schema:identifier" schemaLocation="identifier.xsd"/>
	<!-- Global Types -->
	<!-- Day Interval -->
	<xs:simpleType name="day_interval_Type">
		<xs:union memberTypes="xs:gYear xs:gYearMonth xs:date"/>
	</xs:simpleType>
	<!-- ASN.1 Identifier -->
	<xs:simpleType name="ASN1_identifier_Type">
		<xs:restriction base="xs:string"/>
	</xs:simpleType>
	<!-- International Text -->
	<xs:complexType name="international_text_Type">
		<xs:sequence>
			<xs:element name="local_string" type="basic:language_string_Type" maxOccurs="unbounded"/>
		</xs:sequence>
	</xs:complexType>
	<!-- ISO Country Code -->
	<xs:simpleType name="ISO_country_code_Type">
		<xs:restriction base="xs:string">
			<xs:pattern value="[A-Z]{2}"/>
		</xs:restriction>
	</xs:simpleType>
	<!-- ISO Currency Code -->
	<xs:simpleType name="ISO_currency_code_Type">
		<xs:restriction base="xs:string">
			<xs:length value="3"/>
		</xs:restriction>
	</xs:simpleType>
	<!-- ISO Language Code -->
	<xs:simpleType name="ISO_language_code_Type">
		<xs:restriction base="xs:string">
			<xs:pattern value="[a-z]{2}"/>
			<xs:pattern value="[a-z]{3}"/>
		</xs:restriction>
	</xs:simpleType>
	<!-- Language String -->
	<xs:complexType name="language_string_Type">
		<xs:sequence>
			<xs:element name="content" type="xs:string"/>
			<xs:element name="language_ref" type="id:IRDI_type" minOccurs="0"/>
			<xs:sequence minOccurs="0">
				<xs:element name="language_code" type="basic:ISO_language_code_Type"/>
				<xs:element name="country_code" type="basic:ISO_country_code_Type" minOccurs="0"/>
			</xs:sequence>
			<!-- Constraint: language or language_code (or both) must be specified. -->
			<!-- Constraint: If both language and language_code are specified, then language and the combination of language_code and country_code must denote the same localized language. -->
		</xs:sequence>
	</xs:complexType>
</xs:schema>

 \end{lstlisting} 
 
\section{identifier.xsd}\index{Schema!identifier.xsd}

 \begin{lstlisting}[caption=identifier.xsd, language=XML, label=lst:identifier_xsd]
<?xml version="1.0" encoding="UTF-8"?>
<!--
$Id: identifier.xsd 420 2009-07-25 21:51:57Z radack $
ISO TC 184/SC 4/WG 12 N5180 - ISO/TS 29002-5 Identifier - XML schema
-->
<!--
The following permission notice and disclaimer shall be included in all copies of this XML schema ("the Schema"), and derivations of the Schema:

Permission is hereby granted, free of charge in perpetuity, to any person obtaining a copy of the Schema, to use, copy, modify, merge and distribute free of charge, copies of the Schema for the purposes of developing, implementing, installing and using software based on the  Schema, and to permit persons to whom the Schema is furnished to do so, subject to the following conditions:

THE SCHEMA IS PROVIDED "AS IS", WITHOUT WARRANTY OF ANY KIND, EXPRESS OR IMPLIED, INCLUDING BUT NOT LIMITED TO THE WARRANTIES OF MERCHANTABILITY, FITNESS FOR A PARTICULAR PURPOSE AND NONINFRINGEMENT. IN NO EVENT SHALL THE AUTHORS OR COPYRIGHT HOLDERS BE LIABLE FOR ANY CLAIM, DAMAGES OR OTHER LIABILITY, WHETHER IN AN ACTION OF CONTRACT, TORT OR OTHERWISE, ARISING FROM, OUT OF OR IN CONNECTION WITH THE SCHEMA OR THE USE OR OTHER DEALINGS IN THE SCHEMA.

In addition, any modified copy of the Schema shall include the following notice:

THIS SCHEMA HAS BEEN MODIFIED FROM THE SCHEMA DEFINED IN ISO/TS 29002-5, AND SHOULD NOT BE INTERPRETED AS COMPLYING WITH THAT STANDARD.

-->
<!DOCTYPE xs:schema [
	<!ENTITY % identifier.dtd SYSTEM "identifier.dtd">
	%identifier.dtd;
]>
<xs:schema xmlns:xs="http://www.w3.org/2001/XMLSchema" xmlns:id="urn:iso:std:iso:ts:29002:-5:ed-1:tech:xml-schema:identifier" targetNamespace="urn:iso:std:iso:ts:29002:-5:ed-1:tech:xml-schema:identifier" elementFormDefault="qualified" attributeFormDefault="unqualified">
	<!-- IRDI -->
	<xs:element name="IRDI" type="id:IRDI_type"/>
	<xs:element name="IRDI_list" type="id:IRDI_list_type"/>
	<xs:simpleType name="IRDI_type">
		<xs:restriction base="xs:string">
			<xs:pattern value="&irdi1;"/>
			<xs:pattern value="&irdi2;"/>
			<xs:pattern value="&irdi3;"/>
		</xs:restriction>
	</xs:simpleType>
	<!-- IRDI sequence -->
	<xs:complexType name="IRDI_sequence_type">
		<xs:sequence>
			<xs:element ref="id:IRDI" minOccurs="0" maxOccurs="unbounded"/>
		</xs:sequence>
	</xs:complexType>
	<!-- IRDI list-->
	<xs:simpleType name="IRDI_list_type">
		<xs:list itemType="id:IRDI_type"/>
	</xs:simpleType>
</xs:schema>

 \end{lstlisting} 

\section{value.xsd}\index{Schema!value.xsd}
 \begin{lstlisting}[caption=value.xsd, language=XML, label=lst:value_xsd]
<?xml version="1.0" encoding="UTF-8"?>
<!--
$Id: value.xsd 332 2009-05-29 02:16:28Z radack $
ISO TC 184/SC 4/WG 12 N5192 - ISO/TS 29002-10 Value - XML schema
-->
<!--
The following permission notice and disclaimer shall be included in all copies of this XML schema ("the Schema"), and derivations of the Schema:

Permission is hereby granted, free of charge in perpetuity, to any person obtaining a copy of the Schema, to use, copy, modify, merge and distribute free of charge, copies of the Schema for the purposes of developing, implementing, installing and using software based on the  Schema, and to permit persons to whom the Schema is furnished to do so, subject to the following conditions:

THE SCHEMA IS PROVIDED "AS IS", WITHOUT WARRANTY OF ANY KIND, EXPRESS OR IMPLIED, INCLUDING BUT NOT LIMITED TO THE WARRANTIES OF MERCHANTABILITY, FITNESS FOR A PARTICULAR PURPOSE AND NONINFRINGEMENT. IN NO EVENT SHALL THE AUTHORS OR COPYRIGHT HOLDERS BE LIABLE FOR ANY CLAIM, DAMAGES OR OTHER LIABILITY, WHETHER IN AN ACTION OF CONTRACT, TORT OR OTHERWISE, ARISING FROM, OUT OF OR IN CONNECTION WITH THE SCHEMA OR THE USE OR OTHER DEALINGS IN THE SCHEMA.

In addition, any modified copy of the Schema shall include the following notice:

THIS SCHEMA HAS BEEN MODIFIED FROM THE SCHEMA DEFINED IN ISO/TS 29002-10, AND SHOULD NOT BE INTERPRETED AS COMPLYING WITH THAT STANDARD.

-->
<xs:schema xmlns:xs="http://www.w3.org/2001/XMLSchema" xmlns:val="urn:iso:std:iso:ts:29002:-10:ed-1:tech:xml-schema:value" xmlns:bas="urn:iso:std:iso:ts:29002:-4:ed-1:tech:xml-schema:basic" xmlns:id="urn:iso:std:iso:ts:29002:-5:ed-1:tech:xml-schema:identifier" targetNamespace="urn:iso:std:iso:ts:29002:-10:ed-1:tech:xml-schema:value" elementFormDefault="qualified">
	<xs:import namespace="urn:iso:std:iso:ts:29002:-5:ed-1:tech:xml-schema:identifier" schemaLocation="identifier.xsd"/>
	<xs:import namespace="urn:iso:std:iso:ts:29002:-4:ed-1:tech:xml-schema:basic" schemaLocation="basic.xsd"/>
	<!-- Global Elements -->
	<xs:element name="combination" type="val:combination_Type"/>
	<xs:element name="bag_value" type="val:bag_value_Type"/>
	<xs:element name="boolean_value" type="val:boolean_value_Type"/>
	<xs:element name="complex_value" type="val:complex_value_Type"/>
	<xs:element name="composite_value" type="val:composite_value_Type"/>
	<xs:element name="controlled_value" type="val:controlled_value_Type"/>
	<xs:element name="currency_value" type="val:currency_value_Type"/>
	<xs:element name="date_value" type="val:date_value_Type"/>
	<xs:element name="date_time_value" type="val:date_time_value_Type"/>
	<xs:element name="environment" type="val:environment_Type"/>
	<xs:element name="file_value" type="val:file_value_Type"/>
	<xs:element name="integer_value" type="val:integer_value_Type"/>
	<xs:element name="item_reference_value" type="val:item_reference_value_Type"/>
	<xs:element name="localized_text_value" type="val:localized_text_value_Type"/>
	<xs:element name="measure_qualified_number_value" type="val:measure_qualified_number_value_Type"/>
	<xs:element name="measure_range_value" type="val:measure_range_value_Type"/>
	<xs:element name="measure_single_number_value" type="val:measure_single_number_value_Type"/>
	<xs:element name="null_value" type="val:null_value_Type"/>
	<xs:element name="one_of" type="val:one_of_Type"/>
	<xs:element name="rational_value" type="val:rational_value_Type"/>
	<xs:element name="real_value" type="val:real_value_Type"/>
	<xs:element name="sequence_value" type="val:sequence_value_Type"/>
	<xs:element name="set_value" type="val:set_value_Type"/>
	<xs:element name="string_value" type="val:string_value_Type"/>
	<xs:element name="time_value" type="val:time_value_Type"/>
	<xs:element name="year_month_value" type="val:year_month_value_Type"/>
	<xs:element name="year_value" type="val:year_value_Type"/>
	<!-- Groups -->
	<!-- Extended Value -->
	<xs:group name="extended_value_Group">
		<xs:choice>
			<xs:group ref="val:value_Group"/>
			<xs:element ref="val:one_of"/>
			<xs:element ref="val:combination"/>
		</xs:choice>
	</xs:group>
	<!-- Value -->
	<xs:group name="value_Group">
		<xs:choice>
			<!-- The first element in the following list is the one that will be created by default in some XML editors whenever a property_value is added. -->
			<xs:element ref="val:string_value"/>
			<xs:element ref="val:bag_value"/>
			<xs:element ref="val:boolean_value"/>
			<xs:element ref="val:complex_value"/>
			<xs:element ref="val:composite_value"/>
			<xs:element ref="val:controlled_value"/>
			<xs:element ref="val:currency_value"/>
			<xs:element ref="val:date_value"/>
			<xs:element ref="val:date_time_value"/>
			<xs:element ref="val:file_value"/>
			<xs:element ref="val:integer_value"/>
			<xs:element ref="val:item_reference_value"/>
			<xs:element ref="val:localized_text_value"/>
			<xs:element ref="val:measure_qualified_number_value"/>
			<xs:element ref="val:measure_range_value"/>
			<xs:element ref="val:measure_single_number_value"/>
			<xs:element ref="val:null_value"/>
			<xs:element ref="val:rational_value"/>
			<xs:element ref="val:real_value"/>
			<xs:element ref="val:sequence_value"/>
			<xs:element ref="val:set_value"/>
			<xs:element ref="val:time_value"/>
			<xs:element ref="val:year_month_value"/>
			<xs:element ref="val:year_value"/>
		</xs:choice>
	</xs:group>
	<!-- Numeric Value -->
	<xs:group name="numeric_value">
		<xs:choice>
			<!-- The first element in the following list is the one that will be created by default in some XML editors whenever a property_value is added. -->
			<xs:element ref="val:real_value"/>
			<xs:element ref="val:complex_value"/>
			<xs:element ref="val:integer_value"/>
			<xs:element ref="val:rational_value"/>
		</xs:choice>
	</xs:group>
	<!-- Global Types -->
	<!-- Bag Value -->
	<xs:complexType name="bag_value_Type">
		<xs:sequence>
			<xs:group ref="val:value_Group" minOccurs="0" maxOccurs="unbounded"/>
		</xs:sequence>
	</xs:complexType>
	<!-- Boolean Value -->
	<xs:complexType name="boolean_value_Type">
		<xs:simpleContent>
			<xs:extension base="xs:boolean"/>
		</xs:simpleContent>
	</xs:complexType>
	<!-- Combination -->
	<xs:complexType name="combination_Type">
		<xs:sequence>
			<xs:group ref="val:value_Group" maxOccurs="unbounded"/>
		</xs:sequence>
	</xs:complexType>
	<!-- Complex Value -->
	<xs:complexType name="complex_value_Type">
		<xs:sequence>
			<xs:element name="real_part" type="xs:double"/>
			<xs:element name="imaginary_part" type="xs:double"/>
		</xs:sequence>
	</xs:complexType>
	<!-- Composite Value -->
	<xs:complexType name="composite_value_Type">
		<xs:sequence>
			<xs:element name="field" type="val:field_Type" minOccurs="0" maxOccurs="unbounded"/>
		</xs:sequence>
	</xs:complexType>
	<!-- Condition Element -->
	<xs:complexType name="condition_element_Type">
		<xs:sequence>
			<xs:group ref="val:value_Group"/>
		</xs:sequence>
		<xs:attribute name="property_ref" type="id:IRDI_type" use="required"/>
	</xs:complexType>
	<!-- Controlled Value -->
	<xs:complexType name="controlled_value_Type">
		<xs:attribute name="value_ref" type="id:IRDI_type" use="optional"/>
		<xs:attribute name="value_code" type="xs:string" use="optional"/>
		<!-- Constraint: value_ref or value_code (or both) must be specified. -->
		<!-- Constraint: If both value_ref and value_code specified, they must denote the same concept. -->
	</xs:complexType>
	<!-- Currency Value -->
	<xs:complexType name="currency_value_Type">
		<xs:sequence>
			<xs:group ref="val:numeric_value"/>
		</xs:sequence>
		<xs:attribute name="currency_ref" type="id:IRDI_type" use="optional"/>
		<xs:attribute name="currency_code" type="bas:ISO_currency_code_Type" use="optional"/>
		<!-- Constraint: currency_ref or currency_code (or both) must be specified. -->
		<!-- Constraint: If both currency_ref and currency_code are specified, they must denote the same concept. -->
	</xs:complexType>
	<!-- Date Value -->
	<xs:complexType name="date_value_Type">
		<xs:simpleContent>
			<xs:extension base="xs:date"/>
		</xs:simpleContent>
	</xs:complexType>
	<!-- Date Time Value -->
	<xs:complexType name="date_time_value_Type">
		<xs:simpleContent>
			<xs:extension base="xs:dateTime"/>
		</xs:simpleContent>
	</xs:complexType>
	<!-- Environment -->
	<xs:complexType name="environment_Type">
		<xs:sequence>
			<xs:element name="property_value" type="val:condition_element_Type" maxOccurs="unbounded"/>
		</xs:sequence>
	</xs:complexType>
	<!-- Field -->
	<xs:complexType name="field_Type">
		<xs:sequence>
			<xs:group ref="val:value_Group"/>
		</xs:sequence>
		<xs:attribute name="property_ref" type="id:IRDI_type" use="optional"/>
	</xs:complexType>
	<!-- File Value -->
	<xs:complexType name="file_value_Type">
		<xs:sequence>
			<xs:element name="URI" type="xs:anyURI"/>
		</xs:sequence>
	</xs:complexType>
	<!-- Integer Value -->
	<xs:complexType name="integer_value_Type">
		<xs:simpleContent>
			<xs:extension base="xs:int"/>
		</xs:simpleContent>
	</xs:complexType>
	<!-- Item Reference Value -->
	<xs:complexType name="item_reference_value_Type">
		<xs:attribute name="item_local_ref" type="xs:IDREF" use="required"/>
	</xs:complexType>
	<!-- Localized Text Value -->
	<xs:complexType name="localized_text_value_Type">
		<xs:sequence>
			<xs:element name="content" type="bas:international_text_Type"/>
		</xs:sequence>
	</xs:complexType>
	<!-- Measure Qualified Number Value -->	
	<xs:complexType name="measure_qualified_number_value_Type">
		<xs:complexContent>
			<xs:extension base="val:measure_value_Type">
				<xs:sequence>
					<xs:element name="qualified_value" type="val:qualified_value_Type" maxOccurs="unbounded"/>	
				</xs:sequence>
			</xs:extension>
		</xs:complexContent>
	</xs:complexType>
	<!-- Measure Range Value -->
	<xs:complexType name="measure_range_value_Type">
		<xs:complexContent>
			<xs:extension base="val:measure_value_Type">
				<xs:sequence>
					<xs:element name="lower_value" type="val:numeric_value_Type"/>
					<xs:element name="upper_value" type="val:numeric_value_Type"/>
				</xs:sequence>
			</xs:extension>
		</xs:complexContent>
	</xs:complexType>
	<!-- Measure Single Number Value -->
	<xs:complexType name="measure_single_number_value_Type">
		<xs:complexContent>
			<xs:extension base="val:measure_value_Type">
				<xs:sequence>
					<xs:choice>
						<xs:group ref="val:numeric_value"/>
					</xs:choice>
				</xs:sequence>
			</xs:extension>
		</xs:complexContent>
	</xs:complexType>
	<!-- Measure Value -->
	<xs:complexType name="measure_value_Type" abstract="true">
		<xs:attribute name="UOM_ref" type="id:IRDI_type" use="optional"/>
		<xs:attribute name="UOM_code" type="xs:string" use="optional"/>
		<!-- Constraint: UOM_ref or UOM_code (or both) must be specified. -->
		<!-- Constraint: If UOM_ref or UOM_code are specified, they must denote the same concept. -->
	</xs:complexType>
	<!-- Null Value -->
	<xs:complexType name="null_value_Type"/>
	<!-- Numeric Value -->
	<xs:complexType name="numeric_value_Type">
		<xs:sequence>
			<xs:group ref="val:numeric_value"/>
		</xs:sequence>
	</xs:complexType>
	<!-- OneOf -->
	<xs:complexType name="one_of_Type">
		<xs:sequence>
			<xs:choice maxOccurs="unbounded">
				<xs:element ref="val:combination"/>
				<xs:group ref="val:value_Group"/>
			</xs:choice>
		</xs:sequence>
	</xs:complexType>
	<!-- Qualified Value -->
	<xs:complexType name="qualified_value_Type">
		<xs:sequence>
			<xs:group ref="val:numeric_value"/>
		</xs:sequence>
		<xs:attribute name="qualifier_ref" type="id:IRDI_type" use="optional"/>
		<xs:attribute name="qualifier_code" type="xs:string" use="optional"/>
		<!-- Constraint: qualifier_ref or qualifier_code (or both) must be specified. -->
		<!-- Constraint: If both qualifier_ref or qualifier_code are specified, they must denote the same concept. -->
	</xs:complexType>
	<!-- Rational Value -->
	<xs:complexType name="rational_value_Type">
		<xs:sequence>
			<xs:element name="whole_part" type="xs:int" minOccurs="0"/>
			<xs:element name="numerator" type="xs:int"/>
			<xs:element name="denominator" type="xs:int"/>
		</xs:sequence>
	</xs:complexType>
	<!-- Real Value -->
	<xs:complexType name="real_value_Type">
		<xs:simpleContent>
			<xs:extension base="xs:double"/>
		</xs:simpleContent>
	</xs:complexType>
	<!-- Sequence Value -->
	<xs:complexType name="sequence_value_Type">
		<xs:sequence>
			<xs:group ref="val:value_Group" minOccurs="0" maxOccurs="unbounded"/>
		</xs:sequence>
	</xs:complexType>
	<!-- Set Value -->
	<xs:complexType name="set_value_Type">
		<xs:sequence>
			<xs:group ref="val:value_Group" minOccurs="0" maxOccurs="unbounded"/>
		</xs:sequence>
	</xs:complexType>
	<!-- String Value -->
	<xs:complexType name="string_value_Type">
		<xs:simpleContent>
			<xs:extension base="xs:string"/>
		</xs:simpleContent>
	</xs:complexType>
	<!-- Time Value -->
	<xs:complexType name="time_value_Type">
		<xs:simpleContent>
			<xs:extension base="xs:time"/>
		</xs:simpleContent>
	</xs:complexType>
	<!-- Year Month Value -->
	<xs:complexType name="year_month_value_Type">
		<xs:simpleContent>
			<xs:extension base="xs:gYearMonth"/>
		</xs:simpleContent>
	</xs:complexType>
	<!-- Year Value -->
	<xs:complexType name="year_value_Type">
		<xs:simpleContent>
			<xs:extension base="xs:gYear"/>
		</xs:simpleContent>
	</xs:complexType>
</xs:schema>

 \end{lstlisting}  
\chapter{Analyse ISO 29002-31 - Exchange of characteristics data}\label{kap:analyse2900231}

\section{XML Datencontaineranalyse ISO 29002-31}\index{ISO 29002-31}
Die Unterkapitel beschreiben die einzelnen XML-Datencontainer aus der ISO 29002-31. Der Ausgangspunkt ist der query\_context, welcher einige Metadaten zum eigentlichen query enthält. 

\begin{figure}[htbp]
	\centering
		\includegraphics[width=0.80\textwidth]{images/query_main.jpg}
		\caption[UML-Diagramm Query Main]{UML-Diagramm Query Main\footnotemark}
	\label{fig:querymain}
\end{figure}
\footnotetext{Quelle: ISO 29002-31 Kapitel 5.2.1}

\subsection{query\_context}
Dies ist eine Art Container für eine Menge von Queries. Inhalt sind Informationen über den Anforderer der Daten zwecks persönlicher Kontaktaufnahme, wie z.B. die Anfragezeit, Informationen über die Organisation, welche die Anfrage schickt, sowie einen gewünschten Antwortzeitpunkt mit Antwort-E-mail Adresse. Siehe dazu \autoref{fig:querymain} und \citep[vgl][Kap. 5.2.2]{iso29002-31}.  

Da die Vorgabe lautet, den Service auf Basis eines \glspl{Webservice} zu erstellen, entfällt die Benutzung des query\_context. Der Grund ist, dass der Kontext  implizit durch den \gls{Webservice} respektive dem Server zur Verfügung gestellt wird. Beispielsweise wird die Anfragezeit zwar nicht explizit durch den Serviceaufrufer selbst übergeben, allerdings durch die Anfrage an den technischen Server wie z.B. Apache Tomcat Server mittels Logeintrag implizit ermittelt. Somit lassen sich diese Metadaten über Verbindungsprotokolle der Infrastruktur herausfinden.  
Siehe dazu auch \citep[Kap. 6][]{iso29002-31}, welche besagt: \\ \enquote{ISO/TS 29002 can be implemented: \\
a. with another envelope standard, such as EDI, or \\
b. by itself, using the query\_context to carry envelope information.}

\subsection{query}\label{sec:query}
Die Unterstützung aller Funktionalitäten des queries entspricht laut ISO 29002-31 der Conformance class 1: simple query \citep[Anhang 6][]{iso29002-31}.
Dies ist der eigentliche Abfrage-Datensatz. Abgefragt werden kann mittels class \gls{IRDI},\footnote{IRDI  - International registration data identifier}, data\_specification IRDI, eine Menge von property \gls{IRDI}, Teiledaten (das sind \glslink{item}{Teile} gefüllt mit Daten ihrer Eigenschaften, die dem Klienten bereits bekannt sind) und einer item\_description. Das bedeutet, dass bereits bekannte Eigenschaften eines \glslink{item}{Teils} übertragen werden können, um die Suche auf \glslink{item}{Teile} mit diesen Werte-Eigenschaften einzuschränken.

Die data\_specification IRDI verweist auf eine Spezifikation aus ISO 22745-30, die besagt welche Properties für dieses \glslink{item}{Teil} sinnvoll sind. Die angegebenen Property \glspl{IRDI} sind dann eine Teilmenge aus den mittels data\_specification IRDI definierten erlaubten Eigenschaften. Für weitere Informationen zur ISO 22745-30 siehe \autoref{kap:identification_guide_anhang}. 

Denkbar sind einfache Abfragen wie z.B.: \enquote{Gib mir alle Teile der Klasse xyz}. Mitgeliefert werden auch \glslink{item}{Teile} von Subklassen. Weiterhin kann die Abfrage nach bestimmten Eigenschaften eingeschränkt werden. Eine weitere Möglichkeit ist es, bereits bekannte Daten über ein \glslink{item}{Teil} zu übermitteln, mit dem Zwecke hierüber die \gls{IRDI}, zu erfahren oder weitere Eigenschaftsdaten zu erhalten. Siehe Beispielqueries simple queries in \autoref{kap:query_beispiele}. 

\subsection{characteristic\_data\_query\_expression (parametric\_query)}\label{sec:characteristicdataqueryexpression}
Das entspricht laut ISO 29002-31 Anhang 6 der Conformance class 2: parametric query.

\begin{figure}[htbp]
	\centering
		\includegraphics[width=0.99\textwidth]{images/query_expression.jpg}
		\caption[UML-Diagramm Query Expression]{UML-Diagramm Query Expression\footnotemark}
	\label{fig:umlqueryexpression}
\end{figure}
\footnotetext{Quelle: ISO 29002-31 Kapitel 5.3.1}

Eine characteristic\_data\_query\_expression kann verschiedene expressions vom Typ query\_expression beinhalten. Von jedem Typ jeweils nur maximal eine. 
Z.B.
\begin{itemize}
\item string\_size\_expression
\item string\_pattern\_expression
\item range\_expression
\item data\_environment\_expression
\item cardinality\_expression
\item subset\_expression
\end{itemize}

Darüberhinaus noch folgende Attribute:

\begin{itemize}
\item property\_reference - die property auf den die query\_expression bezogen ist
\end{itemize}
Solch eine Expression ermöglicht das Filtern, gleichsam ein Einschränken bestimmter Properties und Werte. 

\subsection{Query Beispiele}\label{kap:query_beispiele}

Nachfolgend seien einige Query-Beispielszenarien aufgestellt, die sich aus der Analyse der Standards ergeben.

Eine Schraube hat die folgenden möglichen Eigenschaften: 

\begin{description}
\item[Klassen-Identifier] 1234-abcd\# ab-cdefgh\# 1 (IRDI)
\item[Typ] M6 (Property IRDI: 1234-abcd\# ab-bbbbbb\# 1)
\item[Länge] 80mm (Property IRDI: 1234-abcd\# ab-cccccc\# 1)
\end{description}

\subsubsection{Simple Query}\index{Query!Simple Query}

Ein simpler query ermöglicht folgende Abfrage: \enquote{Gib mir alle Teile zum Konzept Kreuzschraube mit dem Identifier (IRDI) 1234-abcd\#ab-cdefgh\#1}. Das Ergebnis ist ein \glslink{item}{Teil}, mit allen Attributen wie oben angegeben. 

Ein anderer Query könnte lauten: \enquote{Gib mir die Properties 1234-abcd\#ab-cccccc\#1 und 1234-abcd\#bbbbbb\# 1 des \glspl{item} der Klasse 1234-abcd\#ab-cdefgh\#1}. Das Ergebnis wäre das \glslink{item}{Teil} mit Typ: M6 und der Länge: 80mm.

Es könnte auch mit Hilfe von vorhandenen Daten gesucht werden, z.B.:  \enquote{Hier ist ein \glslink{item}{Teil} mit der Property Typ: M6 (Property IRDI: 1234-abcd\# ab-bbbbbb\# 1), gib mir bitte dazu die Properties 1234-abcd\#ab-cccccc\#1 und 1234-abcd\# bbbbbb\#1} 

\subsubsection{Parametric Query}\index{Query!Parametric Query}

Hat man jetzt noch eine Schraube mit folgenden Eigenschaften:
\begin{description}
\item[Klassen-Identifier] 1234-abcd\#ab-cdefgh\#1 (\gls{IRDI})
\item[Typ] M5 (Property IRDI: 1234-abcd\#xx-bbbbbb\#1)
\item[Länge] 100mm (Property \gls{IRDI}: 1234-abcd\#xx-cccccc\#1)
\end{description}

ermöglicht der Parametric Query mit Hilfe der characteristic\_data\_query\_expression folgende Abfragen:  \enquote{Gib mir die Properties 1234-abcd\# ab-cccccc\#1 (Länge) und 1234-abcd\#bbbbbb\#1 (Typ) des Konzeptes 1234-abcd\#ab-cdefgh\#1 (Schraube) mit einer Länge zwischen 50 und 150mm und dem Typen M5 oder M6.}

Dies ermöglicht das Filtern auf genau eine übergebene Property. Rekursive Abfragen sind auch möglich, beispielsweise wenn die gesuchte Property eine Multi-Property ist (Property: Loch als Wert zwei Properties mit Form und Durchmesser und Durchmesser soll gefiltert werden).

\section{Analyse ISO 22745-30 - Identification Guide}\label{kap:identification_guide_anhang}\index{ISO 22745-30}\index{Identification Guide}

Ein \gls{IG} beschreibt, welche Daten für ein Objekt benötigt werden, damit dies überhaupt sinnvoll für einen bestimmten Zweck eingesetzt werden kann. Der Käufer, Produktmanager oder Benutzer definiert die Anforderungen an die Daten. Ein  \enquote{Datenanforderungsstatement} wird als ein i-xml \gls{IG} xml file erzeugt \citep[vgl.][Slide 14 - Automating the Data Supply Chain]{bensonQuality}. Es wird die Frage beantwortet, welche Daten (Properties) zu einem bestimmten Konzept eines Objektes benötigt werden, um den Artikel zu kaufen oder zu sinnvoll zu verwalten. Diese Anforderungen werden von der Abfrageseite (Kundenseite) definiert, also derjenige, der Daten abfragen möchte\footnote{Quelle: ECCMA\_ISO\_8000\_certification.pdf - Zertifizierungspräsentation der ECCMA zur ISO 8000}.
Ein \gls{IG} referenziert Konzepte eines \glslink{Dictionary}{Dictionaries}, um Datenanforderungen einer bestimmten Klasse zu beschreiben \citep[vgl.][Kapitel 5]{iso22745-30}.  
Ein Datenempfänger kann eine Organisation oder eine Gruppe von Organisationen oder Firmen sein, welche ähnliche Datenanforderungen haben. Somit wird eine \gls{IG} Gruppe von einer speziellen Organisation verwaltet, welche wiederum selbst Datenempfänger sein kann.  


%\setchapterpreamble[u]{
%\dictum[Johann Wolfgang von Goethe]{Es ist nicht genug, zu wissen, man muß auch anwenden; es ist nicht genug, zu wollen, man muß auch tun. \dots}}
\chapter{Anwendungsfälle}\label{kap:analyse_use_cases}
% Funktionale Anforderungen

Dieses Kapitel beinhaltet weiteren Anwendungsfälle des Systems. Der in \autoref{kap:Use_Cases} bereits beschriebenen Anwendungsfall wird hier nicht erneut aufgeführt.  

\subsection{Charakteristische Daten eines Produkts validieren}

{\small

\begin{description}
     \item[use case] Charakteristische Daten validieren
     \item[  actors]~\\
     Klient
     \item[  precondition]~\\
     Der Klient verwendet einen gültigen Identifier sowie auf den Identifier passende Daten..
     \item[  main flow]~\\
     Der Klient gibt einen Identifier eines \glslink{item}{Teils} ein. Zusätzlich übermittelt er zu diesem bekanntem \glslink{item}{Teil} Eigenschaften dieser Instanz des \glslink{item}{Teils} und sendet eine Anfrage ab. Die Anfrage wird auf Gültigkeit überprüft. Als Antwort bekommt er ein oder mehrere Datensätze von \glslink{item}{Teilen} mit den entsprechenden charakteristischen Daten zurück, auf welche die übergebenen Eigenschaften zutreffen. 
     \item[  postcondition]~\\
     Alle Daten aller \glslink{item}{Teile} der gewählten Klassen des Identifiers werden zurückgegeben. Dies ermöglicht dem Klienten eine Validierung der ihm bereits bekannten Daten über ein Element.
      \item[  alternative flow]~\\
     Die übermittelten Eigenschaften des \glslink{item}{Teils} stimmen nicht mit den gespeicherten Daten überein.
      \item[  postcondition]~\\
     Es werden keine Daten zurückgeliefert. Die übermittelten Daten sind nicht valide. 
      \item[  exceptional flow] Ungültige Identifier oder ungültige Anfrage ~\\
     Der oder die übergebenen Identifier oder die gesamte Anfrage ist gemäß Spezifikation ungültig.   
     \item[  postcondition]~\\
     Es wird eine Fehlermeldung zurückgegeben.   
     \item[end] Charakteristische Daten validieren
\end{description}

~\\

} %end small

\subsubsection{Beispiel}

In diesem Anwendungsfall verfügen wir bereits über \glslink{item}{Teile}/Wertepaare eines bestimmten Konzeptes, z.B. eben jenen Schraubendreher.

\enquote{Ich habe hier ein mir bekanntes \glslink{item}{Teils} mit bestimmten Eigenschaften (Properties), Länge=300mm. Gib mir alle \glslink{item}{Teile} und alle Properties der Klasse mit dem Identifier 0173-1\#01-AAA352\#4 (Kreuzschraube) welche die mitgelieferten Eigenschaften haben.}
Das Ergebnis sind \glslink{item}{Teile} mit allen Properties des angegebenen Konzeptes, welche über die übergebenen Eigenschaften (Properties) verfügen. In unserem Fall vervollständigen wir unsere Properties mit den weiteren Properties \enquote{Typ} und \enquote{Spannungsfest}.

Die XML-Abfrage gemäß query.xsd\footnote{Schema Datei ist referenziert in ISO 29002-31, liegt der Arbeit bei} sieht so aus:

\begin{lstlisting}[caption=Query Beispiel - Daten validieren, language=XML, label=UseCaseDatenvalidieren]
<?xml version="1.0" encoding="UTF-8"?>
<qy:query xsi:schemaLocation="...query query.xsd" xmlns:xsi="http://www.w3.org/2001/XMLSchema-instance" xmlns:cat="...catalogue" xmlns:val="...value" xmlns:qy="...query" xmlns:bas="...basic">
	<cat:item class_ref="0173-1#01-AAA352#4..">
		<cat:property_value property_ref="0173-1#01-BBB111#1">
			<val:integer_value></val:integer_value>
		</cat:property_value>
	</cat:item>
</qy:query>
\end{lstlisting}

\subsection{Chrarakteristische Daten mittels Suchausdruck abfragen }

{\small

\begin{description}
     \item[use case] Charakteristische Daten mit Suchausdruck abfragen
     \item[  actors]~\\
     Klient
     \item[  precondition]~\\
     Der Klient verwendet einen gültigen Identifier.
     \item[  main flow]~\\
     Der Klient gibt einen Identifier eines Konzeptes ein. Ferner übergibt er ein oder mehrere bekannte Property Identifier sowie passend dazu Werte zur Sucheinschränkung. 
     \item[  postcondition]~\\
     Alle Elemente auf jene diese Einschränkung der übergebenen Werte zutrifft werden zurückgegeben.       
     \item[  alternative flow] Keine Werte zur Werteeinschränkung gefunden ~\\
     Die übermittelten Werte der mittels Property Identifier identifzierten Eigenschaftswerte sind nicht zum \glslink{item}{Teile} gespeichert.
      \item[  postcondition]~\\
     Es werden keine Daten zurückgeliefert.
     \item[  exceptional flow] Ungültige Identifier oder ungültige Anfrage ~\\
     Der oder die übergebenen Identifier oder die gesamte Anfrage ist gemäß Spezifikation ungültig.   
     \item[  postcondition]~\\
     Es wird eine Fehlermeldung zurückgegeben.      
     \item[end] Charakteristische Daten mit Suchausdruck abfragen
\end{description}

~\\

} %end small

\subsubsection{Beispiel}

Wir nehmen das Schraubendreher Beispiel aus \autoref{lab:schraubendreher} zur Hand, und möchten eine Abfrage absenden, welche von der Klasse Schraubendreher alle Items erhalten soll die eine Länge zwischen 200 und 300 mm haben. 

Um nun alle Eigenschaften (Properties), wie Länge, Typ und Spannungsfest zu erhalten muss folgende Abfrage gesendet werden: 
\enquote{Gib mir alle Items und alle Properties der Klasse mit dem Identifier 0173-1\#01-AAA352\#4 (Kreuzschraube).}
Das Ergebnis ist ein Item mit allen Attributen (Properties) der gewünschten Klassen und gegebenenfalls vorhandenen Unterklassen. In unserem Falle genau die oben angegebenen Werte.

Die XML-Abfrage gemäß query.xsd\footnote{Schema Datei ist referenziert in ISO 29002-31, liegt der Arbeit bei} sieht so aus:

\begin{lstlisting}[caption=Query Beispiel - Daten mit Suchausdruck abfragen, language=XML, label=lst:UseCaseDatenabfragenAnhang]
<?xml version="1.0" encoding="UTF-8"?>
<qy:query xsi:schemaLocation="...query query.xsd" xmlns:xsi="http://www.w3.org/2001/XMLSchema-instance" xmlns:cat="...catalogue" xmlns:val="...value" xmlns:qy="...query" xmlns:bas="...basic">
	<qy:class_ref>0173-1#01-AAA352#4</qy:class_ref>
	<qy:characteristic_data_query_expression>
		<qy:range>
			<qy:property_reference property_ref="0173-1#01-BBB111#1"/>
			<qy:min_value>200</qy:min_value>
			<qy:max_value>300</qy:max_value>
			<qy:is_inclusive>true</qy:is_inclusive>
		</qy:range>
	</qy:characteristic_data_query_expression>
</qy:query>
\end{lstlisting}

\chapter{Installationsanleitung Apache Tomcat 7}\label{kap:anhangtomcat}

\section{Installation auf Mac OS 10.8}

Diese Installationsanleitung bezieht sich auf die Installation des Apache Tomcat 7 in Mac OS 10.8. 

\subsection{Prüfen der Java Version}

Mittels 

\lstinline[basicstyle=\ttfamily\small\mdseries]{java -version}

prüfen, ob Version 1.6 oder 1.7 installiert ist. Falls nicht, muss Java vorher installiert werden. Dazu das Java Development Kit herunterladen und installieren: \\ 
\href{http://www.oracle.com/technetwork/java/javase/overview/index.html}{http://www.oracle.com/technetwork/java/javase/overview/index.html}

Für den Mac ist das ggf. über die Apple-Webseite verfügbar. 

\subsection{Tomcat herunterladen und entpacken}

\begin{itemize}
\item Auf \href{https://tomcat.apache.org/download-70.cgi}{https://tomcat.apache.org/download-70.cgi} Tomcat herunterladen. Darauf achten eine Binary distribution herunterzuladen, z.B. apache-tomcat-7.0.42.tar.gz
\item Das Paket entpacken mittels tar -xvzf apache-tomcat-7.0.42.tar.gz
\end{itemize}

\begin{itemize}
\item Ein Verzeichnis unter \href{file:///usr/local}{/usr/local} erstellen, wo der Apache später laufen soll, danach die Dateien dort hinkopieren
\begin{itemize}
	\item sudo mkdir -p \href{file:///usr/local}{/usr/local}
	\item sudo mv \href{file:///Users/stefan/Downloads/apache-tomcat-7.0.42}{~/Downloads/apache-tomcat-7.0.42} \href{file:///usr/local}{/usr/local/}
\end{itemize}\item Einen symbolischen Link erstellen um später einfacherer zwischen Versionen umzuschalten:
\begin{itemize}
	\item sudo rm -f \href{file:///Library/Tomcat}{/Library/Tomcat}
	\item sudo ln -s \href{file:///usr/local/apache-tomcat-7.0.42}{/usr/local/apache-tomcat-7.0.42} \href{file:///Library/Tomcat}{/Library/Tomcat}
\end{itemize}\item Allen Apache Dateien den aktuellen User als Besitzer setzen und Rechte vergeben:
\begin{itemize}
	\item sudo chown -R \textless{}dein\_username\textgreater{} \href{file:///Library/Tomcat}{/Library/Tomcat}
	\item sudo chmod +x \href{file:///Library/Tomcat/bin/*.sh}{/Library/Tomcat/bin/*.sh}
\end{itemize}\end{itemize}

\subsection{Tomcat starten}

\href{file:///Library/Tomcat/bin/startup.sh}{/Library/Tomcat/bin/startup.sh}

\subsection{Tomcat stoppen}

\href{file:///Library/Tomcat/bin/shutdown.sh}{/Library/Tomcat/bin/shutdown.sh}

\section{Installation unter Windows}

Am besten wird der Tomcat mittels Installer installiert. Hier wird man durch eine grafische Benutzeroberfläche geführt und Apache Tomcat wird als Dienst in das System integriert. 

\section{Tomcat mittels Maven starten}

Um Tomcat mittels Maven zu starten, muss Tomcat als Plugin in der Maven \gls{pom} konfiguriert werden. Das \autoref{lst:tomcat_plugin} zeigt die Konfiguration für die \gls{pom}.

 \begin{lstlisting}[caption=Tomcat 7 Maven Plugin, language=XML, label=lst:tomcat_plugin]
<build>
  <plugins>
    <plugin>
      <groupId>org.apache.tomcat.maven</groupId>
      <artifactId>tomcat7-maven-plugin</artifactId>
    </plugin>
  </plugins>
</build>
 \end{lstlisting}  
 
Anschließend kann der Tomcat-Server mit folgendem Befehl gestartet werden:
 
\lstinline[basicstyle=\ttfamily\small\mdseries]{mvn tomcat7:run}
 

 


\chapter{Automatische Entwicklertests} \index{JUnit}\label{anh:automatischeentwicklertests}
 
\section{Unit Test}\index{Unit Test}
Nachfolgend ein Beispiel eines Unit Tests. Das Testobjekt ist die Klasse XMLMarshaller, welcher für das \gls{Marshalling} und \gls{Unmarshalling} der XML-Daten verantwortlich ist. 
 
\begin{lstlisting}[caption=Beispiel eines Unit Tests, language=XML, label=lst:unittest_beispiel]
package de.feu.plib.xml;

import de.feu.plib.xml.catalogue.CatalogueType;
import de.feu.plib.xml.catalogue.ItemType;
import de.feu.plib.xml.catalogue.PropertyValueType;
import de.feu.plib.xml.query.QueryType;
import de.feu.plib.xml.value.BooleanValueType;
import org.apache.log4j.Logger;
import org.junit.After;
import org.junit.Before;
import org.junit.Test;

import static org.junit.Assert.assertEquals;
import static org.junit.Assert.assertTrue;

/**
 * Tests the marshalling and unmarshalling of xml files.
 */
public class XMLMarshallerTest extends AbstractXMLTest {

    /** XML Marshaller instance under test */
    XMLMarshaller marshaller;

    /** Logger instance */
    private static final Logger LOGGER = Logger.getLogger(XMLMarshallerTest.class);

    /**
     * Simple marshalling test with arbitrary catalogue item.
     */
    @Test
    public void testMarshallingWithValidArbitraryCatalogue() {
        String catalogue = marshaller.marshall(createCatalogueWithOneItem());
        LOGGER.info(catalogue);
        assertTrue(catalogue.contains("0173-1#01-AAA352#4"));
        assertTrue(catalogue.contains("true"));
    }

    /**
     * Simple test with unmarshalling an arbitrary item from xml.
     */
    @Test
    public void testUnMarshallingWithValidArbitraryClassIrdi() {
        QueryType queryType = marshaller.unmarshallXML(readXMLFrom("/de/feu/plib/xml/query_class_irdi.xml"),
                QueryType.class);
        assertEquals("0173-1#01-BAD803#2", queryType.getClassRef());
    }

    /**
     * Throws an exception while parsing on illegal irdi passed.
     *
     * @throws Exception on illegal irdi
     */
    @Test(expected = RuntimeException.class)
    public void shouldThrowExceptionDuringXMLValidationWithIllegalIrdi() throws Exception {
        QueryType queryType = marshaller.unmarshallXML(readXMLFrom("/de/feu/plib/xml/query_class_irdi_illegal.xml"),
                QueryType.class);
    }

    /**
     * creates a sample catalogue for testing
     * 
     * @return the sample catalogue with one item
     */
    private CatalogueType createCatalogueWithOneItem() {
        ItemType item = new ItemType();
        item.setClassRef("0173-1#01-AAA352#4");
        PropertyValueType propertyValueType = new PropertyValueType();

        BooleanValueType bvt = new BooleanValueType();
        bvt.setValue(true);
        propertyValueType.setBooleanValue(bvt);
        propertyValueType.setPropertyRef("0173-1#01-A35AA2#4");
        item.getPropertyValue().add(propertyValueType);
        CatalogueType catalogue = new CatalogueType();
        catalogue.getItem().add(item);

        return catalogue;
    }

    /**
     * @throws java.lang.Exception
     */
    @Before
    public void setUp() {
        marshaller = new XMLMarshallerImpl();
    }

    /**
     * @throws java.lang.Exception
     */
    @After
    public void tearDown() {
        marshaller = null;
    }

}
\end{lstlisting}  

Das \autoref{lst:abstrakte_unit_testklasse} zeigt die abstrakte Klasse \enquote{AbstractXMLTest}, welche Hilfsmethoden für XML-Tests kapselt. Hier kapselt diese Klasse die Funktionalität XML-Dateien einzulesen.  

\begin{lstlisting}[caption=Abstrakte Unit Testklasse, language=Java, label=lst:abstrakte_unit_testklasse]
package de.feu.plib.xml;

import org.apache.log4j.Logger;
import org.junit.After;
import org.junit.Before;

import java.io.BufferedReader;
import java.io.IOException;
import java.io.InputStream;
import java.io.InputStreamReader;

/**
 * Abstract test class. Extend from this class to get functionality to read XML files for your test.
 */
public class AbstractXMLTest {

    protected XMLMarshaller marshaller;

    /**
     * Logger instance
     */
    private static Logger LOGGER = Logger.getLogger(AbstractXMLTest.class);


    @Before
    public void setUp() {
        marshaller = new XMLMarshallerImpl();
    }

    @After
    public void tearDown() {
        marshaller = null;
    }

    /**
     * Reads the xml file from given filename
     *
     * @param filename the filename of the xml
     * @return the string content of the xml file
     */
    protected String readXMLFrom(String filename) {
        BufferedReader br = null;
        StringBuffer sb = new StringBuffer();

        try {
            String currentLine;

            InputStream is = XMLMarshallerImpl.class.getResourceAsStream(filename);
            br = new BufferedReader(new InputStreamReader(is));

            while ((currentLine = br.readLine()) != null) {
                sb.append(currentLine);
            }

        } catch (IOException e) {
            e.printStackTrace();
        } finally {
            try {
                if (br != null) br.close();
            } catch (IOException ex) {
                LOGGER.info("Exception occured during reading file: " + ex);
            }
        }

        return sb.toString();
    }
}
\end{lstlisting}  

\section{Integrationstest}\index{Integrationstest}

Dieser Abschnitt zeigt beispielhaft einen Integrationstest auf. Der Integrationstest unterscheidet sich vom Unit-Test darin, dass eine Komponente integriert mit mehreren abhängigen Komponenten getestet wird. 
Der Nachfolgende Integrationstestfall \enquote{PlibDaoIT} tested die Klasse \enquote{PlibDao}. Es ist ein Integrationstest, da für den Test die gesamte Datenbank benötigt wird. Nur so kann sinnvoll das Datenzugriffsobjekt getestet werden. 

\begin{lstlisting}[caption=Beispiel eines Integrationstests, language=Java, label=lst:integrationstest_beispiel]
package de.feu.plib.dao;

import de.feu.plib.processor.analyser.EnrichedQuery;
import de.feu.plib.processor.analyser.Irdi;
import de.feu.plib.processor.analyser.QueryKind;
import de.feu.plib.xml.catalogue.CatalogueType;
import de.feu.plib.xml.catalogue.ItemType;
import de.feu.plib.xml.query.QueryType;
import org.apache.log4j.Logger;
import org.junit.Ignore;
import org.junit.Test;
import org.junit.runner.RunWith;
import org.springframework.beans.factory.annotation.Autowired;
import org.springframework.test.context.ContextConfiguration;
import org.springframework.test.context.junit4.SpringJUnit4ClassRunner;

import java.math.BigDecimal;
import java.util.ArrayList;
import java.util.List;
import java.util.Map;
import java.util.Set;

import static org.junit.Assert.*;

/**
 * Integration Test of PLIB Dao
 */
@RunWith(SpringJUnit4ClassRunner.class)
@ContextConfiguration(locations = {"/beans_for_tests.xml"})
public class PlibDaoIT {

    /**
     * Logger instance
     */
    private static Logger LOGGER = Logger.getLogger(PlibDaoIT.class);

    @Autowired
    private PlibDao plib;

    @Test
    public void shouldReturnTrueWithExistingIRDI() {
        Irdi irdi = createMultiListTestIrdi();
        assertTrue(plib.doObjectsExistsWithThis(irdi));
    }

    @Test
    public void shouldReturnFalseWithIRDINotinDB() {
        Irdi irdi = createNonExistingTestIrdi();
        assertFalse(plib.doObjectsExistsWithThis(irdi));
    }

    @Test
    public void shouldReturnFalseWithEmptyIRDI() {
        Irdi irdi = new Irdi() {
            @Override
            public String getIrdi() {
                return "";
            }
        };
        assertFalse(plib.doObjectsExistsWithThis(irdi));
    }

    @Test
    public void testGetNumberOfObjectsOfIrdiExistingIrdi() {
        Irdi irdi = createMultiListTestIrdi();
        assertEquals(8, plib.getNumberOfObjectsOfIrdi(irdi));
    }

    @Test
    public void testGetNumberOfObjectsOfIrdiNotExistingIrdi() {
        Irdi irdi = createNonExistingTestIrdi();
        assertEquals(0, plib.getNumberOfObjectsOfIrdi(irdi));
    }

    @Test
    public void testThatWhenReadExternalProductIdsByIrdiMustReturnSome() throws Exception {
        Irdi irdi = createMultiListTestIrdi();
        List<BigDecimal> productIds = plib.readExternalProductIdsBy(irdi);
        assertNotNull(productIds);
        assertEquals(8, productIds.size());
    }

    /**
     * Currently there are two instances in the database, seems that these are duplicates but not sure.
     */
    @Test
    public void shouldReturnOneTestExternalIdWithTestIrdi() {
        Irdi irdi = createSkalpellIrdi();
        List<BigDecimal> externalProductIds = plib.readExternalProductIdsBy(irdi);
        assertNotNull(externalProductIds);
        assertEquals(2, externalProductIds.size());
        LOGGER.info("external product ids: " + externalProductIds.get(0));
        assertEquals(new BigDecimal("300000001"), externalProductIds.get(0));
    }

    /**
     * This is a bigger integration test.
     * <ul>
     *     <li>First create an irdi instance</li>
     *     <li>create an enriched query</li>
     *     <li>Then load the objects from the database with its properties</li>
     *     <li>There should be the same number ob objects than with the previous check</li>
     * </ul>
     */
    @Test
    @Ignore("loadObjectsFrom is obsolte, maybe later reimplemented")
    public void shouldReturnOneInstanceOfSkalpellWithTwoProperties() {
        Irdi skalpellIrdi = createSkalpellIrdi();

        List<BigDecimal> externalProductIds = plib.readExternalProductIdsBy(skalpellIrdi);

        EnrichedQuery query = createEnrichedQueryFrom(skalpellIrdi);
        query.setType(QueryKind.SIMPLE);
        CatalogueType catalogueType = plib.loadObjectsFrom(query);
        List<ItemType> itemTypes = catalogueType.getItem();
        assertEquals(externalProductIds.size(), itemTypes.size());
        assertEquals(1, itemTypes.get(0).getPropertyValue().size());
    }

    /**
     * Test should return two items where the first one would be checked.
     * Should have two properties.
     */
    @Test
    public void testLoadPropertiesByExternalIds() {
        List<BigDecimal> externalIds = new ArrayList<BigDecimal>();
        BigDecimal bigDecimal = new BigDecimal("300000001");

        externalIds.add(bigDecimal);
        List<List<Map<String, Object>>> valueTypeList = plib.loadStringPropertiesByExternalIds(externalIds);
        LOGGER.info("valuetype list: " + valueTypeList);
        assertEquals("should be one instance", 1, valueTypeList.size());
        assertEquals("should be two properties",2, valueTypeList.get(0).size());
        assertThatIrdiAndValueAreAvailable(valueTypeList.get(0).get(0).entrySet(), "0173-1#02-AAA762#1");
        assertThatIrdiAndValueAreAvailable(valueTypeList.get(0).get(1).entrySet(), "0173-1#02-AAB011#1");
    }

    private void assertThatIrdiAndValueAreAvailable(Set<Map.Entry<String, Object>> entrySet, String knownIRDI) {

        Set<Map.Entry<String, Object>> entries = entrySet;
        boolean irdiFound = false;
        boolean valueFound = false;
        for (Map.Entry<String, Object> entry : entries) {
            LOGGER.info("key of property: " + entry.getKey());
            LOGGER.info("value of property: " + entry.getValue());
            if ("IRDI".equals(entry.getKey()) && null != entry.getValue() && !"null".equals(entry.getValue())) {
                irdiFound = true;
                assertEquals(knownIRDI, entry.getValue());
            }
            if ("VALUE".equals(entry.getKey()) && null != entry.getValue() && !"null".equals(entry.getValue())) {
                valueFound = true;
            }

        }
        assertTrue(irdiFound && valueFound);
    }

    /**
     * Load data type and unit for skalpell length property which should be mm and mm as well.
     * Property id: 300903090000033914 and 300903090000034450
     */
    @Test
    public void testLoadDataTypeAndUnitForAPropertyById() {
        List<Map<String, Object>> propertyTypeAndUnit = plib.loadTypeAndUnitOfPropertyBy("300903090000033914");
        LOGGER.info("property size: " + propertyTypeAndUnit.size());
        assertTrue(propertyTypeAndUnit.size() == 1);

        assertThatUnitAndSubTypeAreAvailable(propertyTypeAndUnit);

    }

    private void assertThatUnitAndSubTypeAreAvailable(List<Map<String, Object>> propertyTypeAndUnit) {
        Set<Map.Entry<String,Object>> entries = propertyTypeAndUnit.get(0).entrySet();

        /*
        * we need to assure that we have the unit (in this case SYMBOL, e.g. mm or m or cm)
        * in the values as well as the type of the value (in this case real_measure_type)
        */
        String property_unit_symbol = "SYMBOL";
        String property_type = "SUB_TYPE";
        boolean unitfound = false;
        boolean typefound = false;
        for (Map.Entry<String, Object> entry : entries) {
            LOGGER.info("property key: " + entry.getKey());
            LOGGER.info("property value: " + entry.getValue());
            if ("SYMBOL".equals(entry.getKey()) && null != entry.getValue() && !"null".equals(entry.getValue())) {
                unitfound = true;
            }
            if ("SUB_TYPE".equals(entry.getKey()) && null != entry.getValue() && !"null".equals(entry.getValue())) {
                typefound = true;
            }
        }
        assertTrue(unitfound && typefound);
    }

    private EnrichedQuery createEnrichedQueryFrom(Irdi skalpellIrdi) {
        QueryType queryType = new QueryType();
        queryType.setClassRef(skalpellIrdi.getIrdi());
        return new EnrichedQuery(queryType);
    }

    private Irdi createNonExistingTestIrdi() {
        return new Irdi() {
            @Override
            public String getIrdi() {
                return "0141-1#01-xxx#1";
            }
        };
    }

    private Irdi createMultiListTestIrdi() {
        return new Irdi() {
            @Override
            public String getIrdi() {
                return "0141-1#01-UKU1#1";
            }
        };
    }

    /**
     * Creates the irdi of an item which was created by me as testdata.
     * It is a Skalpell (PREFERRED_NAME of class in DB)
     * @return the irdi of the test item skalpell.
     */
    private Irdi createSkalpellIrdi() {
        return new Irdi() {
            @Override
            public String getIrdi() {
                return "0173-1#01-BAD803#2";
            }
        };
    }
}

\end{lstlisting}  
   
\chapter{Testszenarios} \index{Testszenarios}\index{Manuelle Entwicklertests}\label{anh:testszenarios}

Anbei die Testszenarien und Testprotokolle der manuellen Entwicklertests.

\todotext{Testszenarien ausfüllen und für jeden Fall anlegen.}

\section{Vorhandene Teile anhand IRDI abfragen}

\begin{description}
\item[Vorbedingungen] 
  \begin{itemize}
   \item Die Datenbankverbindung aufbauen: Oracle Datenbank muss gestartet werden.
  \end{itemize}
\item[Eingabedaten] Testdatei query\_class\_irdi.xml. 
\item[Durchführung]
   \begin{itemize}
   \item XML-Datei wird mittels curl Befehl an den Server gesendet.
   \item curl -v -H "Content-Type: application/xml" -X POST --data "@query\_class\_irdi.xml" http://localhost:8080/rest/ws/query
  \end{itemize}
\item[Erwartetes Ergebnis] ...
\item[Tatsächliches Ergebnis] ...
\end{description}

\chapter{PLIB Datenbanktabellen - Ausschnitt}\label{kap:anh_plib_db}

\autoref{fig:plib_db_tabellen} zeigt einen Ausschnitt der Datenbanktabellen. Die Abbildung wurde von Herrn Karsten Mende zur Verfügung gestellt. 

\begin{figure}[htbp]
	\centering
		\includegraphics[height=0.98\textwidth, angle=90]{images/plib_datenbankausschnitt.jpg}
	\caption{PLIB Datenbanktabellen - Ausschnitt}
	\label{fig:plib_db_tabellen}
\end{figure}

% Index
\renewcommand{\indexname}{Index}
\addcontentsline{toc}{chapter}{Index}

%\renewcommand{\nomname}{Index}

\printindex

\end{document}

