\chapter{Use Cases}

Dieses Dokument beschreibt mögliche Use Case Scenarios die sich aus ISO 29002-31 und ISO 22745-30 ergeben. Es werden keine Use Case Beschreibungen erstellt, da mögliche Ausnahmefälle und alternative Flows zur aktuellen Zeit nicht bekannt sind. 

Es gibt Szenarios auf verschiedenen Benutzerebenen. 

\section{Manuelle Benutzerebene}
Diese Ebene beschreibt mögliche Abfrageszenarios eines Einzelbenutzers, gleichsam einer Person, die eine Anfrage über ein Formular einer Webseite schickt oder die gewünschten Informationen über einen anderen Kanal anfragt und geliefert bekommt. 

\begin{description}
\item[Charakteristische Daten abfragen/validieren] Der Klient gibt einen Identifier ein und bekommt ein oder mehrere Datensätze von Elementen mit charakteristischen Daten zurück. 
\item[Identifier abfragen] Der Klient übermittelt zu einem ihm bekannten Element die ihm bekannten zugehörigen charakteristischen Daten, und erhält den entsprechenden Identifier.
Kommentar: in welchem Gesamtszenario ist das sinnvoll? Soll hier genau nach Element, also Instanz einer Klasse gesucht werden oder nur die Klasse zurückgegeben werden?
\item[Abfrage mittels Suchausdruck] Der Klient übergibt einen bekannten Property Identifier sowie passend dazu Werte zur Sucheinschränkung um Elemente auf jene diese Einschränkung zutrifft zu erhalten. Dies ermöglicht eine Präzise Elementsuche. 
\item[query to supply missing characteristic data]
\end{description}

\section{Automatisierte Benutzerebene}
Der Unterschied zur manuellen Benutzerebene ist der, dass hierbei automatisiert Daten angefragt und übermittelt werden. Es findet keine Mensch zu Maschine Kommunikation statt sondern eine Maschine zu Maschine Kommunikation. 
Ziel der automatisierten Anfragen ist das Abgleichen oder Validieren von Massendaten eines (Teil)-Katalogs. 
