\chapter{Use Cases}\label{Use Cases}

Dieses Kapitel beschreibt mögliche Use Case Scenarios die sich aus ISO 29002-31 und ISO 22745-30 ergeben. Es werden keine Use Case Beschreibungen erstellt, da mögliche Ausnahmefälle und alternative Flows zur aktuellen Zeit nicht bekannt sind. 

Es gibt Szenarios auf verschiedenen Benutzerebenen. 

\section{Manuelle Benutzerebene}
Diese Ebene beschreibt mögliche Abfrageszenarios eines Einzelbenutzers, gleichsam einer Person, die eine Anfrage über ein Formular einer Webseite schickt oder die gewünschten Informationen über einen anderen Kanal anfragt und geliefert bekommt. 

\begin{description}
\item[Charakteristische Daten abfragen/validieren] Der Klient gibt einen Identifier (IRDI\footnote{International Registration Data Identifier}) eines Elementes ein und sendet eine Anfrage ab. Als Antwort bekommt er ein oder mehrere Datensätze von Elementen mit den entsprechendencharakteristischen Daten des Elementes mit dem übergebenen Identifier zurück. 
\item[Identifier abfragen] Der Klient übermittelt zu einem ihm bekannten Element die ihm bekannten zugehörigen charakteristischen Daten. Als Antwort erhält erhält er den entsprechenden Identifier des Elementes zu welchen die übermittelten charakteristischen Daten gehören.
\item[Abfrage mittels Suchausdruck] Der Klient übergibt einen bekannten Property Identifier sowie passend dazu Werte zur Sucheinschränkung. Als Antwort erhält er Elemente auf jene diese Einschränkung der übergebenen Werte zutrifft. 
\end{description}

\section{Automatisierte Benutzerebene}
Der Unterschied zur manuellen Benutzerebene ist der, dass hierbei automatisiert Daten angefragt und übermittelt werden. Es findet keine Mensch zu Maschine Kommunikation statt sondern eine Maschine zu Maschine Kommunikation. 
Ziel der automatisierten Anfragen ist das Abgleichen oder Validieren von Massendaten eines (Teil)-Katalogs. 

\begin{description}
\item[Alle Klassen abfragen] Der Klient sendet eine Anfrage und erhält alle vorhandene Klassen (ohne Items).
\item[Items einer Klasse abgleichen] Der Klient möchte seine Daten abgleichen und fragt alle Items einer Klasse ab.  
\item[Items einer Klasse validieren] Der Klient möchte seine Daten validieren und fragt alle Items einer Klasse ab.
\end{description}