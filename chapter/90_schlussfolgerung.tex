\chapter*{Schlussfolgerung und Ausblick in die Zukunft}

\addcontentsline{toc}{chapter}{Schlussfolgerung}

Die Zielsetzung dieser Arbeit ist, eine Lösung für die in der Regel unflexiblen und starren Produktdatenaustauschschnittellen in der Industrie zu erarbeiten. Hierbei wurde erfolgreich als Prototyp eine flexible \gls{Abfrageschnittstelle} analog zu SQL \gls{Abfrageschnittstelle} mit Hilfe der in der ISO 29002-31 beschriebenen \gls{Ontologie} erstellt. Ferner wurde diese Schnittstelle an die \gls{PLIB} als Datenbasis angebunden, sowie die bereitgestellten Prozeduren zum Datenaustausch genutzt. Es konnte festgestellt werden, dass mit aktuelle Techniken und Frameworks eine relativ schnelle Integration der den Standards zugehörigen Schemata möglich ist. Es lässt sich ein \gls{Webservice} erstellen, welcher die Flexibilität einer \gls{Abfrageschnittstelle} besitzt und auf Schemata basiert. Dabei wurde ein \gls{REST}ful \gls{Webservice} erstellt welcher die Abfrage entgegennimmt, für die Datenbankprozeduren transformiert und die Antwort entsprechend auf das Modell der Antwortnachricht transformiert. Sind diese Transformationslogiken implementiert, lassen sich darauf aufbauend weitere Schnittstellen implementieren wie in diesem Falle ein weiterer \gls{Webservice} mittels \gls{SOAP}-Protokoll. 

Während der Analyse und Implementierung ergaben sich einige Problemstellungen, die gelöst werden konnten. Beispielsweise ist die Heterogenität der Datenstrukturen ein nicht triviales Problem. Die Prozeduren konnten nicht direkt mit den von der Abfrage erhaltenen Daten aufgerufen werden. Es sind vorab Anfragen zur Transformation an die Datenbank nötig. 
Die zurückgelieferten Daten der Prozeduren liefern nicht alle Daten zurück, die für das Füllen der XML-Katalog Daten nötig sind. Ferner passt die Struktur und Semantik der zurückgelieferten Daten nicht mit den Antwort des Standards überein. Nicht zu verkennen sind auch technische Probleme, denn wird von der darüberliegenden Schicht ein Standard nicht unterstützt, in diesem Falle war es, dass RECORD-Typen nicht von Java-JDBC unterstützt werden, so gilt es auch dafür eine Lösung zu finden. 
Dies konnte durch entsprechende Absprache unter den Studenten des Fachbereiches gelöst werden, sodass an den Prozeduren und Rückgabetypen Änderungen durchgeführt wurden, welche das technische Problem lösen.

Die Umsetzung der Arbeit erfolgte mit Hilfe aktueller Technologien auf Basis von \gls{REST} \glspl{Webservice}. Die entwickelte Applikation kann automatisiert mit Hilfe eines Build Management Systems erzeugt werden. Der Vorteil ist das einfache und standardisierte Erzeugen der Applikation. Die Quelltexte sind in einem Versionskontrollsystem abgelegt, dies ermöglicht die genauen Entwicklungsschritte zu verfolgen und gibt die Möglichkeit auf ältere Versionsstände zu blicken oder diese zu erzeugen. 

Die Arbeit umfasst nicht die Implementierung einer GUI, es wurde dennoch zu Testzwecken und Simulation eine einfache Benutzeroberfläche geschaffen. Ebenfalls nicht Bestandteil der Arbeit ist die Implementierung einer Schnittstelle nach ISO 22745-30 Identification Guide. Dieser Standard beschreibt eine Möglichkeit, Regeln zu definieren, welche Konzepte (z.B aus \gls{PLIB}) näher beschreiben. Das ermöglicht sinnvollerweise eine Einschränkung der Anfragedaten in der Form, wie es der Kunde respektive Klient in seinem Kontext benötigt. Man stelle sich vor, dass in einer Produktionsabteilung andere Daten von bestimmten Konzepten nötig sind als in der Verkaufsabteilung. Ein Beispiel wäre bestimmte Eigenschaften des Materials über die Verformung bei großer Hitze in der Produktion. Bei einem Metall ist diese Information in der Produktion wichtig, da hier ggfs. hohe Hitzeentwicklung vorzufinden ist. Der Verkauf benötigt diese Informationen nicht oder nur gewisse Teile davon. Dies kann mit Hilfe von Identification Guides eingeschränkt werden. 
Somit wäre als Folgearbeit beispielsweise denkbar, die Implementierung von Identification Guides vorzunehmen und mit der in dieser Arbeit entwickelten Lösung zu verbinden. Um diese Einschränkung nach Regeln gemäß ISO 22745-30 zu demonstrieren, könnte eine Benutzeroberfläche aus den Angaben gemäß ISO 22745-30 generiert werden. Nehmen wir aus obigem Beispiel eine Applikation, welche eine Benutzeroberfläche für die Abteilung Produktion und eine welche die Abteilung Verkauf repräsentiert. Diese könnten nach ISO 22745-30 unterschiedliche definierte Regelwerke besitzen und folglich eine andere Maske darstellen. Die Abfrage die somit von der Abteilung Produktion über die Maske gesendet werden kann, ermöglicht gewisse Eigenschaften eines Produktes abzufragen die für die Abteilung relevant sind. Die Abfrage der Abteilung Verkauf kann folglich andere gleichsam zusätzliche oder ggfs. weniger Eigenschaften abfragen die nur für diese Abteilung relevant sind. Diese Abfragen werden dann eine Abfrage gemäß dieser Arbeit erzeugen, eingeschränkt gemäß des dynamisch erzeugtem Formular nach Identification Guide. 

Als mögliche weitere Folgearbeit wäre eine Gesamtintegration aller Arbeiten im Fachbereich denkbar. Diese Arbeit könnte die Arbeiten von Herrn Mende, Herrn Loth, Frau Janßen und Herrn Sobek zu einer der Praxis entsprechenden Gesamtlösung integrieren. Hierdurch erlangt man Erkenntnisse, wie flexibel die eingesetzten Standards sind.  
