\chapter*{Schlussfolgerung und Ausblick in die Zukunft}

\addcontentsline{toc}{chapter}{Schlussfolgerung}

Die Zielsetzung dieser Arbeit, die Machbarkeit und Integration der ISO-Schnittstelle aufbauend auf die vorhandene PLIB-Datenbankimplementierung aufzuzeigen, wurde erreicht. Aktuelle Techniken und Frameworks ermöglichen eine relativ schnelle Integration der den Standards zugehörigen Schemata. Es lässt sich mit überschaubarem Aufwand ein Web Service erstellen. 
Es wurde aufgezeigt, dass im Rahmen der Abschlussarbeiten des Fachbereiches die Abfrageschnittstelle aufbauend auf die PLIB-Datenbank des Fachbereiches möglich ist. Während der Analyse und Implementierung ergaben sich einige Problemstellungen, die gelöst werden konnten. Beispielsweise ist die Heterogenität der Datenstrukturen ein nicht triviales Problem. Die Prozeduren konnten nicht direkt mit den von der Abfrage erhaltenen Daten aufgerufen werden. Es sind vorab Anfragen zur Transformation an die Datenbank nötig. 
Die zurückgelieferten Daten der Prozeduren liefern nicht alle Daten zurück die für das Füllen der XML-Katalog Daten nötig sind. Ferner passt die Struktur und Semantik der zurückgelieferten Daten nicht mit den Antwort des Standards überein. 
Dies konnte durch entsprechende Absprache unter den Studenten des Fachbereiches gelöst werden konnte. An den Prozeduren wurden keine umfangreichenden Änderungen durchgeführt, jedoch ermöglichten die Absprachen ein Verständnis, welches einen Lösungsweg ermöglichten. 

Die Umsetzung der Arbeit erfolgte mit Hilfe aktueller Technologien auf Basis von REST Web Services. Die Applikation kann automatisiert mit Hilfe eines Build Management Systems einfach erzeugt werden. Das ermöglicht es, dass andere Studenten das System ohne Aufwand erzeugen können. Die Quelltexte sind in einem Versionskontrollsystem abgelegt und ermöglichen anderen Studenten und Entwicklern den Zugriff darauf. 

Die Arbeit umfasst nicht die Implementierung einer GUI und ebenfalls nicht die Implementierung der Schnittstelle nach ISO 22745-30 Identification Guide. Dieser Standard beschreibt eine Möglichkeit, Regeln zu definieren, welche Konzepte (z.B aus PLIB) näher beschreiben. Das ermöglicht sinnvollerweise eine Einschränkung der Anfragedaten in der Form, wie es der Kunde respektive Klient in seinem Kontext benötigt. Man stelle sich vor, dass in einer Produktionsabteilung andere Daten von bestimmten Konzepten nötig sind als in der Verkaufsabteilung. Ein Beispiel wären bestimmte Eigenschaften des Materials über die Verformung bei großer Hitze in der Produktion. Bei einem Metall ist diese Information in der Produktion wichtig, da hier ggfs. hohe Hitzeentwicklung vorzufinden ist. Der Verkauf benötigt diese Informationen nicht oder nur gewisse Teile davon. Dies kann mit Hilfe von Identification Guides eingeschränkt werden. 
Als Folgearbeit wäre beispielsweise denkbar, die Implementierung von Identification Guides vorzunehmen. Um diese Einschränkung nach Regeln gemäß ISO 22745-30 zu demonstrieren könnte eine Benutzeroberfläche aus den Angaben generiert werden. Nehmen wir aus obigem Beispiel eine Applikation welche eine Benutzeroberfläche für die Abteilung Produktion und eine welche die Abteilung Verkauf repräsentiert. Diese könnten nach ISO 22745-30 unterschiedliche definierte Regelwerke besitzen und folglich eine andere Maske darstellen. Die Abfrage die somit von der Abteilung Produktion über die Maske gesendet werden kann, kann nur gewisse Eigenschaften eines Produktes abfragen die für die Abteilung relevant sind. Die Abfrage der Abteilung Verkauf kann folglich andere Eigenschaften abfragen die nur für diese Abteilung relevant sind.

Als mögliche weitere Arbeit wäre eine Gesamtintegration aller Arbeiten im Fachbereich denkbar. Diese könnte die Arbeiten von Herrn Mende, Herrn Loth, Frau Janßen und Herrn Sobek integrieren.  

\todotext{Folgenarbeiten beschreiben,  22745-30 zu implementieren, näher beschreiben warum und was da getan werden könnte, z.B. generierung der GUI aus dem Identification Guide Angaben der Produkte. Sinn: Vorauswahl beim Kunden, GUI wird generiert, dann Erzeugung query.xml nach 29002-31, wie in dieser Arbeit usw. Möglicherweise Gesamtintegration eine weitere Arbeit (Hr. Mende, Hr. Loth, Frau Janßen, Hr. Sobek; so könnte ein Praxisbeispiel komplett gezeigt werden, möglicherweise anhand eines simplen konkretem Beispiels.}