\chapter{Testszenarios} \index{Testszenarios}\index{Manuelle Entwicklertests}\label{anh:testszenarios}

Anbei die Testprotokolle der manuellen Entwicklertests.

Alle Tests können alternativ mit der Testbenutzeroberfläche durchgeführt werden:

http://localhost:8080/plib-characteristic-query/query.xhtml

\section{Simple query}

\subsection{Vorhandene Teile anhand IRDI abfragen}

Es wird eine IRDI übermittelt.

\begin{description}
\item[Vorbedingungen] 
  \begin{itemize}
   \item Die Datenbankverbindung aufbauen: Oracle Datenbank muss gestartet werden.
  \end{itemize}
\item[Eingabedaten] Testdatei simple\_query\_irdi.xml. 
\item[Durchführung]
   \begin{itemize}
   \item XML-Datei wird mittels curl Befehl an den Server gesendet.
   \item curl -v -H 'Content-Type: application/xml' -X POST --data '@simple\_query\_irdi.xml' http://localhost:8080/plib-characteristic-query/rest/ws/query
  \end{itemize}
\item[Erwartetes Ergebnis] Zwei Items mit je zwei Eigenschaften werden erwartet. 
\item[Tatsächliches Ergebnis] Zwei Items mit zwei Eigenschaften.
\item[OK/Nicht OK?] OK
\end{description}


\subsection{Ungültige IRDI angegeben}

Es wird eine ungültige IRDI (gemäß XSD ungültig) übermittelt. 

\begin{description}
\item[Vorbedingungen] 
  \begin{itemize}
   \item Die Datenbankverbindung aufbauen: Oracle Datenbank muss gestartet werden.
  \end{itemize}
\item[Eingabedaten] Testdatei simple\_query\_illegal\_irdi.xml. 
\item[Durchführung]
   \begin{itemize}
   \item XML-Datei wird mittels curl Befehl an den Server gesendet.
   \item curl -v -H 'Content-Type: application/xml' -X POST --data '@simple\_query\_illegal\_irdi.xml' http://localhost:8080/plib-characteristic-query/rest/ws/query
  \end{itemize}
\item[Erwartetes Ergebnis] Fehlermeldung
\item[Tatsächliches Ergebnis] Fehlermeldung, marshalling error
\item[OK/Nicht OK?] OK
\end{description}

\subsection{Vorhandene Teile anhand IRDI abfragen mit Projektion}

Es wird eine IRDI übermittelt und zusätzlich eine Projektion (Eigenschaftsauswahl) vorgenommen.

\begin{description}
\item[Vorbedingungen] 
  \begin{itemize}
   \item Die Datenbankverbindung aufbauen: Oracle Datenbank muss gestartet werden.
  \end{itemize}
\item[Eingabedaten] Testdatei simple\_query\_projection\_one\_property.xml. 
\item[Durchführung]
   \begin{itemize}
   \item XML-Datei wird mittels curl Befehl an den Server gesendet.
   \item curl -v -H 'Content-Type: application/xml' -X POST --data '@simple\_query\_projection\_one\_property.xml' http://localhost:8080/plib-characteristic-query/rest/ws/query
  \end{itemize}
\item[Erwartetes Ergebnis] Zwei Items mit je einer Property, die angefragte Eigenschaft wird zurückgegeben. 
\item[Tatsächliches Ergebnis] Zwei Items mit einer Property. 
\item[OK/Nicht OK?] OK
\end{description}


\subsection{Vorhandene Teile mit bekannten Werten validieren}

Es wird eine IRDI übermittelt und zusätzlich Werte dieses Items übermittelt. Diese sollen validiert werden. Es wird dabei geprüft ob die Werte auch so in der Datenbank vorhanden. 

\begin{description}
\item[Vorbedingungen] 
  \begin{itemize}
   \item Die Datenbankverbindung aufbauen: Oracle Datenbank muss gestartet werden.
  \end{itemize}
\item[Eingabedaten] Testdatei simple\_query\_validation.xml. 
\item[Durchführung]
   \begin{itemize}
   \item XML-Datei wird mittels curl Befehl an den Server gesendet.
   \item curl -v -H 'Content-Type: application/xml' -X POST --data '@simple\_query\_validation.xml' http://localhost:8080/plib-characteristic-query/rest/ws/query
  \end{itemize}
\item[Erwartetes Ergebnis] Die Werte des Items stimmen überein, daher wird exakt das Item zurückerwartet.  
\item[Tatsächliches Ergebnis] Exakt das übergebene Item wird zurückgegeben, somit erfolgreich validiert. 
\item[OK/Nicht OK?] OK
\end{description}

\section{Parametric query}

\subsection{Abfrage mit Werteeinschränkung auf eine Eigenschaft eines Items}

Es wird eine IRDI übermittelt und zusätzlich eine Sucheinschränkung auf einen Eigenschaftswert dieses Teils vorgenommen. In diesem Falle sollen \gls{item}{Teile} gefunden werden, dessen Eigenschaftswerte sich im Bereich der übergebenen Werte befinden (Range). 

\begin{description}
\item[Vorbedingungen] 
  \begin{itemize}
   \item Die Datenbankverbindung aufbauen: Oracle Datenbank muss gestartet werden.
  \end{itemize}
\item[Eingabedaten] Testdatei parametric\_query\_range.xml. 
\item[Durchführung]
   \begin{itemize}
   \item XML-Datei wird mittels curl Befehl an den Server gesendet.
   \item curl -v -H 'Content-Type: application/xml' -X POST --data '@parametric\_query\_range.xml' http://localhost:8080/plib-characteristic-query/rest/ws/query
  \end{itemize}
\item[Erwartetes Ergebnis] Es wird nur ein Wert von zwei der eingeschränkten Eigenschaft zurückgegeben.  
\item[Tatsächliches Ergebnis] Nur ein Wert wurde zurückgegeben. 
\item[OK/Nicht OK?] OK
\end{description}

