\chapter*{Zusammenfassung}

\addcontentsline{toc}{chapter}{Zusammenfassung}

In der heutigen Zeit stehen Daten im absoluten Fokus. Schlagwörter wie \enquote{Big Data}, \enquote{Cloud Services} und \enquote{Mobile Devices} sind in aller Munde. Daten werden überall erfasst, seien es Multimedia-Daten wie Bilder, Ton oder Filme oder seien es Personendaten oder Produktdaten. Mobile Endgeräte wie z.B. ein Smartphone, allerdings ebenso Kraftfahrzeuge, erfassen laufend Daten wie die Position, Geschwindigkeit oder über Sensoren Werte wie Temperatur oder Luftdruck. Solche Daten werden gespeichert, und gegebenenfalls weiterverarbeitet und fallen in schier unfassbaren Mengen an. 

Oft spielt in diesen gesammelten Massendaten die Qualität der Daten eine untergeordnete Rolle, wobei in diesem Kontext mit Qualität eine möglichst präzise konzeptuelle Beschreibung der Daten gemeint ist.  
Betrachtet man beispielsweise Stammdaten in der Industrie, findet man eine große Menge an Daten, wie z.B. Daten zu Kunden, Lieferanten, Produkten, Materialien, Wirtschaftsgüter oder Angestellten. Die Qualität dieser Daten hat in der Industrie eine deutlich höhere Gewichtung. Die Unternehmen katalogisieren und beschreiben Ihre Daten, so dass diese innerhalb der Organisation zwischen einzelnen Systemen ausgetauscht werden können und definiert ist, was die einzelnen Datensätze bedeuten. Nehmen wir beispielsweise einen Hersteller von Sechkantschrauben zur Befestigung von Rädern an Fahrzeugen. Die Information über diese herzustellende Schraube muss definiert und in der Produktion bekannt sein. Der Verkauf benötigt allerdings alle diese Informationen für den Vertrieb des Produktes. Ferner möchte der Kunde der diese Schraube für die Produktion seines Fahrzeuges benötigt selbstverständlich auch diese Information. Zum einen für den Einkauf, zum anderen in der Planung und Fertigung, denn es muss bekannt sein aus welchen Einzelteilen (Stückliste) ein Produkt besteht. 

Es ist hier ersichtlich, dass Stammdaten in den verschiedenen Bereichen benötigt werden. Zu nennen seien beispielsweise die Lieferketten, Design und Herstellungprozesse als auch im Produkt Lebenszyklus Management.     

Probleme und somit eine hoher Kostenaufwand können auftreten, wenn die Beschreibung der Daten nicht vollständig ist, wenn gleichsam Informationen zu Stammdaten an einem Punkt der Lieferkette oder im Produktlebenszyklusmanagement fehlen oder ungenau sind. Daten müssen im schlimmsten Falle manuell gesichtet, manuell geprüft und angereichert werden.  

Diese Daten in Konzepte zu beschreiben, diese Konzepte verfügbar zu machen und diese Daten standardisiert und vor allem die automatisierte Austauschbarkeit zu schaffen ist das Ziel. 

Die Abschlussarbeiten des Fachbereiches rund um die PLIB (Parts Library) befassen sich mit ISO-Standards zu diesen Themen. Das Ziel dieser Arbeit ist es, die Implementierung der Schnittstelle nach \enquote{ISO 29002-31 - Query for characteristic data} zu erstellen, die eben genannte automatisierte Austauschbarkeit der konzeptuell Beschriebenen Daten ermöglicht. 

Die Voraussetzung für diese Untersuchungen und Implementierung sind Vorarbeiten xxx

Das Speichern, Verarbeiten, Abfragen und Übermitteln von Stammdaten sind ein wichtiger Teil heutiger Produktions- und Servicearbeiten in Firmen. Dabei stellt sich die Frage des \enquote{Wie} diese Daten gespeichert und ausgetauscht werden können häufig nicht oder viel zu spät im Entwicklungsprozess der Industrie. Dabei ist das  \enquote{Wie} die entscheidene Frage, auf die eine Lösung gefunden werden muss; denn ist diese Frage nicht beantwortet stößt jedes Unternehmen in der Industrie früher oder später auf Probleme. Heterogene Daten und Systemlandschaften sind oft die Folge. Weiter betrachtet wirkt sich das auf den gesamten Lifecycle eines Produktes oder einer Verarbeitung aus. Man denke nur daran, dass ausgehend vom Bestellprozess eines Klienten nicht genau klar definiert ist welches Produkt er bestellen möchte. Alternativ kann es mehrere Ausprägungen geben und pro Bestellung ist jeweils ein menschliches Eingreifen zur Klärung nötig. 

