\chapter*{Zusammenfassung}

\addcontentsline{toc}{chapter}{Zusammenfassung}

Mit Beginn des 21. Jahrhunderts und dem Einzug des Internets in nahezu jeden Haushalt, jedes Geschäft und in die Industrie stehen Daten im absoluten Fokus. Schlagwörter wie \enquote{Big Data}, \enquote{Cloud Services} und \enquote{Mobile Devices} sind in aller Munde. Daten werden überall erfasst, gespeichert, ausgewertet und verarbeitet, seien es Multimedia-Daten wie Bild, Ton oder Film oder Personendaten und Produktdaten. Mittlerweile erfassen mobile Endgeräte wie z.B. ein Smartphone, aber auch Kraftfahrzeuge oder Internet-Webseiten laufend Daten. Bei Fahrzeugen ist dies beispielsweise die Position, Geschwindigkeit oder über Sensoren Werte wie Temperatur oder Luftdruck. Internetseiten sammeln Zugriffsdaten und analysieren das Verhalten der Nutzer. Solche Daten werden gespeichert und gegebenenfalls weiterverarbeitet. Sie fallen in schier unfassbaren Mengen an. 

Oft spielt in diesen gesammelten Massendaten die Qualität der Daten eine untergeordnete Rolle, wobei in diesem Kontext mit Qualität eine möglichst präzise konzeptuelle Beschreibung der Daten gemeint ist. Solche Massendaten werden oft erst einmal gesammelt, um dann später aufwändig ausgewertet zu werden.
Betrachtet man beispielsweise Stammdaten in der Industrie, findet man z.B. Daten zu Kunden, Lieferanten, Produkten, Materialien, Wirtschaftsgütern oder Angestellten. Die Qualität dieser Daten hat in der Industrie eine deutlich höhere Gewichtung. Der Zugriff und die Weiterverarbeitung muss schnell erfolgen. Die Unternehmen katalogisieren und beschreiben ihre Daten, sodass diese innerhalb und außerhalb der Organisation zwischen einzelnen Systemen ausgetauscht werden können und definiert ist, was die einzelnen Datensätze bedeuten. Nehmen wir beispielsweise einen Hersteller von Sechskantschrauben zur Befestigung von Rädern an Fahrzeugen. Die Information über diese herzustellende Schraube muss definiert und in der Produktion bekannt sein. Der Verkauf benötigt allerdings alle diese Informationen für den Vertrieb des Produktes. Ferner möchte der Kunde, der diese Schraube für die Produktion seines Fahrzeuges benötigt, selbstverständlich auch diese Information, zum einen für den Einkauf, zum anderen für die Planung und Fertigung, denn es muss bekannt sein aus welchen Einzelteilen (Stückliste) ein Produkt besteht. Diese Daten müssen folglich zur Verfügung gestellt werden. 

Es ist ersichtlich, dass Stammdaten in den verschiedenen Bereichen benötigt werden. Zu nennen seien beispielsweise die Lieferketten, das Design und die Herstellungsprozesse als auch im \gls{PLM}.

Wenn die Beschreibung der Daten nicht vollständig ist, wenn gleichsam Informationen zu Stammdaten an einem Punkt der Lieferkette oder im \gls{PLM} fehlen oder ungenau sind, so führt das zu Problemen und hohem Kostenaufwand. 
Diese Daten in Konzepte zu beschreiben, diese Konzepte verfügbar zu machen und diese Daten automatisiert auszutauschen ist das Ziel. Der automatisierte Austausch von Produktdaten ist in der Praxis übliches Vorgehen und durch Prozesse abgebildet. Damit diese Daten automatisiert vom Sender zum Empfänger gelangen, müssen sie übertragen werden. Dazu muss ein genaues Modell der Daten definiert sein, z.B. als XML-Repräsentation. Herkömmlicherweise ist das Modell als Schema definiert, sodass der Sender und Empfänger den Aufbau der Daten kennen. Allerdings stellt sich hier das Problem, dass die Schnittstellen und Schemata oft starr und unflexibel sind. Anpassungen daran sind häufig mit hohem Einsatz und hohen Kosten verbunden. Jedoch sind Flexibilität und Änderbarkeit sehr wichtige Eigenschaften, um schnell und effizient auf Änderungen an den Anforderungen reagieren zu können. 

In den Abschlussarbeiten des Fachbereiches geht es um Problemstellungen rund um die \gls{PLIB} (Parts Library), die sich mit der Beschreibung eines Datenmodells für \glslink{Dictionary}{Dictionaries} und Bibliotheken befasst. Im Rahmen der Untersuchungen werden als mögliche Lösungen ISO-Standards zu diesen Themen analysiert und implementiert. 

Diese Arbeit handelt von der Problemstellung der automatisierten Austauschbarkeit von charakteristischen Produktdaten, das sind Eigenschaften, die ein Produkt möglichst präzise und eindeutig beschreiben, um einen bestimmten Zweck zu erfüllen. Dabei werden mögliche Lösungsoptionen für diese unflexiblen Austauschschnittstellen betrachtet. Eine Lösungsoption sind flexible \glslink{Abfrageschnittstelle}{Abfrage- und Antwortschnittstellen}. Hier kommt beispielsweise der ISO Standard \enquote{ISO 29002-31 - Query for characteristic data} als Beschreibung einer flexiblen \gls{Abfrageschnittstelle} in Frage. Nach einer Analyse des ISO-Standards wird darauf aufbauend eine Implementierung eines \gls{REST}ful \glspl{Webservice} vorgenommen. 
Eine Anfrage der \gls{Abfrageschnittstelle} gemäß \enquote{ISO 29002-31 - Query for characteristic data} referenziert mittels eindeutigem \glslink{IRDI}{Identifier} (IRDI) \glslink{Ontologie}{Ontologien}. Das Ergebnis einer Abfrage ist eine Antwort, welche die Eigenschaftsdaten der angefragten Produktdaten beinhaltet. Dazu ist ebenfalls eine Ontologie nötig, welche der Standard \enquote{ISO 29002-10 - Characteristic data exchange format} beschreibt. Hierin wird das Datenformat der zurückgelieferten Daten spezifiziert. Die Datenquelle für den Webservice ist die PLIB-Datenbank, eine Implementierung einer Produktdatenbank nach ISO 13584-42, welche die referenzierten Ontologiekonzepte liefert. 

Aufbauend auf die Implementierung der Anfrageverarbeitung lässt sich anschließend sehr einfach ein weiterer \gls{Webservice} basierend auf \gls{SOAP} implementieren. Da bereits ein Modell für die gewählte Programmiersprache im Rahmen der \gls{REST}ful \glspl{Webservice} Implementierung generiert wird, kann eben diese benutzt werden. Lediglich eine weitere Implementierung der Schnittstellentechnologie basierend auf einem Java Framework ist nötig.  

Aus dem Zusammenspiel mehrerer Standards lässt sich eine Implementierung sinnvoller Anwendungsfälle erstellen. 
Ein sinnvoller Anwendungsfall wäre beispielsweise die Anfrage nach allen Eigenschaften eines Produkts, welches durch einen weltweit eindeutigen \glslink{IRDI}{Identifier} identifiziert ist. Die Anfrage wird per XML nach ISO 29002-31 formuliert und an einen \gls{Webservice} gesendet und verarbeitet. Die Antwort wird als XML nach ISO 29002-10 formuliert und an den Kosumenten zurückgesendet. Komplexere Fälle schränken die Anfrage nach bestimmten Wertebereichen einer Eigenschaft ein oder selektieren nur bestimmte Eigenschaften, welche dann als Antwort zurückgesendet werden sollen. Dies entspricht etwa einer flexiblen \gls{Abfrageschnittstelle}, wie sie beispielsweise von SQL bekannt ist.  

Während der Analyse und Implementierung ergeben sich diverse Herausforderungen und Problemstellungen, welche es zu bewältigen gilt, seien es Heterogenität in der Datenrepräsentation der aufzurufenden Datenbankprozeduren, die Datenquelle, oder technische Problemstellungen wie z.B. bei der Generierung der Modellklassen zur Weiterverarbeitung der Anfrage im System oder fehlende Unterstützung von benutzten Typen aus der Oracle Datenbank in den eingesetzten Programmiersprachen und Frameworks.  




