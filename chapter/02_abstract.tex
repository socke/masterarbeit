\chapter*{Zusammenfassung}

\addcontentsline{toc}{chapter}{Zusammenfassung}

Mit Beginn des 21. Jahrhunderts und dem Einzug des Internets in nahezu jeden Haushalt, Geschäft und Industrie, stehen Daten im absoluten Fokus. Schlagwörter wie \enquote{Big Data}, \enquote{Cloud Services} und \enquote{Mobile Devices} sind in aller Munde. Daten werden überall erfasst, gespeichert, ausgewertet und verarbeitet, seien es Multimedia-Daten wie Bilder, Ton oder Filme oder seien es Personendaten oder Produktdaten. Mobile Endgeräte wie z.B. ein Smartphone, allerdings ebenso Kraftfahrzeuge, erfassen laufend Daten wie die Position, Geschwindigkeit oder über Sensoren Werte wie Temperatur oder Luftdruck. Solche Daten werden gespeichert, und gegebenenfalls weiterverarbeitet und fallen in schier unfassbaren Mengen an. 

Oft spielt in diesen gesammelten Massendaten die Qualität der Daten eine untergeordnete Rolle, wobei in diesem Kontext mit Qualität eine möglichst präzise konzeptuelle Beschreibung der Daten gemeint ist. Solche Massendaten werden oft erst ein Mal gesammelt um dann zeitlich später aufwändig ausgewertet zu werden.
Betrachtet man beispielsweise Stammdaten in der Industrie, findet man eine große Menge an Daten, wie z.B. Daten zu Kunden, Lieferanten, Produkten, Materialien, Wirtschaftsgüter oder Angestellte. Die Qualität dieser Daten hat in der Industrie eine deutlich höhere Gewichtung. Der Zugriff und die Weiterverarbeitung muss schnell erfolgen. Die Unternehmen katalogisieren und beschreiben Ihre Daten, so dass diese innerhalb der Organisation zwischen einzelnen Systemen ausgetauscht werden können und definiert ist, was die einzelnen Datensätze bedeuten. Nehmen wir beispielsweise einen Hersteller von Sechkantschrauben zur Befestigung von Rädern an Fahrzeugen. Die Information über diese herzustellende Schraube muss definiert und in der Produktion bekannt sein. Der Verkauf benötigt allerdings alle diese Informationen für den Vertrieb des Produktes. Ferner möchte der Kunde der diese Schraube für die Produktion seines Fahrzeuges benötigt selbstverständlich auch diese Information. Zum einen für den Einkauf, zum anderen in der Planung und Fertigung, denn es muss bekannt sein aus welchen Einzelteilen (Stückliste) ein Produkt besteht. 

Es ist hier ersichtlich, dass Stammdaten in den verschiedenen Bereichen benötigt werden. Zu nennen seien beispielsweise die Lieferketten, Design und Herstellungprozesse als auch im \gls{PLM}.     

Probleme und somit eine hoher Kostenaufwand können auftreten, wenn die Beschreibung der Daten nicht vollständig ist, wenn gleichsam Informationen zu Stammdaten an einem Punkt der Lieferkette oder im \gls{PLM} fehlen oder ungenau sind. Daten müssen im schlimmsten Falle manuell gesichtet, manuell geprüft und angereichert werden.  

Diese Daten in Konzepte zu beschreiben, diese Konzepte verfügbar zu machen und diese Daten standardisiert und vor allem die automatisierte Austauschbarkeit zu schaffen ist das Ziel. 

Ein weiteres Problem ist der automatisierte Austausch dieser Daten. Die Schnittstellen und Schemata sind oft starr und unflexibel. Anpassungen daran sind mit hohem Einsatz und Kosten verbunden. Allerdings ist heutzutage Flexibilität sehr wichtig, um schnell und effizient auf Änderungen reagieren zu können. 

Die aktuellen Abschlussarbeiten des Fachbereiches befassen sich mit den Problemstellungen rund um die \gls{PLIB} (Parts Library), die sich mit der Beschreibung eines Datenmodells für Dictionaries und Bibliotheken befasst. Ihm Rahmen der Untersuchungen werden als mögliche Lösungen ISO-Standards zu diesen Themen betrachten und implementiert. 

Diese Arbeit befasst sich mit dem Problem der automatisierten Austauschbarkeit von Charakteristischen Produktdaten basierend auf das PLIB Datenmodell. Zu diesem Zweck wird der ISO Standard \enquote{ISO 29002-31 - Query for characteristic data} als Beschreibung einer flexiblen Abfrage- und Antwortschnittstelle betrachtet und gemäß dieses Standards eine Implementierung eines \glspl{Webservice} vorgenommen. 
Für die Umsetzung der Abfrageschnittstelle wird der Standard \enquote{ISO 29002-31 - Query for characteristic data} benötigt. Dieser stellt die eigenliche Abfrageschnittstelle dar und referenziert mittels eindeutigem \glslink{IRDI}{Identifier} (IRDI) \glslink{Ontologie}{Ontologien}. Darüberhinaus ist das Ergebnis einer Abfrage eine Antwort, welche die Eigenschaftsdaten der angefragten Produktdaten beinhaltet. Dazu ist ebenfalls eine Ontologie nötig, welche der Standard \enquote{ISO 29002-10 - Characteristic data exchange format} beschreibt. Hierin wird das Datenformat der zurückgelieferten Daten spezifiziert. Somit lässt sich aus dem Zusammenspiel mehrerer Standards eine Implementierung sinnvoller Anwendungsfälle erstellen. 
Ein simpler sinnvoller Anwendungsfall wäre beispielsweise die Anfrage nach allen Eigenschaften eines Produkt, welches durch einen weltweit eindeutigen \glslink{IRDI}{Identifier} identifiziert ist. Die Anfrage wird per XML nach ISO 29002-31 formuliert und an einen \gls{Webservice} gesendet und verarbeitet. Die Antwort wird als XML nach ISO 29002-10 formuliert und an den Kosumenten zurückgesendet. Komplexere Fälle schränken die Anfrage nach bestimmten Wertebereichen einer Eigenschaft ein oder selektieren nur bestimmte Eigenschaften, welche dann als Antwort zurückgesendet werden sollen. Dies entspricht etwa einer flexiblen Abfrageschnittstelle, wie sie beispielsweise von SQL bekannt ist.  

Während der Analyse und Implementierung ergeben sich diverse Herausforderungen und Problemstellungen, welche es zu bewältigen gilt, seien es Heterogenität in der Datenrepräsentation der aufzurufenden Datenbankprozeduren des Fachbereiches, welche die Daten bereitstellen, oder seien es technische Problemstellungen wie z.B. bei der Generierung der Modellklassen zur Weiterverarbeitung der Anfrage im System oder fehlende Unterstützung von benutzen Typen aus der Oracle Datenbank in den eingesetzten Programmiersprachen und Frameworks.  




