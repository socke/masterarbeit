\documentclass[
a4paper,							% alle weiteren Papierformat einstellbar
%landscape,						% Querformat
12pt,						  		% Schriftgrˆfle (12pt, 11pt (Standard))
BCOR1cm,							% Bindekorrektur, bspw. 1 cm
DIV=calc,							% f¸hrt die Satzspiegelberechnung neu aus scrguide 2.4
twoside, 						% zweiseitig twoside/ einseitig oneside
%twocolumn,						% zweispaltiger Satz
%openany,							% Kapitel kˆnnen auch auf linken Seiten beginnen
openright,						% Kapitel beginnen auf der rechten Seite
parskip=half*,			   	% Absatzformatierung s. scrguide 3.1
headsepline,					% Trennline zum Seitenkopf	
footsepline,					% Trennline zum Seitenfufl
%notitlepage,					% in-page-Titel, keine eigene Titelseite
chapterprefix,				% vor Kapitel¸berschrift wird "Kapitel Nummer" gesetzt
appendixprefix,				% Anhang wird "Anhang" vor die ‹berschrift gesetzt 
%normalheadings,			% ‹berschriften etwas kleiner (smallheadings)
%smallheadings, 
%idxtotoc,						% Index im Inhaltsverzeichnis
%liststotoc,					% Abb.- und Tab.verzeichnis im Inhalt
%bibtotoc,					% Literaturverzeichnis im Inhalt
%leqno,						% Nummerierung von Gleichungen links
%fleqn,						% Ausgabe von Gleichungen linksb¸ndig
%draft,						% ¸berlangen Zeilen in Ausgabe gekennzeichnet
%pointlessnumbers,
%DIV13,                % Seitenrand ist 1/13 (Standard: 1/10 -> ziemlich breit)
%BCOR5mm								% Bindekorrektur, links 5mm mehr Platz f¸r Bindung
] {scrreprt}

% breitere seite
\usepackage{a4wide}

%% Normales LaTeX oder pdfLaTeX? %%%%%%%%%%%%%%%%%%%%%%%%%%%%
%% ==> Das neue if-Kommando "\ifpdf" wird an einigen wenigen
%% ==> Stellen benˆtigt, um die Kompatibilit‰t zwischen
%% ==> LaTeX und pdfLaTeX herzustellen.
\newif\ifpdf
\ifx\pdfoutput\undefined
	\pdffalse              %%normales LaTeX wird ausgef¸hrt
\else
	\pdfoutput=1           
	\pdftrue               %%pdfLaTeX wird ausgef¸hrt
\fi


%% Fonts f¸r pdfLaTeX %%%%%%%%%%%%%%%%%%%%%%%%%%%%%%%%%%%%%%%
%% ==> Nur notwendig, falls keine cm-super-Fonts installiert
%\ifpdf
	%\DeclareGraphicsExtensions{.pdf,.jpg,.png}
	%\usepackage{ae}       %%Benutzen Sie nur eines dieser Pakete:
	%\usepackage{zefonts}  %%je nachdem, welches Sie besitzen.
%\else
	%\DeclareGraphicsExtensions{.eps}
	%%Normales LaTeX - keine speziellen Fontpackages notwendig
%\fi

%% Deutsche Anpassungen %%%%%%%%%%%%%%%%%%%%%%%%%%%%%%%%%%%%%
\usepackage[ngerman]{babel}
\usepackage[T1]{fontenc}
%\usepackage[latin1]{inputenc}
\usepackage[utf8]{inputenc}


%% Packages f¸r Grafiken & Abbildungen %%%%%%%%%%%%%%%%%%%%%%
\ifpdf %%Einbindung von Grafiken mittels \includegraphics{datei}
	\usepackage[pdftex]{graphicx} %%Grafiken in pdfLaTeX
\else
	\usepackage[dvips]{graphicx} %%Grafiken und normales LaTeX
\fi
%\usepackage[hang,tight,raggedright]{subfigure} %%Teilabbildungen in einer Abbildung
%\usepackage{pst-all} %%PSTricks - nicht verwendbar mit pdfLaTeX
\let\ifpdf\relax

%% Packages f¸r Formeln %%%%%%%%%%%%%%%%%%%%%%%%%%%%%%%%%%%%%
\usepackage{amsmath}
\usepackage{amsthm}
\usepackage{amsfonts}

\usepackage{times}    % Schriftstil Times New Roman (wie Word)
%\usepackage{url}      % zur Dartellung von URLs mit Befehl \url{}
%\usepackage{xspace}   % Intelligenter Platzhalter nach Makros
\usepackage{booktabs} % schˆne Tabellen mit \toprule, \midrule und \bottomrule

%\usepackage{type1cm} % scalable Fonts
%\usepackage{courier} % Adobe Courier
%\usepackage{sectsty} % eigene Kapitel-Stile
% Control the fonts and formatting used in the table of contents.
%\usepackage[titles]{tocloft}

%\usepackage[Lenny]{fncychap} % Kapitel-Rahmen am Beginn jedes neuen Kapitels

%% Aesthetic spacing redefines that look nicer to me than the defaults.
%\setlength{\cftbeforechapskip}{2ex}
%\setlength{\cftbeforesecskip}{0.5ex}

%\newcommand{\autor}[1]{\textsc{#1}} % Makro f¸r Autor(en) im Fliefltext
%\newcommand{\degr}{\ensuremath{^\circ}} % Grad-Kreis
%\newcommand{\mum}{\ensuremath{\,\mu\textrm{m}}}
%\newcommand{\subcaption}[1]{\footnotesize\itshape #1}
%\newcommand{\matlab}{\emph{MATLAB}\xspace}

%% Use Helvetica-Narrow Bold for Chapter entries
%\renewcommand{\cftchapfont}{%
 % \fontsize{11}{13}\usefont{OT1}{phv}{bc}{n}\selectfont
%}

% \renewcommand{\baselinestretch}{1.5} % Zeilenabstand 1.5

% Verhindern von "`Schusterjungen"' und "`Hurenkindern"'
\clubpenalty = 10000
\widowpenalty = 10000
\displaywidowpenalty = 10000
\tolerance=500 %Zeilenumbruch


%% Zeilenabstand %%%%%%%%%%%%%%%%%%%%%%%%%%%%%%%%%%%%%%%%%%%%
\usepackage{setspace}
%\singlespacing        %% 1-zeilig (Standard)
\onehalfspacing       %% 1,5-zeilig
%\doublespacing        %% 2-zeilig
\usepackage{fancyhdr} %%Fancy Kopf- und Fuflzeilen
\usepackage{longtable} %%F¸r Tabellen, die eine Seite ¸berschreiten
\usepackage[babel,german=guillemets]{csquotes} % Franzˆsische Anf¸hrungszeichen  \enquote{}
% fuer Zitate
%\usepackage[round]{natbib}
\usepackage[numbers,round]{natbib}

% for listings
\usepackage{listings}

%\usepackage[automark]{scrpage2}

%workaround for lstlistoflistings
\makeatletter% --> De-TeX-FAQ
\renewcommand*{\lstlistoflistings}{%
  \begingroup 
    \if@twocolumn
      \@restonecoltrue\onecolumn
    \else
      \@restonecolfalse
    \fi
    \lol@heading
    \setlength{\parskip}{\z@}%
    \setlength{\parindent}{\z@}%
    \setlength{\parfillskip}{\z@ \@plus 1fil}%
    \@starttoc{lol}%
    \if@restonecol\twocolumn\fi
  \endgroup
}
\makeatother% --> \makeatletter

\usepackage[usenames]{color}

% for writing line numbers          
\usepackage[pagewise,mathlines,displaymath]{lineno}

\lstloadlanguages{PHP, XML, VBScript, Java, HTML}

%color definitions
\definecolor{mygray}{rgb}{0.2,0.2,0.2}
\definecolor{mydarkblue}{rgb}{0.2,0.2,0.9}
\definecolor{mydarkred}{rgb}{0.9,0.20,0.2}
\definecolor{mylightergray}{rgb}{0.9,0.9,0.9}

% listing settings
%\lstset{frame=single, numbers=left, numberstyle=\tiny, basicstyle=\footnotesize, stepnumber=1, numbersep=5pt, backgroundcolor=\color{MyGray}, breaklines=true}
\lstset{
        basicstyle=\ttfamily\scriptsize\mdseries,
        keywordstyle=\bfseries\color{mydarkblue},
        identifierstyle=,
        commentstyle=\color{mygray},      
        stringstyle=\itshape\color{mydarkred},
        numbers=left,
        numberstyle=\tiny,
        stepnumber=1,
        breaklines=true,
        frame=none,
        showstringspaces=false,
        tabsize=4,
        backgroundcolor=\color{mylightergray},
        captionpos=b,
        float=htbp,
} 

% f¸r tabellen
\usepackage{array}
% f¸r lange tabellen
\usepackage{longtable} 

\addtokomafont{caption}{\normalsize}

% paket f¸r farbige tabellen
\usepackage{colortbl}

\usepackage[pdftex,colorlinks=false,
                      pdfstartview=FitV,
                      linkcolor=blue,
                      citecolor=blue,
                      urlcolor=blue,
          ]{hyperref}
          \pdfinfo{
            /Title      (Implementierung eines Web Services nach ISO 29002-31 - Query for characteristic data)
            /Author     (Stefan Sobek)
            /Keywords   (FernUni Hagen, Masterarbeit, Stefan Sobek)
          }

\hypersetup{
    pdftitle={Masterarbeit - Implementierung eines Web Services nach ISO 29002-31 - Query for characteristic data},
    pdfauthor={Stefan Sobek},
    pdfkeywords={FernUni Hagen, ISO 29002-31, characteristic product data, 2013}
}

%Darstellung des Glossars einstellen
%\usepackage[style=long,toc]{glossaries}
\usepackage[
nonumberlist, %keine Seitenzahlen anzeigen
acronym,      %ein Abkürzungsverzeichnis erstellen
toc,          %Einträge im Inhaltsverzeichnis
%section %im Inhaltsverzeichnis auf section-Ebene erscheinen
]      
{glossaries}

%glossar befehle einschalten
\makeglossaries

%index
\usepackage{makeidx}

\makeindex

