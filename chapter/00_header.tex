\documentclass[
paper=a4,							% alle weiteren Papierformat einstellbar
%landscape,						% Querformat
fontsize=12pt,						  		% Schriftgrˆfle (12pt, 11pt (Standard))
BCOR=20mm,							% Bindekorrektur, bspw. 1 cm
DIV=13,							% f¸hrt die Satzspiegelberechnung neu aus scrguide 2.4
twoside=true, 						% zweiseitig twoside/ einseitig oneside
%twocolumn,						% zweispaltiger Satz
%openany,							% Kapitel kˆnnen auch auf linken Seiten beginnen
open=right,						% Kapitel beginnen auf der rechten Seite
%parskip=half*,				% Absatzformatierung s. scrguide 3.1
parskip=half+,			   	
headsepline,					% Trennline zum Seitenkopf	
footsepline,					% Trennline zum Seitenfufl
%notitlepage,					% in-page-Titel, keine eigene Titelseite
chapterprefix=false,				% vor Kapitel¸berschrift wird "Kapitel Nummer" gesetzt
appendixprefix,				% Anhang wird "Anhang" vor die ‹berschrift gesetzt 
%normalheadings,			% ‹berschriften etwas kleiner (smallheadings)
%smallheadings, 
%idxtotoc,						% Index im Inhaltsverzeichnis
%liststotoc,					% Abb.- und Tab.verzeichnis im Inhalt
%bibtotoc,					% Literaturverzeichnis im Inhalt
%leqno,						% Nummerierung von Gleichungen links
%fleqn,						% Ausgabe von Gleichungen linksb¸ndig
%draft,						% ¸berlangen Zeilen in Ausgabe gekennzeichnet
%pointlessnumbers,
] {scrreprt}

% breitere seite
%\usepackage{a4wide}

%% Normales LaTeX oder pdfLaTeX? %%%%%%%%%%%%%%%%%%%%%%%%%%%%
%% ==> Das neue if-Kommando "\ifpdf" wird an einigen wenigen
%% ==> Stellen benˆtigt, um die Kompatibilit‰t zwischen
%% ==> LaTeX und pdfLaTeX herzustellen.
\newif\ifpdf
\ifx\pdfoutput\undefined
	\pdffalse              %%normales LaTeX wird ausgef¸hrt
\else
	\pdfoutput=1           
	\pdftrue               %%pdfLaTeX wird ausgef¸hrt
\fi

%% Fonts f¸r pdfLaTeX %%%%%%%%%%%%%%%%%%%%%%%%%%%%%%%%%%%%%%%
%% ==> Nur notwendig, falls keine cm-super-Fonts installiert
%\ifpdf
	%\DeclareGraphicsExtensions{.pdf,.jpg,.png}
	%\usepackage{ae}       %%Benutzen Sie nur eines dieser Pakete:
	%\usepackage{zefonts}  %%je nachdem, welches Sie besitzen.
%\else
	%\DeclareGraphicsExtensions{.eps}
	%%Normales LaTeX - keine speziellen Fontpackages notwendig
%\fi

%% Deutsche Anpassungen %%%%%%%%%%%%%%%%%%%%%%%%%%%%%%%%%%%%%
\usepackage[ngerman]{babel}
\usepackage[T1]{fontenc}
%\usepackage[latin1]{inputenc}
\usepackage[utf8]{inputenc}


%% Packages f¸r Grafiken & Abbildungen %%%%%%%%%%%%%%%%%%%%%%
\ifpdf %%Einbindung von Grafiken mittels \includegraphics{datei}
	\usepackage[pdftex]{graphicx} %%Grafiken in pdfLaTeX
\else
	\usepackage[dvips]{graphicx} %%Grafiken und normales LaTeX
\fi
%\usepackage[hang,tight,raggedright]{subfigure} %%Teilabbildungen in einer Abbildung
%\usepackage{pst-all} %%PSTricks - nicht verwendbar mit pdfLaTeX
\let\ifpdf\relax

%% Packages f¸r Formeln %%%%%%%%%%%%%%%%%%%%%%%%%%%%%%%%%%%%%
\usepackage{amsmath}
\usepackage{amsthm}
\usepackage{amsfonts}

%\usepackage{ccfonts} % Schrift  concrete
%\usepackage{chancery} % Schrift  Zapf Chancery -> sehr schnörkelig
%\usepackage{bookman} % Schrift  bookman -> sehr schöne Schrift!
\usepackage{mathpazo} % Schrift  Palatino -> sehr schöne Schrift!
\linespread{1.05}  
%\usepackage{newcent} % New Century Schoolbook -> sehr schön, kräftig!
%\usepackage{charter} % Schrift  Charter -> 
%\usepackage{times}    % Schriftstil Times New Roman (wie Word)
\usepackage{url}      % zur Dartellung von URLs mit Befehl \url{}
%\usepackage{xspace}   % Intelligenter Platzhalter nach Makros
\usepackage{multirow} 
\usepackage{tabularx} 
\usepackage{booktabs} % schˆne Tabellen mit \toprule, \midrule und \bottomrule


% Bildunterschrift

%\usepackage{type1cm} % scalable Fonts
%\usepackage{courier} % Adobe Courier
%\usepackage{sectsty} % eigene Kapitel-Stile
% Control the fonts and formatting used in the table of contents.
%\usepackage[titles]{tocloft}

%\usepackage[Lenny]{fncychap} % Kapitel-Rahmen am Beginn jedes neuen Kapitels

%% Aesthetic spacing redefines that look nicer to me than the defaults.
%\setlength{\cftbeforechapskip}{2ex}
%\setlength{\cftbeforesecskip}{0.5ex}

%\newcommand{\autor}[1]{\textsc{#1}} % Makro f¸r Autor(en) im Fliefltext
%\newcommand{\degr}{\ensuremath{^\circ}} % Grad-Kreis
%\newcommand{\mum}{\ensuremath{\,\mu\textrm{m}}}
%\newcommand{\subcaption}[1]{\footnotesize\itshape #1}
%\newcommand{\matlab}{\emph{MATLAB}\xspace}

%% Use Helvetica-Narrow Bold for Chapter entries
%\renewcommand{\cftchapfont}{%
%  \fontsize{11}{13}\usefont{OT1}{phv}{bc}{n}\selectfont
%}

% \renewcommand{\baselinestretch}{1.5} % Zeilenabstand 1.5

% Verhindern von "`Schusterjungen"' und "`Hurenkindern"'
\clubpenalty = 10000
\widowpenalty = 10000
\displaywidowpenalty = 10000
\tolerance=500 %Zeilenumbruch


%% Zeilenabstand %%%%%%%%%%%%%%%%%%%%%%%%%%%%%%%%%%%%%%%%%%%%
\usepackage{setspace}
\singlespacing        %% 1-zeilig (Standard)
%\onehalfspacing       %% 1,5-zeilig
%\doublespacing        %% 2-zeilig
\usepackage{fancyhdr} %%Fancy Kopf- und Fuflzeilen
\usepackage{longtable} %%F¸r Tabellen, die eine Seite ¸berschreiten
\usepackage[babel,german=guillemets]{csquotes} % Franzˆsische Anf¸hrungszeichen  \enquote{}
% fuer Zitate
%\usepackage[round]{natbib}
\usepackage[numbers,round]{natbib}

% for listings
\usepackage{listings}

% Schriften-Größen
\setkomafont{chapter}{\Huge\rmfamily} % Überschrift der Ebene
\setkomafont{section}{\Large\rmfamily}
\setkomafont{subsection}{\large\rmfamily}
\setkomafont{subsubsection}{\large\rmfamily}
\setkomafont{chapterentry}{\large\rmfamily} % Überschrift der Ebene in Inhaltsverzeichnis
\setkomafont{descriptionlabel}{\bfseries\rmfamily} % für description Umgebungen
\setkomafont{captionlabel}{\small\bfseries}
\setkomafont{caption}{\small}

%\B

%workaround for lstlistoflistings
%\makeatletter% --> De-TeX-FAQ
%\renewcommand*{\lstlistoflistings}{%
%  \begingroup 
 %   \if@twocolumn
 %     \@restonecoltrue\onecolumn
 %   \else
 %     \@restonecolfalse
 %   \fi
 %   \lol@heading
 %   \setlength{\parskip}{\z@}%
 %   \setlength{\parindent}{\z@}%
 %   \setlength{\parfillskip}{\z@ \@plus 1fil}%
 %   \@starttoc{lol}%
 %   \if@restonecol\twocolumn\fi
 % \endgroup
%}
%\makeatother% --> \makeatletter

\usepackage[usenames]{color}

%cmyk for coloring 
\usepackage[cmyk]{xcolor}

% for writing line numbers          
\usepackage[pagewise,mathlines]{lineno}

\lstloadlanguages{PHP, XML, VBScript, Java, HTML,SQL,sh}

%color definitions
\definecolor{mygray}{cmyk}{0.9,0.9,0.9,0,9}
\definecolor{mydarkblue}{cmyk}{1,0.7,0,0.16}
\definecolor{mydarkred}{cmyk}{0.1,0.75,0.75,0.33}
\definecolor{mylightergray}{cmyk}{0.02,0.02,0.02,0.02}
\definecolor{myyellow}{cmyk}{0.0, 0.05, 1.0, 0.24} 
\definecolor{mydarkgreen}{cmyk}{1.0, 0.0, 1.63, 0.67} 

\definecolor{hellgelb}{rgb}{1,1,0.9}
\definecolor{colKeys}{rgb}{0,0,1}
\definecolor{colIdentifier}{rgb}{0,0,0}
\definecolor{colComments}{rgb}{0,0.5,0}
\definecolor{colString}{rgb}{1,0,0}

% listing settings
%\lstset{frame=single, numbers=left, numberstyle=\tiny, basicstyle=\footnotesize, stepnumber=1, numbersep=5pt, %backgroundcolor=\color{mygray}, breaklines=true}
\lstset{
       basicstyle=\ttfamily\scriptsize\mdseries,
        %keywordstyle=\bfseries\color{mydarkblue},
        keywordstyle=\color{colKeys}, 
        identifierstyle=\color{colIdentifier},
        %commentstyle=\color{mydarkgreen},   
        commentstyle=\color{colComments},
        stringstyle=\color{colString},   
        %stringstyle=\itshape\color{mydarkred},
        numbers=left,
        numberstyle=\tiny,
        stepnumber=1,
        breaklines=true,
        frame=single,
        showstringspaces=false,
        tabsize=2,
        %backgroundcolor=\color{mylightergray},
        backgroundcolor=\color{hellgelb},
        captionpos=b,
        float=htbp,
        breakautoindent=true
} 

% Zum Einbinden von Programmcode --------------------------------------------

%\lstset{%
% float=hbp,%
%    basicstyle=\texttt\small, %
%    identifierstyle=\color{colIdentifier}, %
%    keywordstyle=\color{colKeys}, %
%    stringstyle=\color{colString}, %
%    commentstyle=\color{colComments}, %
%    columns=flexible, %
%    tabsize=2, %
%    frame=single, %
%    extendedchars=true, %
%    showspaces=false, %
%    showstringspaces=false, %
%    numbers=left, %
%    numberstyle=\tiny, %
%    breaklines=true, %
%    backgroundcolor=\color{hellgelb}, %
%    breakautoindent=true, %
%    captionpos=b%
%}

% f¸r tabellen
\usepackage{array}
% f¸r lange tabellen
\usepackage{longtable} 

\addtokomafont{caption}{\normalsize}

% paket f¸r farbige tabellen
\usepackage{colortbl}

% Einstellungen für links, unter anderem farbige links
\usepackage[pdftex,colorlinks=false,
                      pdfstartview=FitV,
                      linkcolor=blue,
                      citecolor=blue,
                      urlcolor=blue,
          ]{hyperref}
          \pdfinfo{
            /Title      (Implementierung eines Web Services nach ISO 29002-31 - Query for characteristic data)
            /Author     (Stefan Sobek)
            /Keywords   (FernUni Hagen, Masterarbeit, Stefan Sobek)
          }

\hypersetup{
    pdftitle={Masterarbeit - Implementierung eines Web Services nach ISO 29002-31 - Query for characteristic data},
    pdfauthor={Stefan Sobek},
    pdfkeywords={FernUni Hagen, ISO 29002-31, characteristic product data, 2014}
}

% dictum kapitel zitatsbreite vergrößern
\renewcommand*{\dictumwidth}{.6667\textwidth}

%Darstellung des Glossars einstellen
%\usepackage[style=long,toc]{glossaries}
\usepackage[
%nonumberlist, %keine Seitenzahlen anzeigen
acronym,      %ein Abkürzungsverzeichnis erstellen
toc,          %Einträge im Inhaltsverzeichnis
%section %im Inhaltsverzeichnis auf section-Ebene erscheinen
]      
{glossaries}

%glossar befehle einschalten
\makeglossaries

%index
\usepackage{makeidx}

\usepackage[intoc]{nomencl}

\makeindex


\DeclareGraphicsExtensions{.pdf,.png,.jpg} % bevorzuge pdf-Dateien
\usepackage{subfigure} % mehrere Abbildungen nebeneinander/übereinander
\newcommand{\subfigureautorefname}{\figurename} % um \autoref auch für subfigures benutzen
\usepackage[all]{hypcap} % Beim Klicken auf Links zum Bild und nicht zu Caption gehen
% Bildunterschrift
\setcapindent{0em} % kein Einrücken der Caption von Figures und Tabellen
\setcapwidth[c]{0.9\textwidth}
\setlength{\abovecaptionskip}{0.2cm} % Abstand der zwischen Bild- und Bildunterschrift

% Eigene Befehle %%%%%%%%%%%%%%%%%%%%%%%%%%%%%%%%%%%%%%%%%%%%%%%%%5
% Matrix
\newcommand{\mat}[1]{
      {\textbf{#1}}
}
\newcommand{\todo}[1]{
      {\colorbox{red}{ TODO: #1 }}
}
\newcommand{\todotext}[1]{
      {\color{red} TODO: #1} \normalfont
}
\newcommand{\info}[1]{
      {\colorbox{blue}{ (INFO: #1)}}
}
% Hinweis auf Programme in Datei
\newcommand{\datei}[1]{
      {\ttfamily{#1}}
}
\newcommand{\code}[1]{
      {\ttfamily{#1}}
}
% bild mit defnierter Breite einfügen
\newcommand{\bild}[4]{
  \begin{figure}[!hbt]
    \centering
      \vspace{1ex}
      \includegraphics[width=#2]{images/#1}
      \caption[#4]{\label{fig:#1} #3}
    \vspace{1ex}
  \end{figure}
}
% bild mit eigener Breite
\newcommand{\bilda}[3]{
  \begin{figure}[!hbt]
    \centering
      \vspace{1ex}
      \includegraphics{images/#1}
      \caption[#3]{\label{fig:#1} #2}
      \vspace{1ex}
  \end{figure}
}


% Bild todo
\newcommand{\bildt}[2]{
  \begin{figure}[!hbt]
    \begin{center}
      \vspace{2ex}
	      \includegraphics[width=6cm]{images/todo}
      %\caption{\label{#1} \color{red}{ TODO: #2}}
      \caption{\label{#1} \todotext{#2}}
      %{\caption{\label{#1} {\todo{#2}}}}
      \vspace{2ex}
    \end{center}
  \end{figure}
}

