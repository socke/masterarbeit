\chapter*{Abstract}

\addcontentsline{toc}{chapter}{Abstract}

Handling, collection and processing data is nowadays in the daily focus. Buzzwords like \enquote{Big Data}, \enquote{Cloud Services} und \enquote{Mobile Devices} are broadcasted through the internet and in the media. Data will be collected in nearly every area like Multimedia-data via pictures, audio or video, or personal data and product data. This data will be collected by Mobile devices as well as cars or trucks. The collect data in type of position, current speed oder simply via sensors the temperature or barometric pressure. Such data will be saved and further processed and the data volume is enormous. 

When collecting such kind of massive Data, the quality of the data is not always very important. In this context with the term \enquote{quality} the precise conceptual description of data is meant. Considering master data in the industry, manufacturing and trade we find data entities like customer, supplier, material, economic goods or simply personal data regarding employees. The quality of the data has a pretty high importance here. Companies create catalogues and descriptions of their data so that the information can be exchanged between several systems. To precisely know what the datasets mean in their context is very important. A manufacturer for hexagon bolts, which are used to mount tires on cars for example needs exact information, description and definition of this particular product for its production division. Furthermore Sales needs the information as well to sell the product to interested customers and provide them with all information about the product. Last but not least, the customer needs precise information about the product, in his design and production department, as well for the parts list which defines the needed parts for his concrete product. 

Master data is needed in nearly every area, in supply chain, design and production as well as product lifecycle management. 

If the description of the data is not complete or not precise there will be problems in the product life cycle management. In worst case the data must be reviewed or enriched manually where normally an automated process would do the job. This is time consuming and leads to higher production costs.   

To describe data in concepts, make those concepts available and create standards which allow to automate the data flow the main goal.  

The problem with automated product data transfer is that interfaces and schemas are fixed and unflexible. Modifications and updates on interfaces and schemas are time consuming and costly. However flexibility is pretty important nowadays as we must react fast and efficiently on changes in specifications. 
 
 \todotext{weiter übersetzen}
 
Die aktuellen Abschlussarbeiten des Fachbereiches befassen sich mit den Problemstellungen rund um die \gls{PLIB} (Parts Library), die sich mit der Beschreibung eines Datenmodells für Dictionaries und Bibliotheken befasst. Ihm Rahmen der Untersuchungen werden als mögliche Lösungen ISO-Standards zu diesen Themen betrachten und implementiert. 

Diese Arbeit befasst sich mit dem Problem der automatisierten Austauschbarkeit von Charakteristischen Produktdaten basierend auf das PLIB Datenmodell. Zu diesem Zweck wird der ISO Standard \enquote{ISO 29002-31 - Query for characteristic data} als Beschreibung einer flexiblen Abfrage- und Antwortschnittstelle betrachtet und gemäß dieses Standards eine Implementierung eines \glspl{Webservice} vorgenommen. 
Für die Umsetzung der Abfrageschnittstelle wird der Standards \enquote{ISO 29002-31 - Query for characteristic data} benötigt. Dieser stellt die eigenliche Abfrageschnittstelle dar und referenziert mittels eindeutigem \glslink{IRDI}{Identifier} (IRDI) \glslink{Ontologie}{Ontologien}. Darüberhinaus ist das Ergebnis einer Abfrage eine Antwort, welche die Eigenschaftsdaten der angefragten Produktdaten beinhaltet. Dazu ist ebenfalls eine Ontologie nötig, welche der Standard \enquote{ISO 29002-10 - Characteristic data exchange format} beschreibt. Hierin wird das Datenformat der zurückgelieferten Daten spezifiziert. Somit lässt sich aus dem Zusammenspiel mehrerer Standards eine Implementierung sinnvoller Anwendungsfälle erstellen. 
Ein simpler sinnvoller Anwendungsfall wäre beispielsweise die Anfrage nach allen Eigenschaften eines Produkt, welches durch einen weltweit eindeutigen \glslink{IRDI}{Identifier} identifiziert ist. Die Anfrage wird per XML nach ISO 29002-31 formuliert und an einen \gls{Webservice} gesendet und verarbeitet. Die Antwort wird als XML nach ISO 29002-10 formuliert und an den Kosumenten zurückgesendet. Komplexere Fälle schränken die Anfrage nach bestimmten Wertebereichen einer Eigenschaft ein oder selektieren nur bestimmte Eigenschaften, welche dann als Antwort zurückgesendet werden sollen. Dies entspricht etwa einer flexiblen Abfrageschnittstelle, wie sie beispielsweise von SQL bekannt ist.  

Während der Analyse und Implementierung ergeben sich diverse Herausforderungen und Problemstellungen, welche es zu bewältigen gilt, seien es Heterogenität in der Datenrepräsentation der aufzurufenden Datenbankprozeduren des Fachbereiches, welche die Daten bereitstellen, oder seien es technische Problemstellungen wie z.B. bei der Generierung der Modellklassen zur Weiterverarbeitung der Anfrage im System oder fehlende Unterstützung von benutzen Typen aus der Oracle Datenbank in den eingesetzten Programmiersprachen und Frameworks.  
