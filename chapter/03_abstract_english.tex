\chapter*{Abstract}

\addcontentsline{toc}{chapter}{Abstract}

With the beginning of the 21st century and the global success of the internet in all households and businesses, collecting and processing data is more and more in the daily focus. Buzzwords like \enquote{Big Data}, \enquote{Cloud Services} und \enquote{Mobile Devices} are broadcasted through the internet and in the media. Data will be collected in nearly every area like Multimedia-data as pictures, audio or video data, or other data types like personal and product data. This data could for example be collected by Mobile devices as well as cars or trucks or through websites in the internet. They collect data in type of position, current speed or via sensors the temperature or barometric pressure. Websites collect and analyse data regarding the user behavior on the website. Such data will be saved and further processed and the amount of data is enormous. 

When collecting such kind of massive amount of data, the quality of the data is not always very important. In this context with the term \enquote{quality} the precise conceptual description of data is meant. The data will be collected and later be analysed through time-cosuming processing. 
Considering master data in the industry, manufacturing and trade there can be observed data entities like customer, supplier, material, economic goods or simply personal data regarding employees. The quality of the data has a pretty high importance here. Companies create catalogues and descriptions of their data so that the information can be exchanged between several systems. It is very important to know precisely what the datasets mean in their context. A manufacturer for hexagon bolts, which are used to mount tires on cars for example needs exact information, description and definition of this particular product for its production division. Furthermore Sales needs the information as well to sell the product to interested customers and provide them with all information about the product. Last but not least, the customer needs precise information about the product, in his design and production department, as well for the parts list which defines the needed parts for his concrete product. 

Master data is needed in nearly every area, in supply chain, design and production as well as \glslink{PLM}{Product-Lifecycle-Management}. 

If the description of the data is not complete or not precise problems occure in the \glslink{PLM}{Product-Lifecycle-Management}. In worst case the data must be reviewed or enriched manually where normally an automated process would do the job. This is time consuming and leads to higher production costs.   

The main goal is to describe data in concepts, make those concepts available and create standards which allow to automate the data flow.  

The problem with automated product data transfer is that interfaces and schemas are fixed and unflexible. Modifications and updates on interfaces and schemas are time consuming and costly. However flexibility is pretty important nowadays as we must react fast and efficiently on changes in specifications. 
 
All current thesis work in the area of studies around \gls{PLIB} topics, which describes a data model for dictionaries and libraries, consider the above mentioned problems. To support solving the problems several ISO-standards regarding these topics will be considered and implemented. 

This thesis considers the problem regarding automated transfer of characteristic product data based on the \gls{PLIB} data model. The ISO-standard \enquote{ISO 29002-31 - Query for characteristic data}, which is a description of a flexible \glslink{Abfrageschnittstelle}{query- response interface}, will be implemented as a \gls{Webservice} due to this purpose. 

The ISO-standard \enquote{ISO 29002-31 - Query for characteristic data} is the main query-interface, which references \glslink{Ontologie}{ontologies} via unique \glslink{IRDI}{identifier} (IRDI).
The response contains the data of the properties of the requested product data. An ontology is needed for that as well which will be described in \enquote{ISO 29002-10 - Characteristic data exchange format}. It describes the data format of the data responded. The data source for the \gls{Webservice} is the PLIB-database, an implementation of a product database according to ISO 13584-42. 

Thus meaningful use cases are a conjunction of several Standards. 
A simple use case is for example the request for all properties of a product which by a unique \glslink{IRDI}{identifier}. The request will be sent via XML according to ISO 29002-31 and to a \glslink{Webservice}{webservice} and then processed. The response will be sent according to ISO 29002-10 as an XML file to the consumer. More complex cases will restrict the request to a specific value range of the requested properties or will select specific properties only. These will be then sent back as response. This \glslink{Abfrageschnittstelle}{query- response interface} correlates pretty much to a flexible \glslink{Abfrageschnittstelle}{query- response interface} like it is known by SQL. 

During the analysis and implementation several challenges and problems occured which must be handled. For example heterogenity in the data representation of the called database procedures which will deliver the requested data or technical problems with the generation of the model classes for further processing in the system. Another problem is that a specific data type used in the procedures of the Oracle database has no support in the used programming language and framework.  
 
