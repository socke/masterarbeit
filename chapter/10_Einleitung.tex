
\chapter{Einleitung}\label{sec:einleitung}

Die Aufgabenstellung dieser Arbeit sowie die Zielsetzung ist in \autoref{kap:aufgabenbeschreibung} beschrieben. Um aufzuzeigen in welchem Kontext sich diese Arbeit befindet sowie zum Zwecke der Abgrenzung der Aufgabe, wird eine Übersicht aller PLIB Abschlussarbeiten gegeben. Das Kapitel geht anschließend auf die technischen und fachlichen Vorgaben des Fachbereiches ein und gibt eine grobe technische Beschreibung der Anforderungen.

Im Rahmen einer Analyse der Anforderungen wird Aufgabenstellung in \autoref{kap:analyse_und_definition} präzisiert. Die Aufgabenstellung des Fachbereiches stellt sich in Form einer abstrakte Anforderungsbeschreibung dar, welche noch durch eine konkrete Analyse der in Frage kommenden ISO Standards bestätigt, respektive erweitert werden muss. Das Ergebnis dieses Kapitels sind Anwendungsfälle mit entsprechenden Query-Beispielen. 

Im nächsten Schritt geht es darum, das System zu entwerfen. Die einzelnen Schritte des System- und Softwareentwurfes werden in \autoref{kap:systemundsoftwarentwurf} erläutert. Dazu gehört eine Beschreibung des Auswahlprozesses der notwendigen Programmiersprache, technische Plattform, Frameworks und Architekturmuster. 
Ferner wird in diesem Kapitel der Architekturentwurf vorgestellt und erläutert. Dabei werden die Komponenten des Systems in Diagrammen dargestellt und weiter verfeinert, so dass eine Übersicht über das Gesamtsystem bis hin zu einzelnen fachlichen Verantwortlichkeiten gegeben wird. 

Das \autoref{kap:implementierung} beschreibt die Implementierung. Die entwickelte Lösung und die Umsetzung wird beschrieben. Eine besonderes Augenmerk richtet sich auf einzelne konkrete Problemstellungen, die sich vor und während der Implementierung ergeben. Die besonderen Herausforderungen, wie Integration und der Nutzen der vorhandenen Prozedurschnittstelle, welche vorgegeben wurde, bekommt eine genauere Betrachtung, das dies einer der Hauptaspekte dieser Arbeit darstellt.

Es folgt das Fazit im Schlussteil. Dieses Kapitel greift das Ergebnis dieser Arbeit auf und zeigt, welche weitere Möglichkeiten und Untersuchungen sich aus den Erkenntnissen dieser Arbeit ergeben. Es wird der Gesamtkontext der Arbeiten im Fachbereich aufgegriffen und ein Gesamtbild erstellt, um die einzelnen Teile und Arbeiten zusammenfügen zu können. 