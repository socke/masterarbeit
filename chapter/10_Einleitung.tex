
\chapter{Einleitung}\label{sec:einleitung}


Eine Beschreibung der Aufgabenstellung dieser Arbeit sowie die konkrete Zielsetzung ist Teil von Kapitel zwei. Um aufzuzeigen, in welchem Kontext sich diese Arbeit befindet sowie zum Zwecke der Abgrenzung wird eine Übersicht aller PLIB Abschlussarbeiten gegeben. Das Kapitel geht kurz auf die Vorgaben des Fachbereiches ein und gibt eine grobe technische Beschreibung der Anforderungen.

Die Aufgabenstellung wird in Kapitel 3 im Rahmen einer Analyse der Anforderungen präzisiert. Die Aufgabenstellung des Fachbereiches beinhaltete grobe Vorstellungen, welche noch durch eine konkrete Analyse der in Frage kommenden ISO Standards bestätigt respektive erweitert werden muss. Ergebnis dieses Kapitels sind Anwendungsfälle mit entsprechenden Query-Beispielen. 

Im nächsten Schritt geht es darum, das eigentliche System zu entwerfen. Da die Aufgabenstellung lautet eine Schnittstelle zu entwerfen, beschreibt Kapitel 4 den System- und Softwareentwurf. Dazu gehört ebenso eine Beschreibung des Auswahlprozesses der notwendigen Programmiersprache, technischen Plattform, Frameworks und Architekturmuster. 
Ferner wird in diesem Kapitel der Architekturentwurf vorgestellt und erläutert. Dabei werden die Komponenten des Systems in Diagrammen dargestellt und weiter verfeinert, so dass eine Übersicht über das Gesamtsystem bis hin zu einzelnen fachlichen Verantwortlichkeiten sichtbar wird. 

Kapitel 5 beschreibt im Detail die Implementierung. Hierbei wird zum einen die entwickelte Lösung nächer beschrieben, als auch auf einzelne konkrete Problemstellungen eingegangen. Besondere Herausforderungen, wie Integration und der Nutzen der vorhandenen Prozedurschnittstelle, welche vorgegeben wurde, bekommt eine genauere Betrachtung, das dies einer der Hauptaspekte dieser Arbeit darstellt. 

Es folgt das Fazit im Schlussteil. Das letzte Kapitel greift das Ergebnis dieser Arbeit auf und zeigt, welche weitere Möglichkeiten und Untersuchungen sich aus den Erkenntnissen dieser Arbeit ergeben. Es wird der Gesamtkontext der Arbeiten im Fachbereich aufgegriffen und ein Gesamtbild erstellt, um die einzelnen Teile und Arbeiten zusammenfügen zu können.     
 

		