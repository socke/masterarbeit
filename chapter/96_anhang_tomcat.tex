\chapter{Installationsanleitung Apache Tomcat 7}\label{kap:anhangtomcat}

\section{Installation auf Mac OS 10.8}

Diese Installationsanleitung bezieht sich auf die Installation des Apache Tomcat 7 in Mac OS 10.8. 

\subsection{Prüfen der Java Version}

Mittels 

\lstinline[basicstyle=\ttfamily\small\mdseries]{java -version}

prüfen, ob Version 1.6 oder 1.7 installiert ist. Falls nicht, muss Java vorher installiert werden. Dazu das Java Development Kit herunterladen und installieren: \href{http://www.oracle.com/technetwork/java/javase/overview/index.html}{http://www.oracle.com/technetwork/java/javase/overview/index.html}

Für den Mac ist das ggf. über die Apple-Webseite verfügbar. 

\subsection{Tomcat herunterladen und entpacken}

\begin{itemize}
\item Auf \href{https://tomcat.apache.org/download-70.cgi}{https://tomcat.apache.org/download-70.cgi} Tomcat herunterladen. Darauf achten eine Binary distribution herunterzuladen, z.B. apache-tomcat-7.0.42.tar.gz
\item Das Paket entpacken mittels tar -xvzf apache-tomcat-7.0.42.tar.gz
\end{itemize}

\begin{itemize}
\item Ein Verzeichnis unter \href{file:///usr/local}{/usr/local} erstellen, wo der Apache später laufen soll, danach die Dateien dort hinkopieren
\begin{itemize}
	\item sudo mkdir -p \href{file:///usr/local}{/usr/local}
	\item sudo mv \href{file:///Users/stefan/Downloads/apache-tomcat-7.0.42}{~/Downloads/apache-tomcat-7.0.42} \href{file:///usr/local}{/usr/local/}
\end{itemize}\item Einen symbolischen Link erstellen um später einfacherer zwischen Versionen umzuschalten:
\begin{itemize}
	\item sudo rm -f \href{file:///Library/Tomcat}{/Library/Tomcat}
	\item sudo ln -s \href{file:///usr/local/apache-tomcat-7.0.42}{/usr/local/apache-tomcat-7.0.42} \href{file:///Library/Tomcat}{/Library/Tomcat}
\end{itemize}\item Allen Apache Dateien den aktuellen User als Besitzer setzen und Rechte vergeben:
\begin{itemize}
	\item sudo chown -R \textless{}dein\_username\textgreater{} \href{file:///Library/Tomcat}{/Library/Tomcat}
	\item sudo chmod +x \href{file:///Library/Tomcat/bin/*.sh}{/Library/Tomcat/bin/*.sh}
\end{itemize}\end{itemize}

\subsection{Tomcat starten}

\href{file:///Library/Tomcat/bin/startup.sh}{/Library/Tomcat/bin/startup.sh}

\subsection{Tomcat stoppen}

\href{file:///Library/Tomcat/bin/shutdown.sh}{/Library/Tomcat/bin/shutdown.sh}

\section{Installation unter Windows}

Am besten wird der Tomcat mittels Installer installiert. Hier wird man durch eine grafische Benutzeroberfläche geführt und Apache Tomcat wird als Dienst in das System integriert. 

\section{Tomcat mittels Maven starten}

Um Tomcat mittels Maven zu starten, muss Tomcat als Plugin in der Maven \gls{pom} konfiguriert werden. Das \autoref{lst:tomcat_plugin} zeigt die Konfiguration für die \gls{pom}.

 \begin{lstlisting}[caption=Tomcat 7 Maven Plugin, language=XML, label=lst:tomcat_plugin]
<build>
  <plugins>
    <plugin>
      <groupId>org.apache.tomcat.maven</groupId>
      <artifactId>tomcat7-maven-plugin</artifactId>
    </plugin>
  </plugins>
</build>
 \end{lstlisting}  
 
Anschließend kann der Tomcat-Server mit folgendem Befehl gestartet werden:
 
\lstinline[basicstyle=\ttfamily\small\mdseries]{mvn tomcat7:run}
 

 

