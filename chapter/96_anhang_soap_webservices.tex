\chapter{SOAP Webservice} \index{Webservice!SOAP}\label{kap:anhang_soap_webservice}

Um die beiden unterschiedlichen Webservicearten vergleichen zu können, wurde evaluiert, wie mit der gleichen Plattform, Programmiersprache und den gleichen Schemadateien ein SOAP \gls{Webservice} implementiert werden kann. 

Genutzt wurde die in der Java-JDK beinhaltete Implementierung der JAX-WS-2.1-Spezifikation, somit muss keine weitere Bibliothek eingebunden werden. 

Die gesamte Business-Logik-Implementierung für den \gls{REST}-\gls{Webservice} kann genutzt werden. Für die Erstellung des \glspl{Webservice} sind folgende Schritte notwendig:

\begin{itemize}
\item Erstellung einer \gls{Webservice}-Klasse, welche die Business Logik aufruft
\item Erstellung eines \gls{Webservice}-Servers, welche den \gls{Webservice} bereitstellt
\end{itemize}

\section{Erstellung einer Webservice-Klasse}

Das \autoref{lst:soap_webservice_send_query} zeigt die Implementierung einer \gls{Webservice}-Klasse. 

Hierbei definiert @WebService diese Klasse mittels Annotation als Webservice gemäß JAX-WS Spezifikation.
 
@SOAPBinding(style=SOAPBinding.Style.DOCUMENT) definiert ein SOAP-Binding als Dokumenten Typ. Ein weiterer Typ ist RPC.
@WebMethod definiert die markierte Methode als \gls{Webservice}-Methode. 

Die Methode leitet den übergebenen query einfach an die Business Logik, der QueryPipe, zur weiteren Verarbeitung. Dies entspricht exakt dem Vorgehen in der query-Methode im \gls{REST}-\gls{Webservice}.  

Hinweis: In allen Quelltextbeispielen wird auf import- und package-Anweisungen verzichtet, um die Lesbarkeit zu erhöhen. Die Quelltexte liegen der Arbeit bei. 

\begin{lstlisting}[caption=SOAP Webservice Klasse zum Senden eines Queries, language=Java, label=lst:soap_webservice_send_query]
/**
 * SOAP Webservice for querying via ISO 29002-31 Schema
 */
@WebService
@SOAPBinding(style=SOAPBinding.Style.DOCUMENT)
public class QuerySOAPService {

    /** Query processor instance */
    private QueryPipe queryPipe;

    /** Application context for receiving beans */
    private ApplicationContext context;

    public QuerySOAPService() {
        this.context = initializeContext();
        this.queryPipe = getQueryPipe();
    }

    /**
     * Takes the query model and passes it to the query filter for further processing.
     * Retreives the result and returns it.
     * @param query the query
     * @return the catalogue response
     */
    @WebMethod
    public CatalogueType query(QueryType query) {
        CatalogueType catalogue = queryPipe.filter(query);
        return catalogue;
    }

    private ClassPathXmlApplicationContext initializeContext() {
        return new ClassPathXmlApplicationContext(
                "beans.xml");
    }

    private QueryPipe getQueryPipe() {
        QueryPipe queryPipe =  (QueryPipe) context.getBean("queryPipe");
        return queryPipe;
    }
}
 \end{lstlisting}  
 
\section{Erstellung eines Webservice Servers}

Das \autoref{lst:soap_webservice_server} zeigt die Implementierung eines beispielhaften Webservice-Servers. Dieser erzeugt eine Instanz der Klasse QuerySOAPService, der Klasse, welche die Methoden für den Webservice beinhaltet. Anschließend wird ein Endpoint definiert. Hier wird die spätere \gls{URL}, unter dem der Webservice zur Verfügung gestellt wird, definiert und der QuerySOAPService übergeben. 

Ferner wird zu Testzwecken, zum Stoppen des Services, ein Dialog-Fenster angezeigt.
 
\begin{lstlisting}[caption=SOAP Webservice Server, language=Java, label=lst:soap_webservice_server]
package de.feu.plib.webservice;

import org.apache.log4j.Logger;

import javax.swing.*;
import javax.xml.ws.Endpoint;

/**
 * Is responsible for creating the SOAP Webserver.
 */
public class QueryServer {
    /** Logger instance */
    static final Logger LOGGER = Logger.getLogger(StringServer.class);

    public static void main(String args[]) {
        LOGGER.info("Start query web service");
        QuerySOAPService queryService = new QuerySOAPService();
        Endpoint endpoint = Endpoint.publish(
                "http://localhost:8088/soap/query", queryService);
        LOGGER.info("Web service started");
        JOptionPane.showMessageDialog(null, "Server beenden");
        endpoint.stop();
        LOGGER.info("Web service started");
    }
}
 \end{lstlisting}   

\section{SOAP Beispielrequest und -response}

Das \autoref{lst:soap_query} zeigt einen Beispielrequest und \autoref{lst:soap_response} zeigt die entsprechende Antwort des Servers. 

\begin{lstlisting}[caption=Beispielanfrage SOAP Webservice query, language=XML, label=lst:soap_query]
 <soapenv:Envelope xmlns:soapenv="http://schemas.xmlsoap.org/soap/envelope/" xmlns:web="http://webservice.plib.feu.de/" xmlns:urn="urn:iso:std:iso:ts:29002:-31:ed-1:tech:xml-schema:query" xmlns:urn1="urn:iso:std:iso:ts:29002:-10:ed-1:tech:xml-schema:catalogue" xmlns:urn2="urn:iso:std:iso:ts:29002:-4:ed-1:tech:xml-schema:basic" xmlns:urn3="urn:iso:std:iso:ts:29002:-10:ed-1:tech:xml-schema:value">
   <soapenv:Header/>
   <soapenv:Body>
      <web:query>
         <arg0>
            <urn:class_ref>0173-1#01-BAD803#2</urn:class_ref>
	 </arg0>
      </web:query>
   </soapenv:Body>
</soapenv:Envelope>    
 \end{lstlisting}  


 \begin{lstlisting}[caption=Beispielantwort SOAP Webservice response, language=XML, label=lst:soap_response]
<S:Envelope xmlns:S="http://schemas.xmlsoap.org/soap/envelope/">
   <S:Body>
      <ns6:queryResponse xmlns:ns2="urn:iso:std:iso:ts:29002:-10:ed-1:tech:xml-schema:catalogue" xmlns:ns3="urn:iso:std:iso:ts:29002:-4:ed-1:tech:xml-schema:basic" xmlns:ns4="urn:iso:std:iso:ts:29002:-10:ed-1:tech:xml-schema:value" xmlns:ns5="urn:iso:std:iso:ts:29002:-31:ed-1:tech:xml-schema:query" xmlns:ns6="http://webservice.plib.feu.de/">
         <return>
            <ns2:item class_ref="0173-1#01-BAD803#2">
               <ns2:property_value property_ref="0173-1#02-AAA762#1">
                  <ns4:measure_single_number_value>
                     <ns4:real_value>2.0</ns4:real_value>
                  </ns4:measure_single_number_value>
               </ns2:property_value>
               <ns2:property_value property_ref="0173-1#02-AAB011#1">
                  <ns4:measure_single_number_value>
                     <ns4:real_value>1.0</ns4:real_value>
                  </ns4:measure_single_number_value>
               </ns2:property_value>
            </ns2:item>
            <ns2:item class_ref="0173-1#01-BAD803#2">
               <ns2:property_value property_ref="0173-1#02-AAA762#1">
                  <ns4:measure_single_number_value>
                     <ns4:real_value>2.0</ns4:real_value>
                  </ns4:measure_single_number_value>
               </ns2:property_value>
               <ns2:property_value property_ref="0173-1#02-AAB011#1">
                  <ns4:measure_single_number_value>
                     <ns4:real_value>1.0</ns4:real_value>
                  </ns4:measure_single_number_value>
               </ns2:property_value>
            </ns2:item>
         </return>
      </ns6:queryResponse>
   </S:Body>
</S:Envelope>  
 \end{lstlisting}  
 
 Die \autoref{lst:wsdl_query_service} zeigt die WSDL des Query Services. Diese wurde automatisch generiert. Das geschieht durch den Aufruf der URL des Services mit Query-Parameter ?wsdl. 
 
 Folgende URL wurde aufgerufen: http://localhost:8088/soap/query?wsdl
 
 Man sieht, dass die benötigten Schemadateien für query, basic, value und catalogue der ISO XSDs automatisch eingebunden werden. Der Grund ist, dass in den Java Sourcen wie in \autoref{lst:soap_webservice_send_query} ersichtlich, die bereits während der REST-Implementierung für das Marshalling und Unmarshalling erzeugten Zugriffsklassen mittels JAXB benutzt werden. Das sind CatalogueType und QueryType. Dies sind JAXB annotierte Klassen, somit werden automatisch die entsprechenden Schemadateien verwendet.
 
 \begin{lstlisting}[caption=WSDL des Query Services mit SOAP, language=XML, label=lst:wsdl_query_service] 
 <!--
 Published by JAX-WS RI at http://jax-ws.dev.java.net. RI's version is JAX-WS RI 2.1.6 in JDK 6.
-->
<!--
 Generated by JAX-WS RI at http://jax-ws.dev.java.net. RI's version is JAX-WS RI 2.1.6 in JDK 6.
-->
<definitions xmlns:soap="http://schemas.xmlsoap.org/wsdl/soap/" xmlns:tns="http://webservice.plib.feu.de/"
             xmlns:xsd="http://www.w3.org/2001/XMLSchema" xmlns="http://schemas.xmlsoap.org/wsdl/"
             targetNamespace="http://webservice.plib.feu.de/" name="QuerySOAPServiceService">
    <types>
        <xsd:schema>
            <xsd:import namespace="urn:iso:std:iso:ts:29002:-4:ed-1:tech:xml-schema:basic"
                        schemaLocation="http://localhost:8088/soap/query?xsd=1"/>
        </xsd:schema>
        <xsd:schema>
            <xsd:import namespace="urn:iso:std:iso:ts:29002:-31:ed-1:tech:xml-schema:query"
                        schemaLocation="http://localhost:8088/soap/query?xsd=2"/>
        </xsd:schema>
        <xsd:schema>
            <xsd:import namespace="urn:iso:std:iso:ts:29002:-10:ed-1:tech:xml-schema:value"
                        schemaLocation="http://localhost:8088/soap/query?xsd=3"/>
        </xsd:schema>
        <xsd:schema>
            <xsd:import namespace="urn:iso:std:iso:ts:29002:-10:ed-1:tech:xml-schema:catalogue"
                        schemaLocation="http://localhost:8088/soap/query?xsd=4"/>
        </xsd:schema>
        <xsd:schema>
            <xsd:import namespace="http://webservice.plib.feu.de/"
                        schemaLocation="http://localhost:8088/soap/query?xsd=5"/>
        </xsd:schema>
    </types>
    <message name="query">
        <part name="parameters" element="tns:query"/>
    </message>
    <message name="queryResponse">
        <part name="parameters" element="tns:queryResponse"/>
    </message>
    <portType name="QuerySOAPService">
        <operation name="query">
            <input message="tns:query"/>
            <output message="tns:queryResponse"/>
        </operation>
    </portType>
    <binding name="QuerySOAPServicePortBinding" type="tns:QuerySOAPService">
        <soap:binding transport="http://schemas.xmlsoap.org/soap/http" style="document"/>
        <operation name="query">
            <soap:operation soapAction=""/>
            <input>
                <soap:body use="literal"/>
            </input>
            <output>
                <soap:body use="literal"/>
            </output>
        </operation>
    </binding>
    <service name="QuerySOAPServiceService">
        <port name="QuerySOAPServicePort" binding="tns:QuerySOAPServicePortBinding">
            <soap:address location="http://localhost:8088/soap/query"/>
        </port>
    </service>
</definitions>
  \end{lstlisting} 
 
 \section{Fazit}
 
Da sämtliche Voraussetzungen bereits während der Implementierung des REST-Webservices in der Business Logik vorhanden waren, konnte in kurzer Zeit ein SOAP-Webservice erstellt werden, der die gleichen Query-Funktionalitäten zur Verfügung stellt wie der REST-Webservice. Das Generieren des XML-Modells in Java-Klassen wurde bereits mittels JAXB durchgeführt, sodass diese Modelle erneut benutzt werden können. Java kennt dann die Modelle und die Bindung an die entsprechenden XML-Schema-Dateien, sodass diese für den \gls{Webservice} zur Verfügung gestellt werden können. 
Es ist also ersichtlich, dass die Webservice Schnittstelle mit relativ geringem Aufwand ausgetauscht werden kann, wenn die Business Logik gekapselt wurde. 