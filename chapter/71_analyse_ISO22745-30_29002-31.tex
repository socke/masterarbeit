\section{Analyse ISO 29002-31 - Exchange of characteristics data}

Die ISO 29002-31 ermöglicht abstrakt folgende Arten von Abfragen:
\begin{itemize}
\item Liefern oder validieren der spezifizierten characteristic data für das mittels Identifier\footnote{IRDI} angegebene Element.
\item Liefern des Identifiers eines Elementes welches den übermittelten characteristic data (am ehesten) entspricht. 
\end{itemize}

\subsection{XML Datencontaineranalyse ISO 29002-31}
Die Unterkapitel beschreiben die einzelnen (XML) Datencontainer aus der ISO 29002-31. Der Ausgangspunkt ist der query\_context, welcher einige Metadaten zum eigentlichen query enthält. 

\subsubsection{query\_context}
Dies ist eine Art Container für eine Menge von Queries. Inhalt sind Informationen über den Anforderer der Daten zwecks persönlicher Kontaktaufnahme, wie z.B. die Anfragezeit, Informationen über die Organisation welche die Anfrage schickt sowie einen gewünschten Antwortzeitpunkt mit Antwort-Email Adresse. Siehe dazu \citep[Kap. 5.2.2][]{iso29002-31}

Da die Vorgabe lautet, den Service auf Basis eines Web Services zu erstellen, entfällt die Benutzung des query\_context. Der Kontext wird implizit durch den Web Service respektive dem Server zur Verfügung gestellt. Beispielsweise wird die Anfragezeit nicht explizit durch den Serviceaufrufer selbst übergeben, allerdings durch die Anfrage an den technischen Server wie z.B. Apache Tomcat Server mittels Logeintrag ermittelt. 
Siehe dazu auch \citep[Kap. 6][]{iso29002-31}, welche besagt: "ISO/TS 29002 can be implemented:
a. with another envelope standard, such as EDI, or
b. by itself, using the query\_context to carry envelope information."

\subsubsection{query}
Die Unterstützung aller Funktionalitäten des queries entspricht laut ISO 29002-31 Anhang 6 der Conformance class 1: simple query.
Dies ist der eigentliche Abfrage-Datensatz. Abgefragt werden kann mittels class IRDI, data\_specification IRDI\footnote{IRDI  - International registration data identifier}, eine Menge von property IRDI, Item Daten (das sind Items gefüllt mit Daten die bereits bekannt sind) und einer item\_description. Das bedeutet, dass bereits bekannt Properties eines Items übertragen werden können, um die Suche zu vereinfachen.

Der query beihaltet folglich auch die data\_specification IRDI. Diese verweist auf eine Spezifikation aus ISO 22754-30, die besagt welche Properties für dieses Item sinnvoll sind. Die angegebenen Property IRDIs sind dann eine Teilmenge aus den mittles data\_specification IRDI definierten erlaubten properties. 

Ad hoc denkbar wären einfache Abfragens wie z.B.: "Gib mir alle Items der Klasse xyz". Mitgeliefert werden auch Items von Subklassen. Weiterhin kann die Abfrage nach bestimmten Properties eingeschränkt werden. Eine weitere Möglichkeit ist es bereits bekannte Daten über ein Element zu übermitteln, mit dem Zwecke hierüber die IRDI zu erfahren oder weitere Property-Daten zu erhalten. Siehe Beispielqueries simple queries in Kapitel \ref{kap:query_beispiele}. 

\subsubsection{characteristic\_data\_query\_expression (parametric\_query)}
Das entspricht laut ISO 29002-31 Anhang 6 der Conformance class 2: parametric query.

Eine characteristic\_data\_query\_expression kann verschieden expressions vom Typ query\_expression beinhalten. Von jedem Typ jeweils nur maximal eine. 
Z.B.
\begin{itemize}
\item string\_size\_expression
\item string\_pattern\_expression
\item range\_expression
\item data\_environment\_expression
\item cardinality\_expression
\item subset\_expression
\end{itemize}
darüberhinaus noch folgende Attribute

\begin{itemize}
\item property\_reference - die property auf den die query\_expression bezogen ist
\end{itemize}
Solch eine Expression ermöglicht das Filtern, gleichsam ein Einschränken bestimmter Properties und Werte. 

\subsubsection{Query Beispiele}\label{kap:query_beispiele}

Eine Schraube hat die folgenden möglichen Properties: 

\begin{description}
\item[Klassen-Identifier] 1234-abcd\# ab-cdefgh\# 1 (IRDI)
\item[Typ] M6 (Property IRDI: 1234-abcd\# ab-bbbbbb\# 1)
\item[Länge] 80mm (Property IRDI: 1234-abcd\# ab-cccccc\# 1)
\end{description}

\paragraph{Simple Query}

Jetzt ermöglicht ein simpler query folgende Abfrage: "Gib mir alle Items zu der Klasse Kreuzschraube mit dem Identifier (IRDI) 1234-abcd\#ab-cdefgh\#1. Das Ergebnis ist ein Item, mit allen Attributen wie oben angegeben. 

Ein anderer Query könnte lauten: "Gib mir die Properties 1234-abcd\#ab-cccccc\#1 und 1234-abcd\#bbbbbb\# 1 des Items der Klasse 1234-abcd\#ab-cdefgh\#1". Das Ergebnis wäre das Item mit Typ: M6 und die Länge: 80mm.

Es könnte auch mit Hilfe von vorhandenen Daten gesucht werden, z.B.: "Hier ist ein Item mit der Property Typ: M6 (Property IRDI: 1234-abcd\# ab-bbbbbb\# 1), gibt mir bitte dazu die Properties 1234-abcd\#ab-cccccc\#1 und 1234-abcd\# bbbbbb\#1 

\paragraph{Parametric Query}

Hat man jetzt noch eine Schraube mit folgenden Eigenschaften:
\begin{description}
\item[Klassen-Identifier] 1234-abcd\#ab-cdefgh\#1 (IRDI)
\item[Typ] M5 (Property IRDI: 1234-abcd\#xx-bbbbbb\#1)
\item[Länge] 100mm (Property IRDI: 1234-abcd\#xx-cccccc\#1)
\end{description}


Mit Hilfe der characteristic\_data\_query\_expression sind folgende Abfragen möglich: "Gib mir die Properties 1234-abcd\# ab-cccccc\#1 (Länge) und 1234-abcd\#bbbbbb\#1 (Typ) der Klasse Klasse 1234-abcd\#ab-cdefgh\#1 (Schraube) wo die Länge zwischen 50 und 150mm und der Typ M5 oder M6 sein soll."


Diese ermöglicht das filtern auf genau eine übergebene Property. Rekursive Abfragen sind auch möglich, beispielsweise wenn die gesuchte Property eine Multi-Property ist (Property: Loch als Wert zwei Properties mit Form und Durchmesser und Durchmesser soll gefiltert werden)

\subsection{Analyse ISO 22745-30 - Identification Guide}\label{kap:identification_guide}

Ein Identification Guide beschreibt welche Daten für ein Objekt benötigt werden, damit dies überhaupt sinnvoll für einen bestimmten Zweck eingesetzt werden kann. Der Käufer, Produktmanager oder Benutzer definiert die Anforderungen an die Daten. Ein "Datenanforderungsstatement" wird als ein eOTD-i-xml identification guide xml file erzeugt. Es wird definiert, was der Name des Artikels ist mit den charakteristischen Daten. Es wird die Frage beantwortet, welche Daten (Properties) zu einer bestimmten Klasse eines Objektes benötigt wird um den Artikeln zu kaufen oder zu sinnvoll zu verwalten. Diese Anforderungen werden von der Abfrageseite (Kundenseite) definiert, also derjenige, der Daten abfragen möchte. \(Quelle: ECCMA\_ISO\_8000\_certification.pdf\)
Ein Identification guide referenziert Konzepte eines Dictionaries um Datenanforderungen einer bestimmten Klasse zu beschreiben. (ISO 22745-30 Kapitel 5). 
Ein Datenempfänger kann eine Organisation oder eine Gruppe von Organisationen oder Firmen sein, welche ähnliche Datenanforderungen haben. Somit wird ein Identification Guide Gruppe von einer speziellen Organisation verwaltet, welche wiederum selbst Datenempfänger sein kann.  

