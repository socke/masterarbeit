\newglossaryentry{Webservice}{name=Webservice, description={Oft auch als Web Service oder web service (Englisch) geschrieben. Stellt einen Web Dienst dar}}

\newglossaryentry{SOAP}{name=SOAP, description={Simple Object Access Protocol. Protokoll für Nachrichten, diese werden zwischen Webservice-Konsument und Webservice-Anbieter ausgetauscht}}

\newglossaryentry{WSDL}{name=WSDL, description={Web Service Description Language. W3C-Standard für die Beschreibung des Services und der Daten, die zwischen Konsument und Anbieter ausgetauscht werden}}

\newglossaryentry{Spring}{name=Spring, description={Ein nicht invasives Open Source Applikationsframework mit dem Ziel, die Softwareentwicklung zu vereinfachen}}

\newglossaryentry{JSF2.0}{name=JSF2.0, description={Java Server Faces in der Version 2.0 - Das ist ein komponentenbasiertes Framework zur Entwicklung von Weboberflächen}}

\newglossaryentry{Jersey}{name=Jersey, description={Framework zur Erstellung von RESTful Web Services in Java}}

\newglossaryentry{Apache Tomcat}{name=Apache Tomcat, description={Web Container zum Ausführen und Ausliefern von Webseiten basierend auf Java}}

\newglossaryentry{REST}{name=REST, description={Representational State Transfer. Ein Architekturmodell basierend auf dem HTTP-Protokoll. Häufig genannt in Verwendung mit RESTful Webservices}}

\newglossaryentry{PLIB}{name=PLIB, description={Parts Library gemäß ISO 13584-42, beschreibt ein Datenmodell für Dictionaries und Bibliotheken }}

\newglossaryentry{Annotation}{name=Annotation, description={Metainformation in der Java Programmierung. Wird zur Reflektion des Codes genutzt}}

\newglossaryentry{HTTP-Methode}{name=HTTP-Methode, description={Eine HTTP-Anfrage an einen Server, gemäß des Standards, z.B. eine GET-, POST-, PUT- oder DELETE Anfrage.  Es werden dadurch Ressourcen auf einem Server abgefragt, geändert oder gelöscht}}

\newglossaryentry{HTTP}{name=HTTP, description={Hyper Transfer Protocol. Protokoll zur Übertragung von Daten über ein Netzwerk. Wird in der Anwendungsschicht angesiedelt und ist das hauptsächlich eingesetzte Protokoll, um Webseiten im Internet (World Wide Web) zu übertragen}}

\newglossaryentry{MIME-Type}{name=MIME-Type, description={Internet Media Type, wird auch Content-Type genannt. Klassifiziert die Daten im Kopfbereich einer Nachricht. Hiermit wird dem Empfänger mitgeteilt, welche Art der Daten gesendet werden. Das können beispielsweise reine Textdaten, Bild-, XML- oder Videodaten sein}}

\newglossaryentry{Namespace}{name=Namespace, description={XML-Namensräume werden benutzt, um mehrere verschiedene Vokabulare in einer XML-Datei zu unterscheiden}}

\newglossaryentry{JAXB}{name=JAXB, description={Java Architecture for XML Binding, Programmierschnittstelle, die Daten aus XML-Schemata an Java-Klassen bindet}}

\newglossaryentry{pom}{name=pom.xml, description={Project Object Model - Konfigurationsdatei eines Maven Projektes}}

\newglossaryentry{Maven}{name=Maven, description={Ein Build-Management-Tool der Apache-Foundation. Wurde mit Java entwickelt und ermöglicht, Java-Programme standardisiert zu erstellen und zu verwalten}}

\newglossaryentry{Unmarshalling}{name=Unmarshalling, description={Beschreibt Umwandeln von XML Daten in programmatisches Modell, z.B. in Java Klassen}}

\newglossaryentry{Marshalling}{name=Marshalling, description={Beschreibt das Umwandeln von einem programmatischen Modell, z.B. einer Java Klasse, in XML Daten}}

\newglossaryentry{URL}{name=URL, description={Uniform Resource Location, identifiziert und lokalisiert eine Ressource, wie beispielsweise eine Webseite oder ein Bild über die verwendeten Netzwerkprotokolle}}

\newglossaryentry{URI}{name=URI, description={Uniform Resource Identifier, identifiziert eine abstrakte oder physische Ressource wie z.B. eine Webseite, ein E-Mail Empfänger oder eine Person}}

\newglossaryentry{VCS}{name=Versionskontrollsystem, description={Wird zur Versionswerwaltung eingesetzt. Erfasst Änderungen an Dokumenten oder Dateien, um diese später wiederherstellen zu können. Unterstützt häufig das gemeinsame Bearbeiten von Dateien und Dokumenten}}

\newglossaryentry{IRDI}{name=IRDI, description={International Registration Data Identifier, global eindeutiger Identifizierer für ein Objekt oder Konzept}}

\newglossaryentry{item}{name=Teil, description={Mit Teil, (englisch) Item oder auch Instanz genannt, ist eine Instanz eines konkreten Konzeptes der Teiledatenbank gemeint. Dieses Teil referenziert das Konzept und enthält konkrete Datensätze samt Wert, z.B. könnte es den Wert einer Eigenschaft eines Konzeptes enthalten}}

\newglossaryentry{Servlet}{name=Servlet, description={Servlets sind Java Klassen, welche nur innerhalb eines Webservers laufen. Diese nehmen Anfragen von Clients entgegen und beantworten sie}}

\newglossaryentry{ECCMA}{name=ECCMA, description={Electronic Commerce Code Management Association ist eine internationale nicht gewinnorientierte Organisation, mit dem Ziel bessere Qualitätsstandards für elektronische Daten zu erforschen, zu entwickeln und zu verbreiten}}

\newglossaryentry{Oracle}{name=Oracle, description={Oracle Corporation, einer der weltweit größten Softwarehersteller. Bekanntestes Produkt ist das Datenbankmanagementsystem Oracle Database. Oracle übernahm im Jahre 2009 Sun Microsystems und somit die Entwicklung der Java Programmiersprache}}

\newglossaryentry{Use Case}{name=Use Case, description={Use Case oder auch Anwendungsfall, beschreibt meist in Textform das Verhalten eines bestimmten Systems in Interaktion mit einem Benutzer in verschiedenen Szenarien}}

\newglossaryentry{POST}{name=POST, description={Eine HTTP Anfrage Methode, POST wird für das Übertragen von theoretisch unbegrenzter Menge an Daten an einen Server eingesetzt}}

\newglossaryentry{Stakeholder}{name=Stakeholder, description={Stakeholder sind Projektbeteiligte, gleichsam Personen oder Institutionen und Dokumente, die in irgendeiner Weise vom Betrieb des Systems betroffen sind}}

\newglossaryentry{Ontologie}{name=Ontologie, description={Der Begriff Ontologie bezeichnet das Studium der Kategorisierung von Dingen, die existieren oder in einigen Interessensgebieten bestehen können. Das Ergebnis einer solchen Studie, eine Ontologie, ist ein Katalog von Typen von Dingen D, von denen man annimmt, dass sie aus der Sicht einer Person, die eine Sprache L benutzt, um über diesen Interessensbereich D zu sprechen, existieren}}

\newglossaryentry{Interoperabel}{name=Interoperabilität, description={Die Fähigkeit homogener Systeme Informationen effizient auszutauschen}}

\newglossaryentry{PLM}{name=Produktlebenszyklusmanagement, description={Auch Product-Lifecycle-Management,  ist ein Konzept zur nahtlosen Integration sämtlicher Informationen, die im Verlauf des Lebenszyklus eines Produktes anfallen}}

\newglossaryentry{Abfrageschnittstelle}{name=Abfrage- und Antwortschnittstelle, description={Auch Query-Response-Schnittstelle. Eine flexible Schnittstelle über die vom Klienten eine Anfragenachricht an einen Server gesendet wird, welcher eine Antwortnachricht zurückliefert. Die Anfragen sind flexibel dadurch, dass Werte- und Attributeinschränkungen gemacht werden können}}

\newglossaryentry{IG}{name=Identification Guide, description={Ein Identification Guide beschreibt eine Menge von Regeln für die Beschreibung von Teilen (Items), welche zu einer bestimmten Klasse gehören. Hierbei werden die Eigenschaften und Klassendefinition zu Konzepten eines Dictionaries verlinkt}}

\newglossaryentry{JUnit}{name=JUnit, description={Ein Framework basierend auf Java, um Unit Tests zu entwickeln und auszuführen}}

\newglossaryentry{Unit-Test}{name=Unit Test, description={Ein Unit Test ist ein Test der Funktionalität einer Komponente ohne Einbindung der Abhängigkeiten von anderen Komponenten. Dieser Test ist somit ein Whiteboxtest, da die Komponente völlig autark auf Funktionalität getestet wird}}

\newglossaryentry{Integrationstest}{name=Integrationstest, description={Ein Integrationstest ist ein Test einer Komponente integriert in die System- oder Komponentenlandschaft. Es wird die Komponente in Zusammenarbeit mit anderen Komponenten im Rahmen von Szenarios getestet. Z.B. Test einer grafischen Benutzeroberfläche mit Speicherung von Eingabedaten in der Datenbank}}

\newglossaryentry{Applikationskontext}{name=Applikationskontext, description={Auch application context. Die Ausführungsumgebung einer Software Applikation innhalb eines Applikationsservers. Wird an eine URL gekoppelt, unter der die Applikation erreichbar ist}}

\newglossaryentry{Dictionary}{name=Dictionary, description={Ein Verzeichnis, welches Daten zu Konzepten enthält. Diese Konzepte werden mittels Eigenschaften präzise beschrieben, sodass im besten Falle die Beschreibung eindeutig ist}}

\newglossaryentry{Excel}{name=Excel, description={Microsoft Excel ist ein bekanntes Tabellenkalkulationsprogramm der Firma Microsoft.}}



