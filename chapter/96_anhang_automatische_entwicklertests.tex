\chapter{Automatische Entwicklertests} \index{JUnit}\label{anh:automatischeentwicklertests}
 
\section{Unit Test}\index{Unit Test}
Nachfolgend ein Beispiel eines Unit Tests. Das Testobjekt ist die Klasse XMLMarshaller, welcher für das \gls{Marshalling} und \gls{Unmarshalling} der XML-Daten verantwortlich ist. Hinweis: Alle Quelltexte verzichten auf import- und package-Anweisungen um die Lesbarkeit zu erhöhen. Die Original Quelltexte liegen der Arbeit bei. 
 
\begin{lstlisting}[caption=Beispiel eines Unit Tests, language=XML, label=lst:unittest_beispiel]
package de.feu.plib.xml;

/**
 * Tests the marshalling and unmarshalling of xml files.
 */
public class XMLMarshallerTest extends AbstractXMLTest {

    /** XML Marshaller instance under test */
    XMLMarshaller marshaller;

    /** Logger instance */
    private static final Logger LOGGER = Logger.getLogger(XMLMarshallerTest.class);

    /**
     * Simple marshalling test with arbitrary catalogue item.
     */
    @Test
    public void testMarshallingWithValidArbitraryCatalogue() {
        String catalogue = marshaller.marshall(createCatalogueWithOneItem());
        LOGGER.info(catalogue);
        assertTrue(catalogue.contains("0173-1#01-AAA352#4"));
        assertTrue(catalogue.contains("true"));
    }

    /**
     * Simple test with unmarshalling an arbitrary item from xml.
     */
    @Test
    public void testUnMarshallingWithValidArbitraryClassIrdi() {
        QueryType queryType = marshaller.unmarshallXML(readXMLFrom("/de/feu/plib/xml/query_class_irdi.xml"),
                QueryType.class);
        assertEquals("0173-1#01-BAD803#2", queryType.getClassRef());
    }

    /**
     * Throws an exception while parsing on illegal irdi passed.
     *
     * @throws Exception on illegal irdi
     */
    @Test(expected = RuntimeException.class)
    public void shouldThrowExceptionDuringXMLValidationWithIllegalIrdi() throws Exception {
        QueryType queryType = marshaller.unmarshallXML(readXMLFrom("/de/feu/plib/xml/query_class_irdi_illegal.xml"),
                QueryType.class);
    }

    /**
     * creates a sample catalogue for testing
     * 
     * @return the sample catalogue with one item
     */
    private CatalogueType createCatalogueWithOneItem() {
        ItemType item = new ItemType();
        item.setClassRef("0173-1#01-AAA352#4");
        PropertyValueType propertyValueType = new PropertyValueType();

        BooleanValueType bvt = new BooleanValueType();
        bvt.setValue(true);
        propertyValueType.setBooleanValue(bvt);
        propertyValueType.setPropertyRef("0173-1#01-A35AA2#4");
        item.getPropertyValue().add(propertyValueType);
        CatalogueType catalogue = new CatalogueType();
        catalogue.getItem().add(item);

        return catalogue;
    }

    /**
     * @throws java.lang.Exception
     */
    @Before
    public void setUp() {
        marshaller = new XMLMarshallerImpl();
    }

    /**
     * @throws java.lang.Exception
     */
    @After
    public void tearDown() {
        marshaller = null;
    }

}
\end{lstlisting}  

Das \autoref{lst:abstrakte_unit_testklasse} zeigt die abstrakte Klasse \enquote{AbstractXMLTest}, welche Hilfsmethoden für XML-Tests kapselt. Hier kapselt diese Klasse die Funktionalität XML-Dateien einzulesen.  

\begin{lstlisting}[caption=Abstrakte Unit Testklasse, language=Java, label=lst:abstrakte_unit_testklasse]

/**
 * Abstract test class. Extend from this class to get functionality to read XML files for your test.
 */
public class AbstractXMLTest {

    protected XMLMarshaller marshaller;

    /**
     * Logger instance
     */
    private static Logger LOGGER = Logger.getLogger(AbstractXMLTest.class);

    @Before
    public void setUp() {
        marshaller = new XMLMarshallerImpl();
    }

    @After
    public void tearDown() {
        marshaller = null;
    }

    /**
     * Reads the xml file from given filename
     *
     * @param filename the filename of the xml
     * @return the string content of the xml file
     */
    protected String readXMLFrom(String filename) {
        BufferedReader br = null;
        StringBuffer sb = new StringBuffer();

        try {
            String currentLine;

            InputStream is = XMLMarshallerImpl.class.getResourceAsStream(filename);
            br = new BufferedReader(new InputStreamReader(is));

            while ((currentLine = br.readLine()) != null) {
                sb.append(currentLine);
            }

        } catch (IOException e) {
            e.printStackTrace();
        } finally {
            try {
                if (br != null) br.close();
            } catch (IOException ex) {
                LOGGER.info("Exception occured during reading file: " + ex);
            }
        }

        return sb.toString();
    }
}
\end{lstlisting}  

\section{Integrationstest}\index{Integrationstest}

Dieser Abschnitt zeigt beispielhaft einen Integrationstest auf. Der Integrationstest unterscheidet sich vom Unit-Test darin, dass eine Komponente integriert mit mehreren abhängigen Komponenten getestet wird. 
Der nachfolgende Integrationstestfall \enquote{PlibDaoIT} aus \autoref{lst:integrationstest_beispiel} testet die Klasse \enquote{PlibDao}. Es ist ein Integrationstest, da für den Test die gesamte Datenbank benötigt wird. Nur so kann sinnvoll das Datenzugriffsobjekt getestet werden. 

Gestartet werden kann der Integrationstest mit Maven mittles Aufruf von 

\lstinline[basicstyle=\ttfamily\small\mdseries]{mvn integration-test}

\begin{lstlisting}[caption=Beispiel eines Integrationstests, language=Java, label=lst:integrationstest_beispiel]

/**
 * Integration Test of PLIB Dao
 */
@RunWith(SpringJUnit4ClassRunner.class)
@ContextConfiguration(locations = {"/beans_for_tests.xml"})
public class PlibDaoIT {

    /**
     * Logger instance
     */
    private static Logger LOGGER = Logger.getLogger(PlibDaoIT.class);

    @Autowired
    private PlibDao plib;

    @Test
    public void shouldReturnTrueWithExistingIRDI() {
        Irdi irdi = createMultiListTestIrdi();
        assertTrue(plib.doObjectsExistsWithThis(irdi));
    }

    @Test
    public void shouldReturnFalseWithIRDINotinDB() {
        Irdi irdi = createNonExistingTestIrdi();
        assertFalse(plib.doObjectsExistsWithThis(irdi));
    }

    @Test
    public void shouldReturnFalseWithEmptyIRDI() {
        Irdi irdi = new Irdi() {
            @Override
            public String getIrdi() {
                return "";
            }
        };
        assertFalse(plib.doObjectsExistsWithThis(irdi));
    }

    @Test
    public void testGetNumberOfObjectsOfIrdiExistingIrdi() {
        Irdi irdi = createMultiListTestIrdi();
        assertEquals(8, plib.getNumberOfObjectsOfIrdi(irdi));
    }

    @Test
    public void testGetNumberOfObjectsOfIrdiNotExistingIrdi() {
        Irdi irdi = createNonExistingTestIrdi();
        assertEquals(0, plib.getNumberOfObjectsOfIrdi(irdi));
    }

    @Test
    public void testThatWhenReadExternalProductIdsByIrdiMustReturnSome() throws Exception {
        Irdi irdi = createMultiListTestIrdi();
        List<BigDecimal> productIds = plib.readExternalProductIdsBy(irdi);
        assertNotNull(productIds);
        assertEquals(8, productIds.size());
    }

    /**
     * Currently there are two instances in the database, seems that these are duplicates but not sure.
     */
    @Test
    public void shouldReturnOneTestExternalIdWithTestIrdi() {
        Irdi irdi = createSkalpellIrdi();
        List<BigDecimal> externalProductIds = plib.readExternalProductIdsBy(irdi);
        assertNotNull(externalProductIds);
        assertEquals(2, externalProductIds.size());
        LOGGER.info("external product ids: " + externalProductIds.get(0));
        assertEquals(new BigDecimal("300000001"), externalProductIds.get(0));
    }

    /**
     * This is a bigger integration test.
     * <ul>
     *     <li>First create an irdi instance</li>
     *     <li>create an enriched query</li>
     *     <li>Then load the objects from the database with its properties</li>
     *     <li>There should be the same number ob objects than with the previous check</li>
     * </ul>
     */
    @Test
    @Ignore("loadObjectsFrom is obsolte, maybe later reimplemented")
    public void shouldReturnOneInstanceOfSkalpellWithTwoProperties() {
        Irdi skalpellIrdi = createSkalpellIrdi();

        List<BigDecimal> externalProductIds = plib.readExternalProductIdsBy(skalpellIrdi);

        EnrichedQuery query = createEnrichedQueryFrom(skalpellIrdi);
        query.setType(QueryKind.SIMPLE);
        CatalogueType catalogueType = plib.loadObjectsFrom(query);
        List<ItemType> itemTypes = catalogueType.getItem();
        assertEquals(externalProductIds.size(), itemTypes.size());
        assertEquals(1, itemTypes.get(0).getPropertyValue().size());
    }

    /**
     * Test should return two items where the first one would be checked.
     * Should have two properties.
     */
    @Test
    public void testLoadPropertiesByExternalIds() {
        List<BigDecimal> externalIds = new ArrayList<BigDecimal>();
        BigDecimal bigDecimal = new BigDecimal("300000001");

        externalIds.add(bigDecimal);
        List<List<Map<String, Object>>> valueTypeList = plib.loadStringPropertiesByExternalIds(externalIds);
        LOGGER.info("valuetype list: " + valueTypeList);
        assertEquals("should be one instance", 1, valueTypeList.size());
        assertEquals("should be two properties",2, valueTypeList.get(0).size());
        assertThatIrdiAndValueAreAvailable(valueTypeList.get(0).get(0).entrySet(), "0173-1#02-AAA762#1");
        assertThatIrdiAndValueAreAvailable(valueTypeList.get(0).get(1).entrySet(), "0173-1#02-AAB011#1");
    }

    private void assertThatIrdiAndValueAreAvailable(Set<Map.Entry<String, Object>> entrySet, String knownIRDI) {

        Set<Map.Entry<String, Object>> entries = entrySet;
        boolean irdiFound = false;
        boolean valueFound = false;
        for (Map.Entry<String, Object> entry : entries) {
            LOGGER.info("key of property: " + entry.getKey());
            LOGGER.info("value of property: " + entry.getValue());
            if ("IRDI".equals(entry.getKey()) && null != entry.getValue() && !"null".equals(entry.getValue())) {
                irdiFound = true;
                assertEquals(knownIRDI, entry.getValue());
            }
            if ("VALUE".equals(entry.getKey()) && null != entry.getValue() && !"null".equals(entry.getValue())) {
                valueFound = true;
            }

        }
        assertTrue(irdiFound && valueFound);
    }

    /**
     * Load data type and unit for skalpell length property which should be mm and mm as well.
     * Property id: 300903090000033914 and 300903090000034450
     */
    @Test
    public void testLoadDataTypeAndUnitForAPropertyById() {
        List<Map<String, Object>> propertyTypeAndUnit = plib.loadTypeAndUnitOfPropertyBy("300903090000033914");
        LOGGER.info("property size: " + propertyTypeAndUnit.size());
        assertTrue(propertyTypeAndUnit.size() == 1);

        assertThatUnitAndSubTypeAreAvailable(propertyTypeAndUnit);

    }

    private void assertThatUnitAndSubTypeAreAvailable(List<Map<String, Object>> propertyTypeAndUnit) {
        Set<Map.Entry<String,Object>> entries = propertyTypeAndUnit.get(0).entrySet();

        /*
        * we need to assure that we have the unit (in this case SYMBOL, e.g. mm or m or cm)
        * in the values as well as the type of the value (in this case real_measure_type)
        */
        String property_unit_symbol = "SYMBOL";
        String property_type = "SUB_TYPE";
        boolean unitfound = false;
        boolean typefound = false;
        for (Map.Entry<String, Object> entry : entries) {
            LOGGER.info("property key: " + entry.getKey());
            LOGGER.info("property value: " + entry.getValue());
            if ("SYMBOL".equals(entry.getKey()) && null != entry.getValue() && !"null".equals(entry.getValue())) {
                unitfound = true;
            }
            if ("SUB_TYPE".equals(entry.getKey()) && null != entry.getValue() && !"null".equals(entry.getValue())) {
                typefound = true;
            }
        }
        assertTrue(unitfound && typefound);
    }

    private EnrichedQuery createEnrichedQueryFrom(Irdi skalpellIrdi) {
        QueryType queryType = new QueryType();
        queryType.setClassRef(skalpellIrdi.getIrdi());
        return new EnrichedQuery(queryType);
    }

    private Irdi createNonExistingTestIrdi() {
        return new Irdi() {
            @Override
            public String getIrdi() {
                return "0141-1#01-xxx#1";
            }
        };
    }

    private Irdi createMultiListTestIrdi() {
        return new Irdi() {
            @Override
            public String getIrdi() {
                return "0141-1#01-UKU1#1";
            }
        };
    }

    /**
     * Creates the irdi of an item which was created by me as testdata.
     * It is a Skalpell (PREFERRED_NAME of class in DB)
     * @return the irdi of the test item skalpell.
     */
    private Irdi createSkalpellIrdi() {
        return new Irdi() {
            @Override
            public String getIrdi() {
                return "0173-1#01-BAD803#2";
            }
        };
    }
}

\end{lstlisting}  
   